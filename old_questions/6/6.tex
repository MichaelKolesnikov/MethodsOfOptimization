\documentclass[17pt]{extarticle}
\usepackage{../mystyle}

\begin{document}

\section{Классы задач нелинейного программирования}

Задача \textbf{нелинейного} программирования:
Либо целевая функция нелинейна, либо есть нелинейное ограничение. Для таких задач отсутствует универсальный метод решения, поэтому выделяются более узкие классы задач, допускающих специальные методы решения.

\subsection{Задача квадратичного программирования}
Все ограничения линейны, а целевая функция квадратичная.

\subsection{Задача сепарабельного программирования}
Сепарабельная функция — это сумма функций одной переменной вида:
\[
    f(x) = \sum\limits_{j=1}^{n} \phi_j(x_j), \quad x = (x_1, x_2, \dots, x_n).
\]
Задача нелинейного программирования, в которой целевая функция и левые части ограничений — сепарабельные функции, является сепарабельной.

\subsection{Задача геометрического программирования}
Задача нелинейного программирования, в которой целевая функция и левые части ограничений — позиномы.

\subsubsection{Позином}
\[
    \sum\limits_{i=1}^{m} c_i p_i(x), \quad c_i > 0, \quad i = \overline{1,m},
\]
где \( p_i(x) \) — моном.

\subsubsection{Моном}
\[
    \prod\limits_{j=1}^{n} x_j^{a_{ij}}, \quad a_{ij} \in \mathbb{R}.
\]

\subsection{Задача кусочно-линейного программирования}
Путём преобразований можно свести к линейным, затем решаются соответствующими методами.

\end{document}