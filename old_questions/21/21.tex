\documentclass[17pt]{extarticle}
\usepackage{../mystyle}
\usepackage{graphicx} % Для вставки изображений

\begin{document}

\section{Алгоритм симплекс-метода}

1. Каноническая система ограничений \( A\vec{x} = \vec{b} \) приводится к виду, в котором базисные переменные выражены через свободные, а целевая функция к примерно тому же:
\[
    f(x) = \gamma_0 - (\gamma_{r+1}x_{r+1} + \dots + \gamma_n x_n)
\]
\[
    \begin{cases}
        x_1 = \beta_1 - (\alpha_{1 r+1}x_{r+1} + \dots + \alpha_{1n}x_n) \\
        x_i = \beta_i - (\alpha_{i r+1}x_{r+1} + \dots + \alpha_{in}x_n) \\
        x_r = \beta_r - (\alpha_{r r+1}x_{r+1} + \dots + \alpha_{rn}x_n) \\
    \end{cases}
\]

2. Заполняется симплекс-таблица:


Каждая строка таблицы соответствует уравнению, а последняя строка соответствует целевой функции.

3. В строке коэффициентов целевой функции (не считая \( \gamma_0 \)) выбирается \( \gamma_j < 0 \), максимальное по модулю.
Если все \( \gamma_j \ge 0 \), то оптимальное решение достигнуто, причём значения переменных определяются столбцом свободных членов, а оптимальное значение функции — клеткой, соответствующей свободному члену.

4. В \( j \)-ом столбце среди положительных коэффициентов \( \alpha_{ij} \) выбирается разрешающий элемент \( \alpha_{lj} \), т.е. элемент, для которого минимально отношение \( \frac{\beta_l}{\alpha_{lj}} \). Если положительных коэффициентов нет, то задача не имеет решений.

5. Делятся все члены строки, содержащей разрешающий элемент, на \( \alpha_{lj} \). Полученная строка вносится на то же место в новой таблице.

6. Из каждой оставшейся \( i \)-й (\( i \ne l \)) строки вычитается получившаяся строка, умноженная на коэффициент при \( x_j \) в \( l \)-й строке. В результате в клетках, соответствующих \( j \)-му столбцу, появляются нули. Преобразованные строки записываются в новой таблице на место прежних.

7. Переменная \( x_j \) вводится в базис вместо \( x_l \).

Далее идёт переход на шаг 3.

\end{document}