\documentclass[17pt]{extarticle}
\usepackage{../mystyle}

\begin{document}

\section{Графический метод решения задач ЛП. Теорема о количестве решений ЗЛП}

Если в задаче ЛП всего две переменных, то для её решения можно использовать геометрическую интерпретацию. В этом случае ограничение типа равенства представляет собой прямую, а каждое ограничение типа неравенства — полуплоскость.

Если задача записана в стандартной форме:
\[
    \begin{cases}
        c_1 x_1 + c_2 x_2 \to \max                                \\
        a_{i1} x_1 + a_{i2} x_2 \le b_i, \quad i = \overline{1,m} \\
        x_1 \ge 0, \quad x_2 \ge 0
    \end{cases}
\]
то допустимое множество \( \Omega \) будет пересечением некоторого множества полуплоскостей и будет представлять собой:
\begin{itemize}
    \item выпуклый многоугольник;
    \item отрезок;
    \item незамкнутую область;
    \item единственную точку;
    \item пустое множество.
\end{itemize}

Линии уровня целевой функции составляют семейство параллельных прямых, общим нормальным вектором которых является вектор \( c = \begin{pmatrix} c_1 \\ c_2 \end{pmatrix} \) коэффициентов целевой функции. При перемещении прямой вдоль этого вектора можно найти её крайнее положение, когда она ещё пересекает многоугольник (прямая \( f^* \)).

Каждая точка пересечения этой прямой с многоугольником изображает оптимальное решение задачи, а значение функции, соответствующее этой линии уровня, есть максимальное значение целевой функции.

\subsection{Теорема}
\begin{theorem}
    Если задача линейного программирования имеет оптимальное решение, то оно совпадает с одной (двумя) из угловых точек допустимого множества.
\end{theorem}

\end{document}