\documentclass[17pt]{extarticle}
\usepackage{../mystyle}

\begin{document}

\section{Выводимая из базиса переменная в ЗЛП. Вырожденная ЗЛП}

Пусть есть задача ЛП в канонической форме. Пусть первые \( r \) переменных будут базисными, а остальные — свободными. Разрешим систему ограничений относительно базисных переменных, а также выразим целевую функцию через свободные переменные:
\[
    f(x) = \gamma_0 - (\gamma_{r+1}x_{r+1} + \dots + \gamma_n x_n)
\]
\[
    \begin{cases}
        x_1 = \beta_1 - (\alpha_{1 r+1}x_{r+1} + \dots + \alpha_{1n}x_n) \\
        x_i = \beta_i - (\alpha_{i r+1}x_{r+1} + \dots + \alpha_{in}x_n) \\
        x_r = \beta_r - (\alpha_{r r+1}x_{r+1} + \dots + \alpha_{rn}x_n) \\
    \end{cases}
\]

Возьмём все свободные переменные равными нулю. Тогда получим базисное решение \( x_i = \beta_i, \quad i = \overline{1,r} \), \( \vec{x_b} = (\beta_1, \beta_2, \dots, \beta_r, 0, \dots, 0)^T \). Значение целевой функции, в свою очередь, равно \( f(\vec{x_b}) = \gamma_0 \).

Легко заметить, что значение целевой функции можно увеличить только в случае, если имеются \( \gamma_j < 0 \). Пусть такой имеется, тогда за счёт увеличения \( x_j \) можно увеличить значение целевой функции. Возьмём все свободные переменные равными нулю, кроме \( x_j \):
\[
    f(\vec{x}) = \gamma_0 - \gamma_j x_j
\]
\[
    x_i = \beta_i - \alpha_{ij} x_j, \quad i = \overline{1,r}
\]

Если все коэффициенты \( \alpha_{ij} \le 0 \), то увеличение \( x_j \) может быть неограниченным, это приводит к неограниченному возрастанию целевой функции. В таком случае считается, что задача не имеет решения.

Если же \( \alpha_{ij} > 0 \), то увеличение \( x_j \) приведёт к тому, что базисная переменная \( x_i \) будет уменьшаться, пока не станет равна нулю. Это определяется уравнением: \( x_i = \beta_i - \alpha_{ij} x_j = 0 \). Отсюда \( x_j = \frac{\beta_i}{\alpha_{ij}} \).

Если несколько \( \alpha_{ij} > 0 \), то первой в ноль обратится переменная \( x_l \), для которой отношение \( \frac{\beta_l}{\alpha_{lj}} \) минимально. Таким образом, нужная переменная выбирается из соотношения
\[
    \frac{\beta_l}{\alpha_{lj}} = \min\limits_i \left\{ \frac{\beta_i}{\alpha_{ij}} \right\} = \rho
\]

Элемент \( \alpha_{lj} \) называется разрешающим: он указывает переменную \( x_l \), которую выводят из базиса.

Может оказаться, что \( \rho = 0 \), тогда целевая функция сохраняет предыдущее значение, хотя переменные меняются. Такой случай называют \textbf{вырожденным}.

\end{document}