\documentclass[17pt]{extarticle}
\usepackage{../mystyle}
\usepackage{amsthm} % Для оформления теорем

\begin{document}

\section{Теорема о допустимом базисном решении}

\begin{theorem}
    Допустимое решение задачи линейного программирования в канонической форме является допустимым базисным решением (ДБР) тогда и только тогда, когда оно является крайней точкой \( \Omega \).
\end{theorem}

\begin{proof}
    У ДБР ненулевыми могут быть только те компоненты, которые отвечают базисным переменным, т.е. система столбцов матрицы \( A \) ограничений, отвечающих ненулевым компонентам вектора, есть часть системы базисных столбцов, а потому она линейно независима. Следовательно, ДБР — крайняя точка допустимого множества.

    У крайней точки \( x \) допустимого множества количество ненулевых значений не превышает ранга матрицы \( A \), а, значит, количества ограничений \( m \). Соответствующие столбцы матрицы \( A \) линейно независимы, и их можно дополнить до набора базисных столбцов. Всем оставшимся столбцам будут соответствовать нулевые компоненты вектора \( x \). Следовательно, вектор \( x \) является базисным решением. Т.к. он принадлежит допустимому множеству, он есть ДБР.
\end{proof}

\end{document}