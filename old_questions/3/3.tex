\documentclass[17pt]{extarticle}
\usepackage{../mystyle}

\begin{document}

\section{Примеры задач оптимизации в экономике}

Рассмотрим \textbf{транспортную задачу}. Предположим, что фирма имеет \( n \) магазинов, в которые поступает товар, хранимый на \( m \) складах. Известна стоимость \( c_{ij} \) перевозки товара с \( i \)-го склада в \( j \)-й магазин, количество \( a_i \) единиц товара на \( i \)-ом складе и заказанный объём \( b_j \) товара для доставки в \( j \)-й магазин. Требуется составить план перевозок товара со складов в магазины так, чтобы суммарная стоимость перевозок была минимальной.

Обозначим через \( x_{ij} \) количество товара, которое планируется перевезти с \( i \)-го склада в \( j \)-й магазин. Тогда стоимость перевозки этого товара составит \( c_{ij}x_{ij} \), а общая стоимость перевозок будет равна:
\[
    \sum_{i=1}^m \sum_{j=1}^n c_{ij}x_{ij}
\]

Магазины должны быть обеспечены товаром в точном соответствии с заказом. Поэтому планируемые объёмы перевозок должны удовлетворять условиям:
\[
    \sum_{i=1}^m x_{ij} = b_j, \quad j = \overline{1,n}
\]

Однако с любого склада нельзя вывести товара больше, чем там его находится. Следовательно, должны выполняться условия:
\[
    \sum_{j=1}^n x_{ij} \le a_i, \quad i = \overline{1,m}
\]

Объединяя перечисленные условия с условием неотрицательности количества товара: \( x_{ij} \ge 0, \quad i = \overline{1,m}, \quad j = \overline{1,n} \), получим постановку задачи оптимизации.

Необходимо отметить, что задача имеет решение только в том случае, когда сумма заказов не превышает количества товара на складах:
\[
    \sum_{j=1}^n b_j \le \sum_{i=1}^m a_i
\]

\end{document}