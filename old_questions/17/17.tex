\documentclass[17pt]{extarticle}
\usepackage{../mystyle}

\begin{document}

\section{Симплекс-метод. Идея. Процесс. Базис ЗЛП}

Поиск оптимального решения задачи линейного программирования в канонической форме следует проводить среди допустимых базисных решений, или, что то же самое, среди крайних точек допустимого множества. Полный перебор всех таких точек неэффективен, так что можно процесс поиска строить, переходя от одного базисного решения к другому, которое будет являться "соседним". Это идея симплекс-метода.

Проиллюстрируем процесс работы симплекс-метода на примере. Пусть требуется максимизировать \( f(x_1, x_2, x_3) = \alpha x_3, \quad \alpha > 0 \). При такой постановке необходимо искать точку многогранника с максимальным значением координаты \( x_3 \).
Выберем одну из вершин в качестве начальной. На каждом шаге переходим в одну из соседних (т.е. вершин, соединённых с текущей ребром многогранника), выбирая среди них такую, у которой \( x_3 \) больше. Если такой соседней вершины нет, то текущая вершина есть оптимальное решение.

Процесс поиска оптимального решения можно представить себе как последовательный переход от одного допустимого базисного решения к "соседнему", дающему большее значение целевой функции. "Соседство" состоит в том, что новое допустимое базисное решение получается из исходного заменой одного из свободных переменных; иначе говоря, из базиса выводится одно переменное и в него включается другое, которое до этого было свободным.

Совокупность базисных переменных будем называть \textbf{базисом задачи линейного программирования}.

\end{document}