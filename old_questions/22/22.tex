\documentclass[17pt]{extarticle}
\usepackage{../mystyle}
\usepackage{graphicx} % Для вставки изображений

\begin{document}

\section{Общий метод нахождения начального допустимого базисного решения ЗЛП}

Для решения задачи ЛП симплекс-методом необходимо выразить базисные переменные через свободные, причём в этом выражении свободные члены должны быть неотрицательными. В случае, если эта проблема не решается тривиально, используется \textit{метод искусственного базиса}, при котором в задачу вводятся дополнительные искусственные переменные.

\subsection{Метод вспомогательной задачи}

Пусть дана задача ЛП:
\[
    \vec{c}\vec{x} \to \max
\]
\[
    \begin{cases}
        a_{i1}x_1 + \dots + a_{in}x_n = b_i, \quad i = \overline{1,m} \\
        x_j \ge 0, \quad j = \overline{1,n}
    \end{cases}
\]

Вспомогательная задача имеет вид:
\[
    -x_{n+1} - x_{n+2} - \dots - x_{n+m} \to \max
\]
\[
    \begin{cases}
        a_{i1}x_1 + \dots + a_{in}x_n + x_{n+i} = b_i, \quad i = \overline{1,m} \\
        x_j \ge 0, \quad j = \overline{1,n+m}
    \end{cases}
\]

Здесь \( \forall b_i \ge 0 \). В вспомогательной задаче очевидно опорное решение: \( x' = (0, \dots, 0, b_1, \dots, b_m) \), где первые \( n \) нулей — значения основных переменных, а остальные значения соответствуют искусственным. Искусственные переменные образуют единичный базис, значит можно решать задачу симплекс-методом и получить оптимальное решение вспомогательной задачи:
\[
    x^* = (x_1^*, \dots, x_{n+m}^*)
\]

В зависимости от того, входят ли искусственные векторы в базис оптимального решения и какие при этом значения имеют искусственные переменные, относительно исходной задачи можно сделать различные выводы.

\subsubsection{Случай 1}
Среди чисел \( x_{n+1}^*, \dots, x_{n+m}^* \) есть отличные от нуля. В этом случае исходная задача не имеет допустимых решений.

\subsubsection{Случай 2}
В оптимальном решении все искусственные переменные имеют нулевое значение.

\paragraph{Случай 2а}
В оптимальном базисе нет искусственных векторов. В этом случае получено опорное решение исходной задачи и известно разложение всех векторов задачи по базису опорного решения. Для решения исходной задачи достаточно:
\begin{itemize}
    \item в полученной симплекс-таблице отбросить столбцы искусственных переменных;
    \item восстановить коэффициенты целевой функции исходной задачи;
    \item пересчитать значение целевой функции и оценки свободных векторов.
\end{itemize}

\paragraph{Случай 2б}
В оптимальный базис входят искусственные векторы. Важно знать, существуют ли среди коэффициентов разложения основных векторов при искусственных базисных векторах отличные от нуля.


Здесь для определённости принято, что первые \( l \) переменных основные, а остальные — искусственные. Рассматриваем область \( x_{rs} \).

\subparagraph{Случай 2б1}
Все \( x_{rs} = 0 \). Тогда любой вектор исходной задачи выражается только через основные вектора, входящие в базис оптимального решения вспомогательной задачи. Таким образом, можно выкинуть строки и столбцы, связанные с искусственными переменными, и действовать как в случае 2а.

\subparagraph{Случай 2б2}
Существует \( x_{rs} \ne 0 \). Тогда основной вектор \( A_s \) принудительно вводится в базис вместо искусственного вектора \( A_{i_r} \) и по обычным правилам перестраивается симплекс-таблица. Процедура повторяется, пока не возникнет ситуация 2а или 2б1. В любом случае будет получено опорное решение исходной задачи и базис этого решения, состоящий только из основных векторов.

\end{document}