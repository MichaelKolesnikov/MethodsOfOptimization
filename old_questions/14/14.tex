\documentclass[17pt]{extarticle}
\usepackage{../mystyle}

\begin{document}

\section{Базисное решение ЗЛП}

Рассмотрим задачу ЛП в канонической форме:
\[
    \begin{cases}
        (c, x) \to \max \\
        A x = b         \\
        x \ge 0
    \end{cases}
\]
Пусть \( A \in \mathbb{R}^{m \times n} \). Предполагается, что \( m < n \) и что \( \text{rang}\, A = m \). Тогда можно выбрать из этой матрицы базисный минор \( B \in \mathbb{R}^{m \times m} \). Оставшиеся столбцы матрицы обозначим как \( N \in \mathbb{R}^{m \times (n - m)} \).

Такая операция разделит неизвестные на базисные и свободные:
\[
    x_B = \begin{pmatrix} x_1 \\ \vdots \\ x_m \end{pmatrix}, \quad
    x_N = \begin{pmatrix} x_{m+1} \\ \vdots \\ x_n \end{pmatrix}.
\]

С учётом введённых обозначений ограничения можно описать в виде матричного уравнения:
\[
    B x_B + N x_N = b
\]

Матрица \( B \) представляет собой базисный минор, поэтому невырождена. Значит \( \exists B^{-1} \). Умножим равенство на \( B^{-1} \), чтобы выразить вектор базисных переменных:
\[
    x_B = B^{-1}(b - N x_N)
\]

Таким образом, вектор
\[
    x = \begin{pmatrix}
        x_B \\ x_N
    \end{pmatrix}
    = \begin{pmatrix}
        B^{-1}(b - N x_N) \\ x_N
    \end{pmatrix}
\]
удовлетворяет системе ограничений, а при \( x \ge 0 \) является допустимым решением. Выделим случай \( x_N = 0 \). Здесь \( x = x_B \), и такое решение называется \textbf{базисным решением}. Базисное решение не является единственным, так как зависит от выбора минора, однако число базисных решений конечно.

\end{document}