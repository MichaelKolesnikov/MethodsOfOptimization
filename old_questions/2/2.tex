\documentclass[17pt]{extarticle}
\usepackage{../mystyle}

\begin{document}

\section{Задача математического программирования}

С математической точки зрения задача конечномерной оптимизации заключается в определении наибольшего (или наименьшего) значения функции \( f(x_1, x_2, \dots, x_n) \) на заданном множестве \( \Omega \) и точки \( x^* \in \Omega \), в которой достигается это значение функции \( f \).

\subsection{Определения}
\begin{itemize}
    \item Целевая функция: \( f(x_1, x_2, \dots, x_n) \)
    \item Допустимое множество: \( \Omega \in \mathbb{R}^{n} \)
    \item Допустимые решения: \( \forall x = (x_1, x_2, \dots, x_n) \in \Omega \)
    \item Оптимальное решение: \( x^* \in \Omega \)
\end{itemize}

\subsection{Постановка задачи математического программирования в общем виде}
\[
    \begin{cases}
        f_0(x_1, x_2, \dots, x_n) \to \max(\min) \\
        f_i(x_1, x_2, \dots, x_n) = 0, \quad \forall i = 1, \dots, m
    \end{cases}
\]

Функция на заданном участке может не достигать наименьшего (наибольшего) значения, если
\begin{itemize}
    \item она не ограничена снизу (сверху);
    \item она ограничена снизу, но не достигает точной нижней грани \textit{(это как?)}.
\end{itemize}

В этих случаях задача не имеет решения.

\end{document}