\documentclass[17pt]{extarticle}
\usepackage{../mystyle}

\begin{document}

\section{Примеры задач оптимального проектирования (оптимизация в проектировании технических устройств)}

Рассмотрим задачу проектирования бака горючего в виде прямого кругового цилиндра заданного объёма \( V \), будем минимизировать количество стали.

В качестве параметров оптимизации выберем радиус \( R \) и высоту \( H \) цилиндра. Тогда затраты материала на изготовление бака будет определять площадь его поверхности. Ограничением будет объём бака:
\[
    \begin{cases}
        2 \pi R (H + R) \to \min \\
        \pi R^2 H = V            \\
        R > 0, \quad H > 0
    \end{cases}
\]

Выразим \( H \) через \( R \):
\[
    H = \frac{V}{\pi R^2}
\]

Подставим в целевую функцию и получим задачу минимизации функции одного переменного:
\[
    S(R) = 2 \frac{V}{R} + 2 \pi R^2
\]

Вычислим производную и приравняем её к нулю:
\[
    S'(R) = -2 \frac{V}{R^2} + 4 \pi R = 0
\]
\[
    R_* = \sqrt[3]{\frac{V}{2 \pi}}
\]

Зная \( R_* \), найдём второй параметр:
\[
    H_* = \frac{V}{\pi R_*^2} = \sqrt[3]{\frac{4 V}{\pi}} = 2 R_*
\]

Получили, что цилиндрический бак горючего будет иметь наименьшую площадь поверхности в том случае, когда его высота совпадает с диаметром его основания.

\end{document}