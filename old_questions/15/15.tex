\documentclass[17pt]{extarticle}
\usepackage{../mystyle}
\usepackage{amsthm} % Для оформления лемм и доказательств

\begin{document}

\section{Лемма о крайних точках}

\begin{lemma}
    Допустимое решение \( x = (x_1 \ x_2 \ \dots \ x_n)^T \) задачи линейного программирования в канонической форме
    является точкой множества допустимых решений тогда и только тогда, когда система всех столбцов матрицы \( A \),
    отвечающих ненулевым значениям \( x_i \), \( i = \overline{1,n} \), линейно независима.
\end{lemma}

\begin{proof}
    Докажем от противного. Можно считать, что вектор \( x \) имеет вид
    \( x = (x_1 \ x_2 \ \dots \ x_r \ 0 \ \dots \ 0)^T \), \( x_i \ne 0 \), \( i = \overline{1,r} \).

    Если вектор \( x \) не является крайней точкой допустимого множества,
    то \( \exists u \ne 0 \) вектор и такое число \( \delta > 0 \), что вектор \( x + tu \), \( |t| < \delta \),
    принадлежит допустимому множеству.

    Т.е. \( A(x + tu) = b \), \( x + tu \ge 0 \). Последнее неравенство можно записать в виде \( x \ge -tu \),
    тогда имеем, что \( \forall i = \overline{r+1, n} \), \( u_i = 0 \), т.к. при \( t > 0 \) имеем \( u_i \ge 0 \),
    а при \( t < 0 \) — \( u_i \le 0 \).

    Поскольку \( x \) принадлежит допустимому множеству, выполняется \( Ax = b \).
    С учётом равенства \( A(x + tu) = b \) получаем, что \( Au = 0 \). С учётом вышепоказанного это выражение записывается так:
    \[
        a_1u_1 + a_2u_2 + \dots + a_ru_r = 0 \quad (*)
    \]
    По условию \( u \ne 0 \), значит по критерию линейной зависимости столбцы \( a_1, a_2, \dots, a_r \) линейно зависимы.

    Теперь пусть столбцы \( a_1, a_2, \dots, a_r \) линейно зависимы.
    Тогда \( \exists u_1, u_2, \dots, u_r \) (\( \sum\limits_{i=0}^r |u_i| \ne 0 \)),
    для которых выполняется равенство \( (*) \). Рассмотрим вектор \( u = (u_1 \ \dots \ u_r \ 0 \ \dots \ 0)^T \).
    Равенство \( (*) \) означает, что \( Au = 0 \). Следовательно, \( A(x + tu) = Ax = b \) для любого \( t \),
    т.е. вектор \( x + tu \) удовлетворяет ограничениям задачи ЛП.
    Покажем, что при малых значениях \( t \) этот вектор удовлетворяет и условию неотрицательности.

    Пусть \( \rho = \max\limits_{i = \overline{1,r}} |u_i| > 0 \). Пусть \( \delta = \min\limits_{i = \overline{1,r}} \frac{|x_i|}{\rho} > 0 \).
    При этом при \( |t| < \delta \) выполняется:
    \[
        x_i + tu_i \ge x_i - |t||u_i| > x_i - \delta|u_i| \ge x_i - \delta\rho \ge 0
    \]

    Следовательно, \( x + u \ge 0 \), и вектор \( x \) не является крайней точкой допустимого множества.
\end{proof}

\end{document}