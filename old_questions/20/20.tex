\documentclass[17pt]{extarticle}
\usepackage{../mystyle}

\begin{document}

\section{Вырожденное допустимое базисное решение. Зацикливание в решении ЗЛП}

Вырожденному ДБР соответствуют несколько наборов базисных переменных. В этом случае смена базиса не вызывает изменения ДБР и продвижения к оптимальному решению не происходит. Чтобы избежать этой ситуации, можно изменить вводимую в базис переменную, выбрав другую положительную симплекс-разность, но это не всегда возможно. Может случиться, что после перебора нескольких базисов данного ДБР произойдёт возврат к уже рассматривавшемуся базису. Произойдёт зацикливание.

Эту ситуацию можно исключить, если все базисы определённым образом упорядочить, используя значение ЦФ на соответствующем ДБР и состав переменных, входящих в базис. Тогда базисы не будут повторяться. При наличии положительных симплекс-разностей уточнённый итерационный процесс перехода от одного ДБР к другому приведёт к новому большому базису. Следовательно, итерационный процесс остановится, когда для очередного базиса все симплекс-разности окажутся положительными. Это будет означать получение оптимального решения.

\subsection{Дополнение из Загребаева (стр. 35–41)}
Можно добавить информацию о методах предотвращения зацикливания, таких как \textbf{антициклины} и другие подходы, описанные в работах Загребаева. Эти методы позволяют упорядочить выбор базисных переменных и исключить повторение базисов, что гарантирует сходимость симплекс-метода.

\end{document}