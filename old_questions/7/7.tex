\documentclass[17pt]{extarticle}
\usepackage{../mystyle}

\begin{document}

\section{Задачи выпуклого программирования. Выпуклое множество. Выпуклая функция}

Задача \textbf{выпуклого} программирования:
Допустимое множество \( \Omega \) выпуклое, и целевая функция \( f_0(x) \) выпуклая.

\subsection{Выпуклое множество}
Множество \( X \subset \mathbb{R}^n \) называется выпуклым, если для \( \forall x, y \in X \) и \( \forall \alpha \in (0,1) \) выполняется условие:
\[
    \alpha x + (1 - \alpha) y \in X.
\]
Пересечение любого семейства выпуклых множеств является выпуклым множеством.

\subsection{Выпуклая функция}
Функция \( f(x) \), определённая на выпуклом множестве \( X \), называется выпуклой, если:
\[
    \forall x, y \in X, \quad \forall \alpha \in (0,1) \Rightarrow f(\alpha x + (1 - \alpha) y) \le \alpha f(x) + (1 - \alpha) f(y).
\]
Если неравенство выполняется строго, то \( f(x) \) строго выпуклая. Пример: квадратичная функция с положительно определённой квадратичной формой.

Если функции \( f_1(x), f_2(x), \dots, f_k(x) \) определены на выпуклом множестве и являются выпуклыми, то и функция:
\[
    f(x) = \min\{ f_1(x), f_2(x), \dots, f_k(x) \}
\]
выпуклая.

Для выпуклой функции любая точка локального минимума есть точка наименьшего значения. Для выпуклой и дифференцируемой функции любая стационарная точка (точка с нулевым значением градиента) есть точка наименьшего значения. Строго выпуклая функция может иметь только одну точку наименьшего значения.

\subsection{Сильно выпуклая функция}
Функция \( f(x) \) называется сильно выпуклой, если:
\[
    \forall x, y \in X, \quad \forall \alpha \in (0,1), \quad \exists \gamma > 0 \Rightarrow
    f(\alpha x + (1 - \alpha) y) \le \alpha f(x) + (1 - \alpha) f(y) - \alpha(1-\alpha)\gamma|x-y|^2.
\]
Сильно выпуклая функция всегда достигает наименьшего значения.

\end{document}