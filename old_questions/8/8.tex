\documentclass[17pt]{extarticle}
\usepackage{../mystyle}

\begin{document}

\section{Задачи стохастического и динамического программирования}

\subsection{Стохастическое программирование}

\textbf{Стохастическое программирование} — это подход, позволяющий учитывать неопределённость в оптимизационных методах. В подобных задачах хотя бы один параметр оптимизации должен являться случайной величиной.

\subsubsection{Постановка}
Пусть целевая функция есть \( f_0(x, \omega) \), где \( \omega \) — некоторые случайные параметры, а \( M \) — оператор матожидания.
\[
    \begin{cases}
        M[f_0(x, \omega)] \to \min                        \\
        M[f_i(x, \omega)] \le 0, \quad i = \overline{1,m} \\
        x = (x_1, x_2, \dots, x_n)
    \end{cases}
\]

\subsection{Динамическое программирование}

\textbf{Динамическое программирование} — это подход, при котором задача разбивается на группу последовательных этапов.

\subsubsection{Принципы работы метода}
Состоянием системы \( x_i \) на этапе \( i \) будем называть некоторый набор данных, получаемый в процессе работы этапа \( i \).

\begin{itemize}
    \item \textit{Последовательность этапов}. Все этапы выполняются друг за другом в жёстко заданной последовательности.
    \item \textit{Передача состояния от текущего этапа к следующему}. Генерируемое в процессе работы этапа \( i \) состояние передаётся этапу \( i+1 \).
    \item \textit{Рекуррентная природа вычислений}. Каждый этап \( i \) представляет собой некоторую функцию, которая получает на вход вектор состояния предыдущего этапа \( x_{i-1} \), по определённому правилу выделяет некоторые альтернативы, выбирает из них лучшие и формирует новое состояние: \( x_i = \Psi(x_{i-1}) \).
    \item \textit{Принцип оптимальности Беллмана}. Оптимальная стратегия на каждом отдельном этапе определяется исключительно на основании текущего состояния, независимо от состояний и оптимальных стратегий на предшествующих этапах.
\end{itemize}

\subsection{Примечание}
В присланных билетах по-другому, там больше формул. Я посчитал, что это излишне.

\end{document}