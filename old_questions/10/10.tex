\documentclass[17pt]{extarticle}
\usepackage{../mystyle}

\begin{document}

\section{Виды задач линейного программирования (ЛП)}

\subsection{Общего вида}
\[
    \begin{cases}
        \sum\limits_{j=1}^{n} c_j x_j \to \max                                 \\
        \sum\limits_{j=1}^{n} a_{ij} x_j \le b_i, \quad i = \overline{1,m_1}   \\
        \sum\limits_{j=1}^{n} a_{ij} x_j = b_i, \quad i = \overline{m_1+1,m_2} \\
        \sum\limits_{j=1}^{n} a_{ij} x_j \ge b_i, \quad i = \overline{m_2+1,m}
    \end{cases}
\]

\subsection{1. Неотрицательных переменных}
\[
    \begin{cases}
        \sum\limits_{j=1}^{n} c_j x_j \to \max                               \\
        \sum\limits_{j=1}^{n} a_{ij} x_j \le b_i, \quad i = \overline{1,m_1} \\
        \sum\limits_{j=1}^{n} a_{ij} x_j = b_i, \quad i = \overline{m_1+1,m} \\
        x_j \ge 0, \quad j = \overline{1,n}
    \end{cases}
\]

\subsection{2. Стандартная форма}
\[
    \begin{cases}
        \sum\limits_{j=1}^{n} c_j x_j \to \max                             \\
        \sum\limits_{j=1}^{n} a_{ij} x_j \le b_i, \quad i = \overline{1,m} \\
        x_j \ge 0, \quad j = \overline{1,n}
    \end{cases}
\]

\subsection{3. Каноническая форма}
\[
    \begin{cases}
        \sum\limits_{j=1}^{n} c_j x_j \to \max                           \\
        \sum\limits_{j=1}^{n} a_{ij} x_j = b_i, \quad i = \overline{1,m} \\
        x_j \ge 0, \quad j = \overline{1,n}
    \end{cases}
\]

\subsection{4. Матричная стандартная форма}
\[
    \begin{cases}
        (c, x) \to \max \\
        A x \le b       \\
        x \ge 0
    \end{cases}
\]

\end{document}