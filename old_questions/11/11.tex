\documentclass[17pt]{extarticle}
\usepackage{../mystyle}
\usepackage{array} % Для таблиц

\begin{document}

\section{Преобразования форм задач ЛП}

\subsection{1. Изменение направления оптимизации}

\begin{center}
    \begin{tabular}{|c|c|}
        \hline
        \textbf{Исходная задача}                   & \textbf{Эквивалентная задача}               \\
        \hline
        \( \sum\limits_{j=1}^n c_j x_j \to \min \) & \( -\sum\limits_{j=1}^n c_j x_j \to \max \) \\
        \hline
    \end{tabular}
\end{center}

Оптимальные решения исходной задачи и задачи, полученной в результате преобразования, будут совпадать, однако оптимальные значения целевых функций будут иметь противоположные знаки.

\subsection{2. Придание ограничениям-неравенствам противоположного направления}

\begin{center}
    \begin{tabular}{|c|c|}
        \hline
        \textbf{Исходная задача}                     & \textbf{Эквивалентная задача}                  \\
        \hline
        \( \sum\limits_{j=1}^n a_{ij} x_j \ge b_i \) & \( -\sum\limits_{j=1}^n a_{ij} x_j \le -b_i \) \\
        \hline
    \end{tabular}
\end{center}

\subsection{3. Наложение на переменные требования неотрицательности}

\begin{center}
    \begin{tabular}{|c|c|}
        \hline
        \textbf{Исходная задача}        & \textbf{Эквивалентная задача}                                          \\
        \hline
        \( x_j \) не ограничена в знаке & Замена переменной: \( x_j = x_j^1 - x_j^2, \quad x_j^1, x_j^2 \ge 0 \) \\
        \hline
    \end{tabular}
\end{center}

\subsection{4. Замена уравнений неравенствами}

\begin{center}
    \begin{tabular}{|c|c|}
        \hline
        \textbf{Исходная задача}                   & \textbf{Эквивалентная задача} \\
        \hline
        \( \sum\limits_{j=1}^n a_{ij} x_j = b_i \) &
        \(
        \begin{cases}
            \sum\limits_{j=1}^n a_{ij} x_j \le b_i \\
            -\sum\limits_{j=1}^n a_{ij} x_j \le -b_i
        \end{cases}
        \)                                                                         \\
        \hline
    \end{tabular}
\end{center}

\subsection{5. Замена нестрогих неравенств уравнениями}

\begin{center}
    \begin{tabular}{|c|c|}
        \hline
        \textbf{Исходная задача}                     & \textbf{Эквивалентная задача} \\
        \hline
        \( \sum\limits_{j=1}^n a_{ij} x_j \le b_i \) &
        \(
        \begin{cases}
            \sum\limits_{j=1}^n a_{ij} x_j + x_{n+1} = b_i \\
            x_{n+1} \ge 0
        \end{cases}
        \)                                                                           \\
        \hline
        \( \sum\limits_{j=1}^n a_{ij} x_j \ge b_i \) &
        \(
        \begin{cases}
            \sum\limits_{j=1}^n a_{ij} x_j - x_{n+1} = b_i \\
            x_{n+1} \ge 0
        \end{cases}
        \)                                                                           \\
        \hline
    \end{tabular}
\end{center}

\end{document}