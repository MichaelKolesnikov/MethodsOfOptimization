\documentclass[17pt]{extarticle}
\usepackage{../mystyle}
\usepackage{graphicx} % Для вставки изображений

\begin{document}

\section{Примеры задач оптимизации в геометрии}

\subsection{Пример 1}
Рассмотрим задачу определения сторон прямоугольника, вписанного в окружность радиуса \( R \) и имеющего наибольшую площадь \( S \):


Известно, что площадь прямоугольника равна половине произведения его диагоналей на синус угла между ними:
\[
    S = 2 R^2 \sin{\phi}
\]

Эта площадь будет наибольшей при \( \sin{\phi} = 1 \) или при \( \phi = \pi / 2 \). Отсюда получаем, что искомый прямоугольник — это квадрат со стороной \( \sqrt{2} R \) площади \( 2 R^2 \).

Сформулируем данную задачу в виде задачи оптимизации. Выберем параметрами оптимизации длины сторон квадрата: \( a \) и \( b \). Тогда ограничение на параметры следует из теоремы Пифагора.
\[
    \begin{cases}
        S = a b \to \max  \\
        a^2 + b^2 = 4 R^2 \\
        a > 0, \quad b > 0
    \end{cases}
\]

\end{document}