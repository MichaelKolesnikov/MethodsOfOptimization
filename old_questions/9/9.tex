\documentclass[17pt]{extarticle}
\usepackage{../mystyle}

\begin{document}

\section{Задачи математического программирования с дискретным множеством параметров оптимизации}

Задача \textbf{дискретного} программирования подразумевает дискретность области определения всех или нескольких параметров.

В линейном целочисленном программировании исследуется модель вида:
\[
    \begin{cases}
        \sum\limits_{j=1}^{n} c_j x_j \to \max (\min)                        \\
        \sum\limits_{j=1}^{n} a_{ij} x_{j} \le b_i, \quad i = \overline{1,m} \\
        x_j \ge 0, \quad j = \overline{1,n}                                  \\
        x_j - \text{целые}, \quad j = \overline{1,n'}
    \end{cases}
\]
При \( n = n' \) имеет место \textit{задача линейного целочисленного программирования}; при \( n > n' \) — задача линейного частично-целочисленного программирования.

Условие целочисленности можно заменить требованием дискретности:
\[
    x_j \in \{ d_1^j, d_2^j, \dots, d_{k_j}^j \}, \quad j = \overline{1, n'}.
\]
В этом случае имеет место \textit{задача линейного программирования с дискретными переменными}.

\subsection{Частный случай: задача ЛЦП с булевыми переменными}
Частным случаем задачи линейного целочисленного программирования является задача ЛЦП с булевыми переменными:
\[
    \begin{cases}
        \sum\limits_{j=1}^{n} c_j x_j \to \max (\min)                        \\
        \sum\limits_{j=1}^{n} a_{ij} x_{j} \le b_i, \quad i = \overline{1,m} \\
        x_j \in \{0,1\}, \quad j = \overline{1,n}
    \end{cases}
\]

\end{document}