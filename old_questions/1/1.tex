\documentclass[17pt]{extarticle}
\usepackage{../mystyle}

\begin{document}

\section{Задачи оптимизации}

\textbf{Задачи оптимизации} — широкий класс задач, связанных с применением численных методов.

\textbf{Задача оптимизации}: \\
Есть некоторое множество возможных решений, называемых альтернативными. Каждой альтернативе можно дать некоторую количественную оценку на основе некоторого критерия оптимальности. Решение задачи оптимизации состоит в определении той альтернативы, для которой критерий оптимальности даёт наибольшую или наименьшую количественную оценку.

\textbf{Виды оптимизации}:
\begin{itemize}
    \item одномерная — множество альтернатив описывается одним числовым параметром;
    \item многомерная — множество альтернатив описывается несколькими параметрами;
    \item бесконечномерная — каждая альтернатива характеризуется бесконечным числом параметров;
    \item конечномерная;
    \item многокритериальная — в задаче оптимизации несколько критериев оптимальности.
\end{itemize}

Критерий представляется в виде целевой функции: \\
\[ f_0(x) \to \min, \quad x \in \Omega \]
Считается, что \( f_0(x) \) определена всюду на допустимом множестве \( \Omega \), т.е. область определения \( f_0(x) \) включает в себя допустимое множество, хотя может и не совпадать с ним.

\end{document}