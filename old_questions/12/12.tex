\documentclass[17pt]{extarticle}
\usepackage{../mystyle}

\begin{document}

\section{Базисные и свободные переменные в задаче ЛП}

Рассмотрим задачу ЛП в канонической форме:
\[
    \begin{cases}
        (c, x) \to \max \\
        A x = b         \\
        x \ge 0
    \end{cases}
\]
Пусть \( A \in \mathbb{R}^{m \times n} \). Предполагается, что \( m < n \) и что \( \text{rang}\, A = m \). Тогда можно выбрать из этой матрицы базисный минор \( B \in \mathbb{R}^{m \times m} \). Оставшиеся столбцы матрицы обозначим как \( N \in \mathbb{R}^{m \times (n - m)} \).

Такая операция разделит неизвестные на базисные и свободные:
\[
    x^B = \begin{pmatrix} x_1 \\ \vdots \\ x_m \end{pmatrix}, \quad
    x^N = \begin{pmatrix} x_{m+1} \\ \vdots \\ x_n \end{pmatrix}.
\]

С учётом введённых обозначений ограничения можно описать в виде матричного уравнения:
\[
    B x^B + N x^N = b
\]

\end{document}