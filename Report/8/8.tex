\documentclass[17pt]{extarticle}
\usepackage{../mystyle}

\begin{document}
\section*{Работа №8}
\subsection*{Решение задачи о назначениях с помощью Венгерского алгоритма}
\begin{verbatim}
    Ввод-вывод:
    10
    1 6 5 5 8 2 8 4 6 8
    9 9 1 6 3 2 4 8 9 9
    2 5 2 6 8 8 2 3 9 7
    8 8 8 6 7 9 5 2 2 2
    5 2 7 5 7 4 7 9 4 2
    1 3 7 7 2 8 7 1 4 7
    5 6 9 2 3 6 8 5 8 3
    8 9 6 7 5 9 2 3 7 3
    2 2 2 9 8 8 2 1 5 1
    2 6 8 4 8 6 7 9 5 7
    Optimal assignment:
    0 0 0 0 0 1 0 0 0 0
    0 0 1 0 0 0 0 0 0 0
    0 0 0 0 0 0 0 1 0 0
    0 0 0 0 0 0 0 0 1 0
    0 1 0 0 0 0 0 0 0 0
    0 0 0 0 1 0 0 0 0 0
    0 0 0 1 0 0 0 0 0 0
    0 0 0 0 0 0 1 0 0 0
    0 0 0 0 0 0 0 0 0 1
    1 0 0 0 0 0 0 0 0 0
    Total cost: 19
\end{verbatim}

\subsection*{Исходная матрица}

Исходная матрица имеет вид:

\[
    \begin{array}{*{10}{c}}
        1 & 6 & 5 & 5 & 8 & 2 & 8 & 4 & 6 & 8 \\
        9 & 9 & 1 & 6 & 3 & 2 & 4 & 8 & 9 & 9 \\
        2 & 5 & 2 & 6 & 8 & 8 & 2 & 3 & 9 & 7 \\
        8 & 8 & 8 & 6 & 7 & 9 & 5 & 2 & 2 & 2 \\
        5 & 2 & 7 & 5 & 7 & 4 & 7 & 9 & 4 & 2 \\
        1 & 3 & 7 & 7 & 2 & 8 & 7 & 1 & 4 & 7 \\
        5 & 6 & 9 & 2 & 3 & 6 & 8 & 5 & 8 & 3 \\
        8 & 9 & 6 & 7 & 5 & 9 & 2 & 3 & 7 & 3 \\
        2 & 2 & 2 & 9 & 8 & 8 & 2 & 1 & 5 & 1 \\
        2 & 6 & 8 & 4 & 8 & 6 & 7 & 9 & 5 & 7 \\
    \end{array}
\]

\subsection*{Шаг №1}

\subsubsection*{1. Проводим редукцию матрицы по строкам}

В связи с этим во вновь полученной матрице в каждой строке будет как минимум один ноль.

\[
    \begin{array}{*{11}{c}}
        0 & 5 & 4 & 4 & 7 & 1 & 7 & 3 & 5 & 7 & \textbf{1} \\
        8 & 8 & 0 & 5 & 2 & 1 & 3 & 7 & 8 & 8 & \textbf{1} \\
        0 & 3 & 0 & 4 & 6 & 6 & 0 & 1 & 7 & 5 & \textbf{2} \\
        6 & 6 & 6 & 4 & 5 & 7 & 3 & 0 & 0 & 0 & \textbf{2} \\
        3 & 0 & 5 & 3 & 5 & 2 & 5 & 7 & 2 & 0 & \textbf{2} \\
        0 & 2 & 6 & 6 & 1 & 7 & 6 & 0 & 3 & 6 & \textbf{1} \\
        3 & 4 & 7 & 0 & 1 & 4 & 6 & 3 & 6 & 1 & \textbf{2} \\
        6 & 7 & 4 & 5 & 3 & 7 & 0 & 1 & 5 & 1 & \textbf{2} \\
        1 & 1 & 1 & 8 & 7 & 7 & 1 & 0 & 4 & 0 & \textbf{1} \\
        0 & 4 & 6 & 2 & 6 & 4 & 5 & 7 & 3 & 5 & \textbf{2} \\
    \end{array}
\]

Затем такую же операцию редукции проводим по столбцам, для чего в каждом столбце находим минимальный элемент.

\[
    \begin{array}{*{10}{c}}
        0 & 5 & 4 & 4 & 6 & 0 & 7 & 3 & 5 & 7 \\
        8 & 8 & 0 & 5 & 1 & 0 & 3 & 7 & 8 & 8 \\
        0 & 3 & 0 & 4 & 5 & 5 & 0 & 1 & 7 & 5 \\
        6 & 6 & 6 & 4 & 4 & 6 & 3 & 0 & 0 & 0 \\
        3 & 0 & 5 & 3 & 4 & 1 & 5 & 7 & 2 & 0 \\
        0 & 2 & 6 & 6 & 0 & 6 & 6 & 0 & 3 & 6 \\
        3 & 4 & 7 & 0 & 0 & 3 & 6 & 3 & 6 & 1 \\
        6 & 7 & 4 & 5 & 2 & 6 & 0 & 1 & 5 & 1 \\
        1 & 1 & 1 & 8 & 6 & 6 & 1 & 0 & 4 & 0 \\
        0 & 4 & 6 & 2 & 5 & 3 & 5 & 7 & 3 & 5 \\
    \end{array}
\]

После вычитания минимальных элементов получаем полностью редуцированную матрицу.

\subsubsection*{2. Методом проб и ошибок проводим поиск допустимого решения}

Фиксируем нулевое значение в клетке (1, 6). Другие нули в строке 1 и столбце 6 вычеркиваем. Для данной клетки вычеркиваем нули в клетках (2; 6), (1; 1).

\[
    \begin{array}{*{10}{c}}
        \text{[0]} & 5            & 4            & 4          & 6            & \textbf{[0]} & 7            & 3            & 5          & 7            \\
        8          & 8            & \textbf{[0]} & 5          & 1            & \text{[0]}   & 3            & 7            & 8          & 8            \\
        \text{[0]} & 3            & \text{[0]}   & 4          & 5            & 5            & \textbf{[0]} & 1            & 7          & 5            \\
        6          & 6            & 6            & 4          & 4            & 6            & 3            & \text{[0]}   & \text{[0]} & \textbf{[0]} \\
        3          & \textbf{[0]} & 5            & 3          & 4            & 1            & 5            & 7            & 2          & \text{[0]}   \\
        \text{[0]} & 2            & 6            & 6          & \text{[0]}   & 6            & 6            & \textbf{[0]} & 3          & 6            \\
        3          & 4            & 7            & \text{[0]} & \textbf{[0]} & 3            & 6            & 3            & 6          & 1            \\
        6          & 7            & 4            & 5          & 2            & 6            & \text{[0]}   & 1            & 5          & 1            \\
        1          & 1            & 1            & 8          & 6            & 6            & 1            & \text{[0]}   & 4          & \text{[0]}   \\
        0          & 4            & 6            & 2          & 5            & 3            & 5            & 7            & 3          & 5            \\
    \end{array}
\]

Поскольку расположение нулевых элементов в матрице не позволяет образовать систему из 10 независимых нулей (в матрице их только 7), то решение недопустимое.

\subsubsection*{3. Проводим модификацию матрицы}

Вычеркиваем строки и столбцы с возможно большим количеством нулевых элементов: столбец 1, строку 4, строку 2, столбец 5, строку 3, столбец 8, строку 5, столбец 4, строку 1, столбец 7, строку 9.

Получаем сокращенную матрицу (элементы выделены):

\[
    \begin{array}{*{10}{c}}
        0 & 5          & 4          & 4 & 6 & 0          & 7 & 3 & 5          & 7          \\
        8 & 8          & 0          & 5 & 1 & 0          & 3 & 7 & 8          & 8          \\
        0 & 3          & 0          & 4 & 5 & 5          & 0 & 1 & 7          & 5          \\
        6 & 6          & 6          & 4 & 4 & 6          & 3 & 0 & 0          & 0          \\
        3 & 0          & 5          & 3 & 4 & 1          & 5 & 7 & 2          & 0          \\
        0 & \textbf{2} & \textbf{6} & 6 & 0 & \textbf{6} & 6 & 0 & \textbf{3} & \textbf{6} \\
        3 & \textbf{4} & \textbf{7} & 0 & 0 & \textbf{3} & 6 & 3 & \textbf{6} & \textbf{1} \\
        6 & \textbf{7} & \textbf{4} & 5 & 2 & \textbf{6} & 0 & 1 & \textbf{5} & \textbf{1} \\
        1 & 1          & 1          & 8 & 6 & 6          & 1 & 0 & 4          & 0          \\
        0 & \textbf{4} & \textbf{6} & 2 & 5 & \textbf{3} & 5 & 7 & \textbf{3} & \textbf{5} \\
    \end{array}
\]

Минимальный элемент сокращенной матрицы (min(2, 6, 6, 3, 6, 4, 7, 3, 6, 1, 7, 4, 6, 5, 1, 4, 6, 3, 3, 5) = 1) вычитаем из всех ее элементов:

\[
    \begin{array}{*{10}{c}}
        0 & 5          & 4          & 4 & 6 & 0          & 7 & 3 & 5          & 7          \\
        8 & 8          & 0          & 5 & 1 & 0          & 3 & 7 & 8          & 8          \\
        0 & 3          & 0          & 4 & 5 & 5          & 0 & 1 & 7          & 5          \\
        6 & 6          & 6          & 4 & 4 & 6          & 3 & 0 & 0          & 0          \\
        3 & 0          & 5          & 3 & 4 & 1          & 5 & 7 & 2          & 0          \\
        0 & \textbf{1} & \textbf{5} & 6 & 0 & \textbf{5} & 6 & 0 & \textbf{2} & \textbf{5} \\
        3 & \textbf{3} & \textbf{6} & 0 & 0 & \textbf{2} & 6 & 3 & \textbf{5} & \textbf{0} \\
        6 & \textbf{6} & \textbf{3} & 5 & 2 & \textbf{5} & 0 & 1 & \textbf{4} & \textbf{0} \\
        1 & 1          & 1          & 8 & 6 & 6          & 1 & 0 & 4          & 0          \\
        0 & \textbf{3} & \textbf{5} & 2 & 5 & \textbf{2} & 5 & 7 & \textbf{2} & \textbf{4} \\
    \end{array}
\]

Затем складываем минимальный элемент с элементами, расположенными на пересечениях вычеркнутых строк и столбцов:

\[
    \begin{array}{*{10}{c}}
        \textbf{1} & 5 & 4 & \textbf{5} & \textbf{7} & 0 & \textbf{8} & \textbf{4} & 5 & 7 \\
        \textbf{9} & 8 & 0 & \textbf{6} & \textbf{2} & 0 & \textbf{4} & \textbf{8} & 8 & 8 \\
        \textbf{1} & 3 & 0 & \textbf{5} & \textbf{6} & 5 & \textbf{1} & \textbf{2} & 7 & 5 \\
        \textbf{7} & 6 & 6 & \textbf{5} & \textbf{5} & 6 & \textbf{4} & \textbf{1} & 0 & 0 \\
        \textbf{4} & 0 & 5 & \textbf{4} & \textbf{5} & 1 & \textbf{6} & \textbf{8} & 2 & 0 \\
        0          & 1 & 5 & 6          & 0          & 5 & 6          & 0          & 2 & 5 \\
        3          & 3 & 6 & 0          & 0          & 2 & 6          & 3          & 5 & 0 \\
        6          & 6 & 3 & 5          & 2          & 5 & 0          & 1          & 4 & 0 \\
        \textbf{2} & 1 & 1 & \textbf{9} & \textbf{7} & 6 & \textbf{2} & \textbf{1} & 4 & 0 \\
        0          & 3 & 5 & 2          & 5          & 2 & 5          & 7          & 3 & 5 \\
    \end{array}
\]

\subsection*{Шаг №2}

\subsubsection*{1. Проводим редукцию матрицы по строкам}

В связи с этим во вновь полученной матрице в каждой строке будет как минимум один ноль.

\[
    \begin{array}{*{10}{c}}
        1 & 5 & 4            & 5 & 7 & \textbf{[0]} & 8 & 4 & 5 & 7 \\
        9 & 8 & \textbf{[0]} & 6 & 2 & \text{[0]}   & 4 & 8 & 8 & 8 \\
        1 & 3 & \text{[0]}   & 5 & 6 & 5            & 1 & 2 & 7 & 5 \\
        7 & 6 & 6            & 5 & 5 & 6            & 4 & 1 & 0 & 0 \\
        4 & 0 & 5            & 4 & 5 & 1            & 6 & 8 & 2 & 0 \\
        0 & 1 & 5            & 6 & 0 & 5            & 6 & 0 & 2 & 5 \\
        3 & 3 & 6            & 0 & 0 & 2            & 6 & 3 & 5 & 0 \\
        6 & 6 & 3            & 5 & 2 & 5            & 0 & 1 & 4 & 0 \\
        2 & 1 & 1            & 9 & 7 & 6            & 2 & 1 & 4 & 0 \\
        0 & 3 & 5            & 2 & 5 & 2            & 5 & 7 & 3 & 4 \\
    \end{array}
\]

Поскольку расположение нулевых элементов в матрице не позволяет образовать систему из 10 независимых нулей (в матрице их только 2), то решение недопустимое.

\subsubsection*{3. Проводим модификацию матрицы}

Вычеркиваем строки и столбцы с возможно большим количеством нулевых элементов: столбец 10, строку 6, строку 2, строку 7, столбец 1, столбец 2, строку 1, столбец 3, строку 4, столбец 7.

Получаем сокращенную матрицу (элементы выделены):

\[
    \begin{array}{*{10}{c}}
        1 & 5 & 4 & 5          & 7          & 0          & 8 & 4          & 5          & 7 \\
        9 & 8 & 0 & 6          & 2          & 0          & 4 & 8          & 8          & 8 \\
        1 & 3 & 0 & \textbf{5} & \textbf{6} & \textbf{5} & 1 & \textbf{2} & \textbf{7} & 5 \\
        7 & 6 & 6 & 5          & 5          & 6          & 4 & 1          & 0          & 0 \\
        4 & 0 & 5 & \textbf{4} & \textbf{5} & \textbf{1} & 6 & \textbf{8} & \textbf{2} & 0 \\
        0 & 1 & 5 & 6          & 0          & 5          & 6 & 0          & 2          & 5 \\
        3 & 3 & 6 & 0          & 0          & 2          & 6 & 3          & 5          & 0 \\
        6 & 6 & 3 & \textbf{5} & \textbf{2} & \textbf{5} & 0 & \textbf{1} & \textbf{4} & 0 \\
        2 & 1 & 1 & \textbf{9} & \textbf{7} & \textbf{6} & 2 & \textbf{1} & \textbf{4} & 0 \\
        0 & 3 & 5 & \textbf{2} & \textbf{5} & \textbf{2} & 5 & \textbf{7} & \textbf{2} & 4 \\
    \end{array}
\]

Минимальный элемент сокращенной матрицы (min(5, 6, 5, 2, 7, 4, 5, 1, 8, 2, 5, 2, 5, 1, 4, 9, 7, 6, 1, 4, 2, 5, 2, 7, 2) = 1) вычитаем из всех ее элементов:

\[
    \begin{array}{*{10}{c}}
        1 & 5 & 4 & 5          & 7          & 0          & 8 & 4          & 5          & 7 \\
        9 & 8 & 0 & 6          & 2          & 0          & 4 & 8          & 8          & 8 \\
        1 & 3 & 0 & \textbf{4} & \textbf{5} & \textbf{4} & 1 & \textbf{1} & \textbf{6} & 5 \\
        7 & 6 & 6 & 5          & 5          & 6          & 4 & 1          & 0          & 0 \\
        4 & 0 & 5 & \textbf{3} & \textbf{4} & \textbf{0} & 6 & \textbf{7} & \textbf{1} & 0 \\
        0 & 1 & 5 & 6          & 0          & 5          & 6 & 0          & 2          & 5 \\
        3 & 3 & 6 & 0          & 0          & 2          & 6 & 3          & 5          & 0 \\
        6 & 6 & 3 & \textbf{4} & \textbf{1} & \textbf{4} & 0 & \textbf{0} & \textbf{3} & 0 \\
        2 & 1 & 1 & \textbf{8} & \textbf{6} & \textbf{5} & 2 & \textbf{0} & \textbf{3} & 0 \\
        0 & 3 & 5 & \textbf{1} & \textbf{4} & \textbf{1} & 5 & \textbf{6} & \textbf{1} & 4 \\
    \end{array}
\]

Затем складываем минимальный элемент с элементами, расположенными на пересечениях вычеркнутых строк и столбцов:

\[
    \begin{array}{*{10}{c}}
        \textbf{2}  & \textbf{6} & \textbf{5} & 5          & 7          & 0          & \textbf{9} & 4          & 5 & \textbf{8} \\
        \textbf{10} & \textbf{9} & \textbf{1} & 6          & 2          & 0          & \textbf{5} & 8          & 8 & \textbf{9} \\
        1           & 3          & 0          & 4          & \textbf{6} & \textbf{5} & 1          & \textbf{2} & 6 & \textbf{6} \\
        \textbf{8}  & \textbf{7} & \textbf{7} & 5          & 5          & 6          & \textbf{5} & 1          & 0 & \textbf{1} \\
        4           & 0          & 5          & 3          & \textbf{5} & \textbf{1} & 6          & \textbf{8} & 1 & \textbf{1} \\
        \textbf{1}  & \textbf{2} & \textbf{6} & 6          & 0          & \textbf{5} & \textbf{7} & 0          & 2 & \textbf{6} \\
        \textbf{4}  & \textbf{4} & \textbf{7} & 0          & \textbf{1} & \textbf{3} & \textbf{7} & 3          & 5 & \textbf{2} \\
        \textbf{7}  & \textbf{7} & \textbf{4} & 5          & \textbf{2} & \textbf{5} & 0          & \textbf{1} & 4 & \textbf{1} \\
        2           & \textbf{1} & \textbf{1} & \textbf{9} & \textbf{7} & \textbf{6} & 2          & \textbf{1} & 4 & \textbf{0} \\
        0           & \textbf{3} & \textbf{5} & 2          & 5          & \textbf{2} & 5          & \textbf{7} & 3 & \textbf{4} \\
    \end{array}
\]

\subsection*{Шаг №3}

\subsubsection*{1. Проводим редукцию матрицы по строкам}

В связи с этим во вновь полученной матрице в каждой строке будет как минимум один ноль.

\[
    \begin{array}{*{10}{c}}
        2  & 6 & 5 & 5 & 7 & \textbf{[0]} & 9 & 4 & 5 & 8 \\
        10 & 9 & 1 & 6 & 2 & \text{[0]}   & 5 & 8 & 8 & 9 \\
        1  & 3 & 0 & 4 & 5 & 4            & 1 & 1 & 6 & 5 \\
        8  & 7 & 7 & 5 & 5 & 6            & 5 & 1 & 0 & 1 \\
        4  & 0 & 5 & 3 & 4 & 1            & 6 & 8 & 1 & 0 \\
        1  & 2 & 6 & 6 & 0 & 5            & 7 & 0 & 2 & 6 \\
        4  & 4 & 7 & 0 & 0 & 2            & 7 & 3 & 5 & 1 \\
        7  & 7 & 4 & 5 & 2 & 5            & 0 & 1 & 4 & 1 \\
        2  & 1 & 1 & 9 & 6 & 6            & 2 & 1 & 4 & 0 \\
        0  & 3 & 5 & 2 & 5 & 2            & 5 & 7 & 3 & 4 \\
    \end{array}
\]

Поскольку расположение нулевых элементов в матрице не позволяет образовать систему из 10 независимых нулей (в матрице их только 1), то решение недопустимое.

\subsubsection*{3. Проводим модификацию матрицы}

Вычеркиваем строки и столбцы с возможно большим количеством нулевых элементов: строку 5, столбец 8, строку 7, столбец 6, строку 8, столбец 1, строку 3, столбец 5, строку 4, столбец 10.

Получаем сокращенную матрицу (элементы выделены):

\[
    \begin{array}{*{10}{c}}
        2  & \textbf{6} & \textbf{5} & \textbf{5} & 7          & 0          & \textbf{9} & 4          & \textbf{5} & 8 \\
        10 & \textbf{9} & \textbf{1} & \textbf{6} & 2          & 0          & \textbf{5} & 8          & \textbf{8} & 9 \\
        1  & 3          & 0          & 4          & 5          & 4          & 1          & 1          & 6          & 5 \\
        8  & 7          & 7          & 5          & 5          & 6          & 5          & 1          & 0          & 1 \\
        4  & 0          & 5          & 3          & 4          & 1          & 6          & 8          & 1          & 0 \\
        1  & \textbf{2} & \textbf{6} & \textbf{6} & 0          & \textbf{5} & \textbf{7} & 0          & \textbf{2} & 6 \\
        4  & 4          & 7          & 0          & 0          & 2          & 7          & 3          & 5          & 1 \\
        7  & 7          & 4          & \textbf{5} & \textbf{2} & \textbf{5} & 0          & \textbf{1} & 4          & 1 \\
        2  & \textbf{1} & \textbf{1} & \textbf{9} & 6          & \textbf{6} & 2          & \textbf{1} & 4          & 0 \\
        0  & \textbf{3} & \textbf{5} & \textbf{2} & 5          & \textbf{2} & 5          & \textbf{7} & 3          & 4 \\
    \end{array}
\]

Минимальный элемент сокращенной матрицы (min(6, 5, 5, 9, 5, 9, 1, 6, 5, 8, 2, 6, 6, 7, 2, 1, 1, 8, 2, 3, 3, 5, 1, 5, 1) = 1) вычитаем из всех ее элементов:

\[
    \begin{array}{*{10}{c}}
        2  & \textbf{5} & \textbf{4} & \textbf{4} & 7          & 0          & \textbf{8} & 4          & \textbf{4} & 8 \\
        10 & \textbf{8} & \textbf{0} & \textbf{5} & 2          & 0          & \textbf{4} & 8          & \textbf{7} & 9 \\
        1  & 3          & 0          & 4          & 5          & 4          & 1          & 1          & 6          & 5 \\
        8  & 7          & 7          & 5          & 5          & 6          & 5          & 1          & 0          & 1 \\
        4  & 0          & 5          & 3          & 4          & 1          & 6          & 8          & 1          & 0 \\
        1  & \textbf{1} & \textbf{5} & \textbf{5} & 0          & \textbf{5} & \textbf{6} & 0          & \textbf{1} & 6 \\
        4  & 4          & 7          & 0          & 0          & 2          & 7          & 3          & 5          & 1 \\
        7  & 7          & 4          & \textbf{4} & \textbf{1} & \textbf{4} & 0          & \textbf{0} & \textbf{3} & 1 \\
        2  & \textbf{0} & \textbf{0} & \textbf{8} & 6          & \textbf{5} & 2          & \textbf{0} & 4          & 0 \\
        0  & \textbf{2} & \textbf{4} & \textbf{1} & 4          & \textbf{1} & 5          & \textbf{6} & \textbf{0} & 4 \\
    \end{array}
\]

Затем складываем минимальный элемент с элементами, расположенными на пересечениях вычеркнутых строк и столбцов:

\[
    \begin{array}{*{10}{c}}
        2          & 5 & 4 & 4 & 7          & 0          & 8 & 4          & 4 & 8          \\
        10         & 8 & 0 & 5 & 2          & 0          & 4 & 8          & 7 & 9          \\
        \textbf{2} & 3 & 0 & 4 & \textbf{6} & \textbf{5} & 1 & \textbf{2} & 6 & \textbf{6} \\
        \textbf{9} & 7 & 7 & 5 & \textbf{6} & \textbf{7} & 5 & \textbf{2} & 0 & \textbf{2} \\
        \textbf{5} & 0 & 5 & 3 & \textbf{5} & \textbf{1} & 6 & \textbf{8} & 1 & \textbf{1} \\
        1          & 1 & 5 & 5 & 0          & 5          & 6 & 0          & 1 & 6          \\
        \textbf{5} & 4 & 7 & 0 & \textbf{1} & \textbf{3} & 7 & \textbf{4} & 5 & \textbf{2} \\
        \textbf{8} & 7 & 4 & 5 & \textbf{2} & \textbf{5} & 0 & \textbf{1} & 4 & \textbf{1} \\
        2          & 0 & 0 & 8 & 6          & 5          & 1 & 0          & 4 & 0          \\
        0          & 2 & 4 & 1 & 4          & 1          & 5 & 6          & 1 & 4          \\
    \end{array}
\]

\subsection*{Шаг №4}

\textbf{1. Проводим редукцию матрицы по строкам.} В связи с этим во вновь полученной матрице в каждой строке будет как минимум один ноль.
Затем такую же операцию редукции проводим по столбцам, для чего в каждом столбце находим минимальный элемент.
После вычитания минимальных элементов получаем полностью редуцированную матрицу.

\textbf{2. Методом проб и ошибок} проводим поиск допустимого решения, для которого все назначения имеют нулевую стоимость.
Фиксируем нулевое значение в клетке (1, 6). Другие нули в строке 1 и столбце 6 вычеркиваем. Для данной клетки вычеркиваем нули в клетках (2; 6).
Фиксируем нулевое значение в клетке (2, 3). Другие нули в строке 2 и столбце 3 вычеркиваем. Для данной клетки вычеркиваем нули в клетках (3; 3), (9; 3).

В итоге получаем следующую матрицу:

\[
    \begin{array}{|*{10}{c|}}
        \hline
        2  & 5 & 4            & 4 & 7 & \textbf{[0]} & 8 & 4 & 4 & 8 \\
        \hline
        10 & 8 & \textbf{[0]} & 5 & 2 & [-0-]        & 4 & 8 & 7 & 9 \\
        \hline
        2  & 3 & [-0-]        & 4 & 6 & 5            & 1 & 2 & 6 & 6 \\
        \hline
        9  & 7 & 7            & 5 & 6 & 7            & 5 & 2 & 0 & 2 \\
        \hline
        5  & 0 & 5            & 3 & 5 & 1            & 6 & 8 & 1 & 1 \\
        \hline
        1  & 1 & 5            & 5 & 0 & 5            & 6 & 0 & 1 & 6 \\
        \hline
        5  & 4 & 7            & 0 & 1 & 3            & 7 & 4 & 5 & 2 \\
        \hline
        7  & 6 & 3            & 4 & 2 & 5            & 0 & 1 & 3 & 1 \\
        \hline
        2  & 0 & [-0-]        & 7 & 6 & 5            & 1 & 0 & 2 & 0 \\
        \hline
        0  & 2 & 4            & 0 & 4 & 1            & 4 & 6 & 0 & 4 \\
        \hline
    \end{array}
\]

Поскольку расположение нулевых элементов в матрице не позволяет образовать систему из 10 независимых нулей (в матрице их только 2), то \textbf{решение недопустимое}.

\textbf{3. Проводим модификацию матрицы.} Вычеркиваем строки и столбцы с возможно большим количеством нулевых элементов:
строку 9, строку 10, столбец 3, столбец 6, строку 6, столбец 2, строку 4, столбец 4, строку 8.

Получаем сокращенную матрицу (элементы выделены):

\[
    \begin{array}{|*{10}{c|}}
        \hline
        \textbf{2}  & 5 & 4 & 4 & \textbf{7} & 0 & \textbf{8} & \textbf{4} & \textbf{4} & \textbf{8} \\
        \hline
        \textbf{10} & 8 & 0 & 5 & \textbf{2} & 0 & \textbf{4} & \textbf{8} & \textbf{7} & \textbf{9} \\
        \hline
        \textbf{2}  & 3 & 0 & 4 & \textbf{6} & 5 & \textbf{1} & \textbf{2} & \textbf{6} & \textbf{6} \\
        \hline
        9           & 7 & 7 & 5 & 6          & 7 & 5          & 2          & 0          & 2          \\
        \hline
        \textbf{5}  & 0 & 5 & 3 & \textbf{5} & 1 & \textbf{6} & \textbf{8} & \textbf{1} & \textbf{1} \\
        \hline
        1           & 1 & 5 & 5 & 0          & 5 & 6          & 0          & 1          & 6          \\
        \hline
        \textbf{5}  & 4 & 7 & 0 & \textbf{1} & 3 & \textbf{7} & \textbf{4} & \textbf{5} & \textbf{2} \\
        \hline
        7           & 6 & 3 & 4 & 2          & 5 & 0          & 1          & 3          & 1          \\
        \hline
        2           & 0 & 0 & 7 & 6          & 5 & 1          & 0          & 2          & 0          \\
        \hline
        0           & 2 & 4 & 0 & 4          & 1 & 4          & 6          & 0          & 4          \\
        \hline
    \end{array}
\]

Минимальный элемент сокращенной матрицы (\(\min(2, 7, 8, 4, 4, 8, 10, 2, 4, 8, 7, 9, 2, 6, 1, 2, 6, 6, 5, 5, 6, 8, 1, 1, 5, 1, 7, 4, 5, 2) = 1\)) вычитаем из всех ее элементов:

\[
    \begin{array}{|*{10}{c|}}
        \hline
        \textbf{1} & 5 & 4 & 4 & \textbf{6} & 0 & \textbf{7} & \textbf{3} & \textbf{3} & \textbf{7} \\
        \hline
        \textbf{9} & 8 & 0 & 5 & \textbf{1} & 0 & \textbf{3} & \textbf{7} & \textbf{6} & \textbf{8} \\
        \hline
        \textbf{1} & 3 & 0 & 4 & \textbf{5} & 5 & \textbf{0} & \textbf{1} & \textbf{5} & \textbf{5} \\
        \hline
        9          & 7 & 7 & 5 & 6          & 7 & 5          & 2          & 0          & 2          \\
        \hline
        \textbf{4} & 0 & 5 & 3 & \textbf{4} & 1 & \textbf{5} & \textbf{7} & \textbf{0} & \textbf{0} \\
        \hline
        1          & 1 & 5 & 5 & 0          & 5 & 6          & 0          & 1          & 6          \\
        \hline
        \textbf{4} & 4 & 7 & 0 & \textbf{0} & 3 & \textbf{6} & \textbf{3} & \textbf{4} & \textbf{1} \\
        \hline
        7          & 6 & 3 & 4 & 2          & 5 & 0          & 1          & 3          & 1          \\
        \hline
        2          & 0 & 0 & 7 & 6          & 5 & 1          & 0          & 2          & 0          \\
        \hline
        0          & 2 & 4 & 0 & 4          & 1 & 4          & 6          & 0          & 4          \\
        \hline
    \end{array}
\]

Затем складываем минимальный элемент с элементами, расположенными на пересечениях вычеркнутых строк и столбцов:

\[
    \begin{array}{|*{10}{c|}}
        \hline
        1 & 5          & 4          & 4          & 6 & 0          & 7 & 3 & 3 & 7 \\
        \hline
        9 & 8          & 0          & 5          & 1 & 0          & 3 & 7 & 6 & 8 \\
        \hline
        1 & 3          & 0          & 4          & 5 & 5          & 0 & 1 & 5 & 5 \\
        \hline
        9 & \textbf{8} & \textbf{8} & \textbf{6} & 6 & \textbf{8} & 5 & 2 & 0 & 2 \\
        \hline
        4 & 0          & 5          & 3          & 4 & 1          & 5 & 7 & 0 & 0 \\
        \hline
        1 & \textbf{2} & \textbf{6} & \textbf{6} & 0 & \textbf{6} & 6 & 0 & 1 & 6 \\
        \hline
        4 & 4          & 7          & 0          & 0 & 3          & 6 & 3 & 4 & 1 \\
        \hline
        7 & \textbf{7} & \textbf{4} & \textbf{5} & 2 & \textbf{6} & 0 & 1 & 3 & 1 \\
        \hline
        2 & \textbf{1} & \textbf{1} & \textbf{8} & 6 & \textbf{6} & 1 & 0 & 2 & 0 \\
        \hline
        0 & \textbf{3} & \textbf{5} & \textbf{1} & 4 & \textbf{2} & 4 & 6 & 0 & 4 \\
        \hline
    \end{array}
\]

\subsection*{Шаг №5}

\textbf{1. Проводим редукцию матрицы по строкам.} В связи с этим во вновь полученной матрице в каждой строке будет как минимум один ноль.
Затем такую же операцию редукции проводим по столбцам, для чего в каждом столбце находим минимальный элемент.
После вычитания минимальных элементов получаем полностью редуцированную матрицу.

\textbf{2. Методом проб и ошибок} проводим поиск допустимого решения, для которого все назначения имеют нулевую стоимость.
Фиксируем нулевое значение в клетке (1, 6). Другие нули в строке 1 и столбце 6 вычеркиваем. Для данной клетки вычеркиваем нули в клетках (2; 6).
Фиксируем нулевое значение в клетке (2, 3). Другие нули в строке 2 и столбце 3 вычеркиваем. Для данной клетки вычеркиваем нули в клетках (3; 3).

В итоге получаем следующую матрицу:

\[
    \begin{array}{|*{10}{c|}}
        \hline
        1 & 5     & 4            & 4     & 6            & \textbf{[0]} & 7            & 3            & 3            & 7            \\
        \hline
        9 & 8     & \textbf{[0]} & 5     & 1            & [-0-]        & 3            & 7            & 6            & 8            \\
        \hline
        1 & 3     & [-0-]        & 4     & 5            & 5            & \textbf{[0]} & 1            & 5            & 5            \\
        \hline
        9 & 8     & 8            & 6     & 6            & 8            & 5            & 2            & \textbf{[0]} & 2            \\
        \hline
        4 & [-0-] & 5            & 3     & 4            & 1            & 5            & 7            & [-0-]        & \textbf{[0]} \\
        \hline
        1 & 2     & 6            & 6     & [-0-]        & 6            & 6            & \textbf{[0]} & 1            & 6            \\
        \hline
        4 & 4     & 7            & [-0-] & \textbf{[0]} & 3            & 6            & 3            & 4            & 1            \\
        \hline
        7 & 7     & 4            & 5     & 2            & 6            & [-0-]        & 1            & 3            & 1            \\
        \hline
        2 & 1     & 1            & 8     & 6            & 6            & 1            & [-0-]        & 2            & [-0-]        \\
        \hline
        0 & 3     & 5            & 1     & 4            & 2            & 4            & 6            & [-0-]        & 4            \\
        \hline
    \end{array}
\]

Поскольку расположение нулевых элементов в матрице не позволяет образовать систему из 10 независимых нулей (в матрице их только 7), то \textbf{решение недопустимое}.

\textbf{3. Проводим модификацию матрицы.} Вычеркиваем строки и столбцы с возможно большим количеством нулевых элементов:
строку 5, столбец 3, строку 6, столбец 6, строку 7, столбец 7, строку 9, столбец 9, строку 10.

Получаем сокращенную матрицу (элементы выделены):

\[
    \begin{array}{|*{10}{c|}}
        \hline
        \textbf{1} & \textbf{5} & 4 & \textbf{4} & \textbf{6} & 0 & 7 & \textbf{3} & 3 & \textbf{7} \\
        \hline
        \textbf{9} & \textbf{8} & 0 & \textbf{5} & \textbf{1} & 0 & 3 & \textbf{7} & 6 & \textbf{8} \\
        \hline
        \textbf{1} & \textbf{3} & 0 & \textbf{4} & \textbf{5} & 5 & 0 & \textbf{1} & 5 & \textbf{5} \\
        \hline
        \textbf{9} & \textbf{8} & 8 & \textbf{6} & \textbf{6} & 8 & 5 & \textbf{2} & 0 & \textbf{2} \\
        \hline
        4          & 0          & 5 & 3          & 4          & 1 & 5 & 7          & 0 & 0          \\
        \hline
        1          & 2          & 6 & 6          & 0          & 6 & 6 & 0          & 1 & 6          \\
        \hline
        4          & 4          & 7 & 0          & 0          & 3 & 6 & 3          & 4 & 1          \\
        \hline
        \textbf{7} & \textbf{7} & 4 & \textbf{5} & \textbf{2} & 6 & 0 & \textbf{1} & 3 & \textbf{1} \\
        \hline
        2          & 1          & 1 & 8          & 6          & 6 & 1 & 0          & 2 & 0          \\
        \hline
        0          & 3          & 5 & 1          & 4          & 2 & 4 & 6          & 0 & 4          \\
        \hline
    \end{array}
\]

Минимальный элемент сокращенной матрицы (\(\min(1, 5, 4, 6, 3, 7, 9, 8, 5, 1, 7, 8, 1, 3, 4, 5, 1, 5, 9, 8, 6, 6, 2, 2, 7, 7, 5, 2, 1, 1) = 1\)) вычитаем из всех ее элементов:

\[
    \begin{array}{|*{10}{c|}}
        \hline
        \textbf{0} & \textbf{4} & 4 & \textbf{3} & \textbf{5} & 0 & 7 & \textbf{2} & 3 & \textbf{6} \\
        \hline
        \textbf{8} & \textbf{7} & 0 & \textbf{4} & \textbf{0} & 0 & 3 & \textbf{6} & 6 & \textbf{7} \\
        \hline
        \textbf{0} & \textbf{2} & 0 & \textbf{3} & \textbf{4} & 5 & 0 & \textbf{0} & 5 & \textbf{4} \\
        \hline
        \textbf{8} & \textbf{7} & 8 & \textbf{5} & \textbf{5} & 8 & 5 & \textbf{1} & 0 & \textbf{1} \\
        \hline
        4          & 0          & 5 & 3          & 4          & 1 & 5 & 7          & 0 & 0          \\
        \hline
        1          & 2          & 6 & 6          & 0          & 6 & 6 & 0          & 1 & 6          \\
        \hline
        4          & 4          & 7 & 0          & 0          & 3 & 6 & 3          & 4 & 1          \\
        \hline
        \textbf{6} & \textbf{6} & 4 & \textbf{4} & \textbf{1} & 6 & 0 & \textbf{0} & 3 & \textbf{0} \\
        \hline
        2          & 1          & 1 & 8          & 6          & 6 & 1 & 0          & 2 & 0          \\
        \hline
        0          & 3          & 5 & 1          & 4          & 2 & 4 & 6          & 0 & 4          \\
        \hline
    \end{array}
\]

Затем складываем минимальный элемент с элементами, расположенными на пересечениях вычеркнутых строк и столбцов:

\[
    \begin{array}{|*{10}{c|}}
        \hline
        0 & 4 & 4          & 3 & 5 & 0          & 7          & 2 & 3          & 6 \\
        \hline
        8 & 7 & 0          & 4 & 0 & 0          & 3          & 6 & 6          & 7 \\
        \hline
        0 & 2 & 0          & 3 & 4 & 5          & 0          & 0 & 5          & 4 \\
        \hline
        8 & 7 & 8          & 5 & 5 & 8          & 5          & 1 & 0          & 1 \\
        \hline
        4 & 0 & \textbf{6} & 3 & 4 & \textbf{2} & \textbf{6} & 7 & \textbf{1} & 0 \\
        \hline
        1 & 2 & \textbf{7} & 6 & 0 & \textbf{7} & \textbf{7} & 0 & \textbf{2} & 6 \\
        \hline
        4 & 4 & \textbf{8} & 0 & 0 & \textbf{4} & \textbf{7} & 3 & \textbf{5} & 1 \\
        \hline
        6 & 6 & 4          & 4 & 1 & 6          & 0          & 0 & 3          & 0 \\
        \hline
        2 & 1 & \textbf{2} & 8 & 6 & \textbf{7} & \textbf{2} & 0 & \textbf{3} & 0 \\
        \hline
        0 & 3 & \textbf{6} & 1 & 4 & \textbf{3} & \textbf{5} & 6 & \textbf{1} & 4 \\
        \hline
    \end{array}
\]

\subsection*{Шаг №6}

\textbf{1. Проводим редукцию матрицы по строкам.} В связи с этим во вновь полученной матрице в каждой строке будет как минимум один ноль.
Затем такую же операцию редукции проводим по столбцам, для чего в каждом столбце находим минимальный элемент.
После вычитания минимальных элементов получаем полностью редуцированную матрицу.

\textbf{2. Методом проб и ошибок} проводим поиск допустимого решения, для которого все назначения имеют нулевую стоимость.
Фиксируем нулевое значение в клетке (1, 6). Другие нули в строке 1 и столбце 6 вычеркиваем. Для данной клетки вычеркиваем нули в клетках (2; 6), (1; 1).

В итоге получаем следующую матрицу:

\[
    \begin{array}{|*{10}{c|}}
        \hline
        [-0-]        & 4            & 4            & 3            & 5            & \textbf{[0]} & 7            & 2     & 3            & 6            \\
        \hline
        8            & 7            & \textbf{[0]} & 4            & [-0-]        & [-0-]        & 3            & 6     & 6            & 7            \\
        \hline
        [-0-]        & 2            & [-0-]        & 3            & 4            & 5            & \textbf{[0]} & [-0-] & 5            & 4            \\
        \hline
        8            & 7            & 8            & 5            & 5            & 8            & 5            & 1     & \textbf{[0]} & 1            \\
        \hline
        4            & \textbf{[0]} & 6            & 3            & 4            & 2            & 6            & 7     & 1            & [-0-]        \\
        \hline
        1            & 2            & 7            & 6            & \textbf{[0]} & 7            & 7            & [-0-] & 2            & 6            \\
        \hline
        4            & 4            & 8            & \textbf{[0]} & [-0-]        & 4            & 7            & 3     & 5            & 1            \\
        \hline
        6            & 6            & 4            & 4            & 1            & 6            & \textbf{[0]} & [-0-] & 3            & [-0-]        \\
        \hline
        2            & 1            & 2            & 8            & 6            & 7            & 2            & [-0-] & 3            & \textbf{[0]} \\
        \hline
        \textbf{[0]} & 3            & 6            & 1            & 4            & 3            & 5            & 6     & 1            & 4            \\
        \hline
    \end{array}
\]

Количество найденных нулей равно \( k = 10 \). В результате получаем эквивалентную матрицу \( C_e \):

\[
    \begin{array}{|*{10}{c|}}
        \hline
        0 & 4 & 4 & 3 & 5 & 0 & 7 & 2 & 3 & 6 \\
        \hline
        8 & 7 & 0 & 4 & 0 & 0 & 3 & 6 & 6 & 7 \\
        \hline
        0 & 2 & 0 & 3 & 4 & 5 & 0 & 0 & 5 & 4 \\
        \hline
        8 & 7 & 8 & 5 & 5 & 8 & 5 & 1 & 0 & 1 \\
        \hline
        4 & 0 & 6 & 3 & 4 & 2 & 6 & 7 & 1 & 0 \\
        \hline
        1 & 2 & 7 & 6 & 0 & 7 & 7 & 0 & 2 & 6 \\
        \hline
        4 & 4 & 8 & 0 & 0 & 4 & 7 & 3 & 5 & 1 \\
        \hline
        6 & 6 & 4 & 4 & 1 & 6 & 0 & 0 & 3 & 0 \\
        \hline
        2 & 1 & 2 & 8 & 6 & 7 & 2 & 0 & 3 & 0 \\
        \hline
        0 & 3 & 6 & 1 & 4 & 3 & 5 & 6 & 1 & 4 \\
        \hline
    \end{array}
\]

\textbf{4. Методом проб и ошибок определяем матрицу назначения \( X \)}, которая позволяет по аналогично расположенным элементам исходной матрицы (в квадратах) вычислить минимальную стоимость назначения.

\[
    \begin{array}{|*{10}{c|}}
        \hline
        [-0-]        & 4            & 4            & 3            & 5            & \textbf{[0]} & 7            & 2            & 3            & 6            \\
        \hline
        8            & 7            & \textbf{[0]} & 4            & [-0-]        & [-0-]        & 3            & 6            & 6            & 7            \\
        \hline
        [-0-]        & 2            & [-0-]        & 3            & 4            & 5            & [-0-]        & \textbf{[0]} & 5            & 4            \\
        \hline
        8            & 7            & 8            & 5            & 5            & 8            & 5            & 1            & \textbf{[0]} & 1            \\
        \hline
        4            & \textbf{[0]} & 6            & 3            & 4            & 2            & 6            & 7            & 1            & [-0-]        \\
        \hline
        1            & 2            & 7            & 6            & \textbf{[0]} & 7            & 7            & [-0-]        & 2            & 6            \\
        \hline
        4            & 4            & 8            & \textbf{[0]} & [-0-]        & 4            & 7            & 3            & 5            & 1            \\
        \hline
        6            & 6            & 4            & 4            & 1            & 6            & \textbf{[0]} & [-0-]        & 3            & [-0-]        \\
        \hline
        2            & 1            & 2            & 8            & 6            & 7            & 2            & [-0-]        & 3            & \textbf{[0]} \\
        \hline
        \textbf{[0]} & 3            & 6            & 1            & 4            & 3            & 5            & 6            & 1            & 4            \\
        \hline
    \end{array}
\]

Ответ:
\[
    C_{\text{min}} = 2 + 2 + 2 + 2 + 2 + 1 + 2 + 2 + 1 + 3 = 19
\]
Путь: (4;9), (5;2), (7;4), (1;6), (10;1), (2;3), (6;5), (8;7), (9;10), (3;8)
или цикл: (4;9), (9;10), (10;1), (1;6), (6;5), (5;2), (2;3), (3;8), (8;7), (7;4)

\clearpage


\subsection*{Решение вручную}

\textbf{Задача коммивояжера}.
Возьмем в качестве произвольного маршрута:
\[ X_0 = (1,2);(2,3);(3,4);(4,5);(5,6);(6,7);(7,8);(8,9);(9,10);(10,1) \]
Тогда \( F(X_0) = 34 + 56 + 91 + 83 + 51 + 77 + 26 + 44 + 87 + 58 = 607 \).

Для определения нижней границы множества воспользуемся \textbf{операцией редукции} или приведения матрицы по строкам, для чего необходимо в каждой строке матрицы \( D \) найти минимальный элемент.
\[ d_i = \min_j d_{ij} \]

\begin{table}[H]
    \centering
    \begin{tabular}{|c|c|c|c|c|c|c|c|c|c|c|c|}
        \hline
        \textbf{i j} & \bfseries 1 & \bfseries 2 & \bfseries 3 & \bfseries 4 & \bfseries 5 & \bfseries 6 & \bfseries 7 & \bfseries 8 & \bfseries 9 & \bfseries 10 & \bfseries d_i \\ \hline\textbf{1}   & M          & 34         & 68         & 18         & 63         & 80         & 12         & 44         & 58         & 87          & 12          \\ \hline
        \textbf{2}   & 56          & M           & 56          & 94          & 62          & 65          & 18          & 38          & 67          & 22           & 18            \\ \hline
        \textbf{3}   & 34          & 53          & M           & 91          & 13          & 73          & 70          & 51          & 13          & 37           & 13            \\ \hline
        \textbf{4}   & 28          & 19          & 14          & M           & 83          & 89          & 25          & 9           & 89          & 22           & 9             \\ \hline
        \textbf{5}   & 67          & 13          & 1           & 19          & M           & 51          & 7           & 13          & 31          & 4            & 1             \\ \hline
        \textbf{6}   & 3           & 78          & 24          & 90          & 14          & M           & 77          & 6           & 35          & 69           & 3             \\ \hline
        \textbf{7}   & 96          & 32          & 100         & 4           & 8           & 19          & M           & 26          & 37          & 36           & 4             \\ \hline
        \textbf{8}   & 79          & 2           & 48          & 25          & 63          & 99          & 17          & M           & 44          & 45           & 2             \\ \hline
        \textbf{9}   & 97          & 49          & 33          & 74          & 23          & 72          & 23          & 73          & M           & 87           & 23            \\ \hline
        \textbf{10}  & 58          & 83          & 24          & 39          & 17          & 76          & 64          & 78          & 100         & M            & 17            \\ \hline
    \end{tabular}
\end{table}

Затем вычитаем \( d_i \) из элементов рассматриваемой строки. В связи с этим во вновь полученной матрице в каждой строке будет как минимум один ноль.

\begin{table}[H]
    \centering
    \begin{tabular}{|c|c|c|c|c|c|c|c|c|c|c|}
        \hline
        \textbf{i j} & \textbf{1} & \textbf{2} & \textbf{3} & \textbf{4} & \textbf{5} & \textbf{6} & \textbf{7} & \textbf{8} & \textbf{9} & \textbf{10} \\ \hline
        \textbf{1}   & M          & 22         & 56         & 6          & 51         & 68         & 0          & 32         & 46         & 75          \\ \hline
        \textbf{2}   & 38         & M          & 38         & 76         & 44         & 47         & 0          & 20         & 49         & 4           \\ \hline
        \textbf{3}   & 21         & 40         & M          & 78         & 0          & 60         & 57         & 38         & 0          & 24          \\ \hline
        \textbf{4}   & 19         & 10         & 5          & M          & 74         & 80         & 16         & 0          & 80         & 13          \\ \hline
        \textbf{5}   & 66         & 12         & 0          & 18         & M          & 50         & 6          & 12         & 30         & 3           \\ \hline
        \textbf{6}   & 0          & 75         & 21         & 87         & 11         & M          & 74         & 3          & 32         & 66          \\ \hline
        \textbf{7}   & 92         & 28         & 96         & 0          & 4          & 15         & M          & 22         & 33         & 32          \\ \hline
        \textbf{8}   & 77         & 0          & 46         & 23         & 61         & 97         & 15         & M          & 42         & 43          \\ \hline
        \textbf{9}   & 74         & 26         & 10         & 51         & 0          & 49         & 0          & 50         & M          & 64          \\ \hline
        \textbf{10}  & 41         & 66         & 7          & 22         & 0          & 59         & 47         & 61         & 83         & M           \\ \hline
    \end{tabular}
\end{table}

Такую же операцию редукции проводим по столбцам, для чего в каждом столбце находим минимальный элемент:
\[ d_j = \min_i d_{ij} \]

\begin{table}[H]
    \centering
    \begin{tabular}{|c|c|c|c|c|c|c|c|c|c|c|}
        \hline
        \textbf{i j} & \textbf{1} & \textbf{2} & \textbf{3} & \textbf{4} & \textbf{5} & \textbf{6} & \textbf{7} & \textbf{8} & \textbf{9} & \textbf{10} \\ \hline
        \textbf{1}   & M          & 22         & 56         & 6          & 51         & 68         & 0          & 32         & 46         & 75          \\ \hline
        \textbf{2}   & 38         & M          & 38         & 76         & 44         & 47         & 0          & 20         & 49         & 4           \\ \hline
        \textbf{3}   & 21         & 40         & M          & 78         & 0          & 60         & 57         & 38         & 0          & 24          \\ \hline
        \textbf{4}   & 19         & 10         & 5          & M          & 74         & 80         & 16         & 0          & 80         & 13          \\ \hline
        \textbf{5}   & 66         & 12         & 0          & 18         & M          & 50         & 6          & 12         & 30         & 3           \\ \hline
        \textbf{6}   & 0          & 75         & 21         & 87         & 11         & M          & 74         & 3          & 32         & 66          \\ \hline
        \textbf{7}   & 92         & 28         & 96         & 0          & 4          & 15         & M          & 22         & 33         & 32          \\ \hline
        \textbf{8}   & 77         & 0          & 46         & 23         & 61         & 97         & 15         & M          & 42         & 43          \\ \hline
        \textbf{9}   & 74         & 26         & 10         & 51         & 0          & 49         & 0          & 50         & M          & 64          \\ \hline
        \textbf{10}  & 41         & 66         & 7          & 22         & 0          & 59         & 47         & 61         & 83         & M           \\ \hline
        \textbf{d}_j & 0          & 0          & 0          & 0          & 0          & 15         & 0          & 0          & 0          & 3           \\ \hline
    \end{tabular}
\end{table}

После вычитания минимальных элементов получаем полностью редуцированную матрицу, где величины \( d_i \) и \( d_j \) называются \textbf{константами приведения}.

\begin{table}[H]
    \centering
    \begin{tabular}{|c|c|c|c|c|c|c|c|c|c|c|}
        \hline
        \textbf{i j} & \textbf{1} & \textbf{2} & \textbf{3} & \textbf{4} & \textbf{5} & \textbf{6} & \textbf{7} & \textbf{8} & \textbf{9} & \textbf{10} \\ \hline
        \textbf{1}   & M          & 22         & 56         & 6          & 51         & 53         & 0          & 32         & 46         & 72          \\ \hline
        \textbf{2}   & 38         & M          & 38         & 76         & 44         & 32         & 0          & 20         & 49         & 1           \\ \hline
        \textbf{3}   & 21         & 40         & M          & 78         & 0          & 45         & 57         & 38         & 0          & 21          \\ \hline
        \textbf{4}   & 19         & 10         & 5          & M          & 74         & 65         & 16         & 0          & 80         & 10          \\ \hline
        \textbf{5}   & 66         & 12         & 0          & 18         & M          & 35         & 6          & 12         & 30         & 0           \\ \hline
        \textbf{6}   & 0          & 75         & 21         & 87         & 11         & M          & 74         & 3          & 32         & 63          \\ \hline
        \textbf{7}   & 92         & 28         & 96         & 0          & 4          & 0          & M          & 22         & 33         & 29          \\ \hline
        \textbf{8}   & 77         & 0          & 46         & 23         & 61         & 82         & 15         & M          & 42         & 40          \\ \hline
        \textbf{9}   & 74         & 26         & 10         & 51         & 0          & 34         & 0          & 50         & M          & 61          \\ \hline
        \textbf{10}  & 41         & 66         & 7          & 22         & 0          & 44         & 47         & 61         & 83         & M           \\ \hline
    \end{tabular}
\end{table}

Сумма констант приведения определяет нижнюю границу \( H \):
\[ H = \sum d_i + \sum d_j \]
\[ H = 12 + 18 + 13 + 9 + 1 + 3 + 4 + 2 + 23 + 17 + 0 + 0 + 0 + 0 + 0 + 15 + 0 + 0 + 0 + 3 = 120 \]

\textbf{Шаг №1}.
\textbf{Определяем ребро ветвления} и разобьем все множество маршрутов относительно этого ребра на два подмножества \( (i,j) \) и \( (i^*,j^*) \).
С этой целью для всех клеток матрицы с нулевыми элементами заменяем поочередно нули на \( M \) (бесконечность) и определяем для них сумму образовавшихся констант приведения, они приведены в скобках.

\begin{table}[H]
    \centering
    \begin{tabular}{|c|c|c|c|c|c|c|c|c|c|c|c|}
        \hline
        \textbf{i j} & \textbf{1} & \textbf{2} & \textbf{3} & \textbf{4} & \textbf{5} & \textbf{6}     & \textbf{7} & \textbf{8} & \textbf{9} & \textbf{10} & \textbf{d}_i \\ \hline
        \textbf{1}   & M          & 22         & 56         & 6          & 51         & 53             & 0(6)       & 32         & 46         & 72          & 6            \\ \hline
        \textbf{2}   & 38         & M          & 38         & 76         & 44         & 32             & 0(1)       & 20         & 49         & 1           & 1            \\ \hline
        \textbf{3}   & 21         & 40         & M          & 78         & 0(0)       & 45             & 57         & 38         & 0(30)      & 21          & 0            \\ \hline
        \textbf{4}   & 19         & 10         & 5          & M          & 74         & 65             & 16         & 0(8)       & 80         & 10          & 5            \\ \hline
        \textbf{5}   & 66         & 12         & 0(5)       & 18         & M          & 35             & 6          & 12         & 30         & 0(1)        & 0            \\ \hline
        \textbf{6}   & 0(22)      & 75         & 21         & 87         & 11         & M              & 74         & 3          & 32         & 63          & 3            \\ \hline
        \textbf{7}   & 92         & 28         & 96         & 0(6)       & 4          & \textbf{0(32)} & M          & 22         & 33         & 29          & 0            \\ \hline
        \textbf{8}   & 77         & 0(25)      & 46         & 23         & 61         & 82             & 15         & M          & 42         & 40          & 15           \\ \hline
        \textbf{9}   & 74         & 26         & 10         & 51         & 0(0)       & 34             & 0(0)       & 50         & M          & 61          & 0            \\ \hline
        \textbf{10}  & 41         & 66         & 7          & 22         & 0(7)       & 44             & 47         & 61         & 83         & M           & 7            \\ \hline
        \textbf{d}_j & 19         & 10         & 5          & 6          & 0          & 32             & 0          & 3          & 30         & 1           & 0            \\ \hline
    \end{tabular}
\end{table}

\[ d(1,7) = 6 + 0 = 6; \quad d(2,7) = 1 + 0 = 1; \quad d(3,5) = 0 + 0 = 0; \quad d(3,9) = 0 + 30 = 30; \]
\[ d(4,8) = 5 + 3 = 8; \quad d(5,3) = 0 + 5 = 5; \quad d(5,10) = 0 + 1 = 1; \quad d(6,1) = 3 + 19 = 22; \]
\[ d(7,4) = 0 + 6 = 6; \quad d(7,6) = 0 + 32 = 32; \quad d(8,2) = 15 + 10 = 25; \quad d(9,5) = 0 + 0 = 0; \]
\[ d(9,7) = 0 + 0 = 0; \quad d(10,5) = 7 + 0 = 7; \]

Наибольшая сумма констант приведения равна \( (0 + 32) = 32 \) для ребра \( (7,6) \), следовательно, множество разбивается на два подмножества \( (7,6) \) и \( (7^*,6^*) \).

\textbf{Исключение ребра} \( (7,6) \) проводим путем замены элемента \( d_{76} = 0 \) на \( M \), после чего осуществляем очередное приведение матрицы расстояний для образовавшегося подмножества \( (7^*,6^*) \), в результате получим редуцированную матрицу.

\begin{table}[H]
    \centering
    \begin{tabular}{|c|c|c|c|c|c|c|c|c|c|c|c|}
        \hline
        \textbf{i j} & \textbf{1} & \textbf{2} & \textbf{3} & \textbf{4} & \textbf{5} & \textbf{6} & \textbf{7} & \textbf{8} & \textbf{9} & \textbf{10} & \textbf{d}_i \\ \hline
        \textbf{1}   & M          & 22         & 56         & 6          & 51         & 53         & 0          & 32         & 46         & 72          & 0            \\ \hline
        \textbf{2}   & 38         & M          & 38         & 76         & 44         & 32         & 0          & 20         & 49         & 1           & 0            \\ \hline
        \textbf{3}   & 21         & 40         & M          & 78         & 0          & 45         & 57         & 38         & 0          & 21          & 0            \\ \hline
        \textbf{4}   & 19         & 10         & 5          & M          & 74         & 65         & 16         & 0          & 80         & 10          & 0            \\ \hline
        \textbf{5}   & 66         & 12         & 0          & 18         & M          & 35         & 6          & 12         & 30         & 0           & 0            \\ \hline
        \textbf{6}   & 0          & 75         & 21         & 87         & 11         & M          & 74         & 3          & 32         & 63          & 0            \\ \hline
        \textbf{7}   & 92         & 28         & 96         & 0          & 4          & M          & M          & 22         & 33         & 29          & 0            \\ \hline
        \textbf{8}   & 77         & 0          & 46         & 23         & 61         & 82         & 15         & M          & 42         & 40          & 0            \\ \hline
        \textbf{9}   & 74         & 26         & 10         & 51         & 0          & 34         & 0          & 50         & M          & 61          & 0            \\ \hline
        \textbf{10}  & 41         & 66         & 7          & 22         & 0          & 44         & 47         & 61         & 83         & M           & 0            \\ \hline
        \textbf{d}_j & 0          & 0          & 0          & 0          & 0          & 32         & 0          & 0          & 0          & 0           & 32           \\ \hline
    \end{tabular}
\end{table}

Нижняя граница гамильтоновых циклов этого подмножества:
\[ H(7^*,6^*) = 120 + 32 = 152 \]

\textbf{Включение ребра} \( (7,6) \) проводится путем исключения всех элементов 7-ой строки и 6-го столбца, в которой элемент \( d_{67} \) заменяем на \( M \), для исключения образования негамильтонова цикла.
В результате получим другую сокращенную матрицу (9 x 9), которая подлежит операции приведения.
После операции приведения сокращенная матрица будет иметь вид:

\begin{table}[H]
    \centering
    \begin{tabular}{|c|c|c|c|c|c|c|c|c|c|c|}
        \hline
        \textbf{i j} & \textbf{1} & \textbf{2} & \textbf{3} & \textbf{4} & \textbf{5} & \textbf{7} & \textbf{8} & \textbf{9} & \textbf{10} & \textbf{d}_i \\ \hline
        \textbf{1}   & M          & 22         & 56         & 6          & 51         & 0          & 32         & 46         & 72          & 0            \\ \hline
        \textbf{2}   & 38         & M          & 38         & 76         & 44         & 0          & 20         & 49         & 1           & 0            \\ \hline
        \textbf{3}   & 21         & 40         & M          & 78         & 0          & 57         & 38         & 0          & 21          & 0            \\ \hline
        \textbf{4}   & 19         & 10         & 5          & M          & 74         & 16         & 0          & 80         & 10          & 0            \\ \hline
        \textbf{5}   & 66         & 12         & 0          & 18         & M          & 6          & 12         & 30         & 0           & 0            \\ \hline
        \textbf{6}   & 0          & 75         & 21         & 87         & 11         & M          & 3          & 32         & 63          & 0            \\ \hline
        \textbf{8}   & 77         & 0          & 46         & 23         & 61         & 15         & M          & 42         & 40          & 0            \\ \hline
        \textbf{9}   & 74         & 26         & 10         & 51         & 0          & 0          & 50         & M          & 61          & 0            \\ \hline
        \textbf{10}  & 41         & 66         & 7          & 22         & 0          & 47         & 61         & 83         & M           & 0            \\ \hline
        \textbf{d}_j & 0          & 0          & 0          & 6          & 0          & 0          & 0          & 0          & 0           & 6            \\ \hline
    \end{tabular}
\end{table}

Сумма констант приведения сокращенной матрицы:
\[ \sum d_i + \sum d_j = 6 \]

Нижняя граница подмножества \( (7,6) \) равна:
\[ H(7,6) = 120 + 6 = 126 \leq 152 \]

Поскольку нижняя граница этого подмножества \( (7,6) \) меньше, чем подмножества \( (7^*,6^*) \), то ребро \( (7,6) \) включаем в маршрут с новой границей \( H = 126 \).
\end{document}