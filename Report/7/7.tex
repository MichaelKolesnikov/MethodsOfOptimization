\documentclass[17pt]{extarticle}
\usepackage{../mystyle}

\begin{document}
\section*{Работа №7}
\subsection*{Метод ветвей и границ решения задачи коммвояжера}
Дано n городов, $C = \left| c_{i,j} \right|, \quad i, j = \overline{1, n}$ - матрица стоимостей переездов из $i$-х городов в $j$-е. Коммивояжер должен выехать из своего города, заехать в каждый город только один раз и вернуться в исходный город. Нужно найти замкнутый маршрут объезда всех городов минимальной стоимости.
\[
    \begin{aligned}
         & x_{i,j} = \begin{cases} 1, \text{если коммивояжер едет из i в j} \\ 0, \text{иначе} \\ \end{cases}
        \qquad F=\sum_{i=1}^n \sum_{j=1}^n c_{i.j} \cdot x_{i,j} \rightarrow min                              \\
         & \sum_{i=1}^n x_{i,j} = 1 \ \forall j=\overline{1,n}, \qquad
        \sum_{j=1}^n x_{i,j} = 1 \ \forall i=\overline{1,n}                                                   \\
    \end{aligned}
\]

Из каждого i-го города только один выезд, в каждый j-й город, только один въезд.
\[u_i \in \overline{1,n} \text{- каким по счету мы посетим город i.}\]
\[u_i-u_j + n \cdot x_{i,j} \leq n - 1 \text{ - обеспечивает замкнутость маршрута и отсутствие петель}\]

\subsubsection{Решение программой}
\begin{verbatim}
import random  
  
class TSPSolver:  
    def __init__(self, matrix):  
        self.n = len(matrix)  
        self.graph = matrix  
        self.best_cost = float('inf')  
        self.best_path = []  
  
    def solve(self):  
        visited = [False] * self.n  
        visited[0] = True  
        self._branch_and_bound(0, visited, 1, 0, [0])  
        return self.best_cost, self.best_path  
  
    def _branch_and_bound(self, curr_node, visited, level, curr_cost, path):  
        if level == self.n:  
            return_cost = self.graph[curr_node][0]  
            if return_cost > 0:  
                total_cost = curr_cost + return_cost  
                if total_cost < self.best_cost:  
                    self.best_cost = total_cost  
                    self.best_path = path + [0]  
            return  
  
        for i in range(self.n):  
            if not visited[i] and self.graph[curr_node][i] > 0:  
                next_cost = curr_cost + self.graph[curr_node][i]  
                if next_cost < self.best_cost:  
                    visited[i] = True  
                    self._branch_and_bound(i, visited, level + 1, next_cost, path + [i])  
                    visited[i] = False  
  
def input_matrix(n):  
    print(f"Введите матрицу размера {n} x {n} (через пробелы, элементы разделяются строками):")  
    matrix = []  
    for i in range(n):  
        row = list(map(int, input().split()))  
        matrix.append(row)  
    return matrix  
  
def generate_random_matrix(n, max_weight=100):  
    matrix = [[random.randint(1, max_weight) if i != j else 0 for j in range(n)] for i in range(n)]  
    return matrix  
  
def print_matrix(matrix):  
    for row in matrix:  
        print(" ".join(map(str, row)))  
  
def main():  
    n = int(input("Введите размер матрицы (n): "))  
    print("Выберите опцию:")  
    print("1. Ввести свою матрицу")  
    print("2. Сгенерировать случайную матрицу")  
    option = int(input("Ваш выбор (1 или 2): "))  
  
    if option == 1:  
        matrix = input_matrix(n)  
    elif option == 2:  
        matrix = generate_random_matrix(n)  
        print("Сгенерированная матрица:")  
        print_matrix(matrix)  
    else:  
        print("Некорректный выбор.")  
        return  
  
    solver = TSPSolver(matrix)  
    cost, path = solver.solve()  
    print("\nМинимальная стоимость:", cost)  
    print("Оптимальный путь:", " -> ".join(map(str, path)))  
  
if __name__ == "__main__":  
    main()
\end{verbatim}
Работа программы:
\begin{verbatim}
Введите размер матрицы (n): 10
Выберите опцию:
1. Ввести свою матрицу
2. Сгенерировать случайную матрицу
Ваш выбор (1 или 2): 2
Сгенерированная матрица:
0 34 68 18 63 80 12 44 58 87
56 0 56 94 62 65 18 38 67 22
34 53 0 91 13 73 70 51 13 37
28 19 14 0 83 89 25 9 89 22
67 13 1 19 0 51 7 13 31 4
3 78 24 90 14 0 77 6 35 69
96 32 100 4 8 19 0 26 37 36
79 2 48 25 63 99 17 0 44 45
97 49 33 74 23 72 23 73 0 87
58 83 24 39 17 76 64 78 100 0

Минимальная стоимость: 127
Оптимальный путь: 0 -> 3 -> 7 -> 1 -> 9 -> 4 -> 2 -> 8 -> 6 -> 5 -> 0
\end{verbatim}

\subsection*{Решение руками}
\textbf{Задача коммивояжера.}
Возьмем в качестве произвольного маршрута:
\[ X_0 = (1,2);(2,3);(3,4);(4,5);(5,6);(6,7);(7,8);(8,9);(9,10);(10,1) \]
Тогда \( F(X_0) = 34 + 56 + 91 + 83 + 51 + 77 + 26 + 44 + 87 + 58 = 607 \)
Для определения нижней границы множества воспользуемся \textbf{операцией редукции} или приведения матрицы по строкам, для чего необходимо в каждой строке матрицы \( D \) найти минимальный элемент.
\[ d_i = \min_j d_{ij} \]
Затем вычитаем \( d_i \) из элементов рассматриваемой строки. В связи с этим во вновь полученной матрице в каждой строке будет как минимум один ноль.
Такую же операцию редукции проводим по столбцам, для чего в каждом столбце находим минимальный элемент:
\[ d_j = \min_i d_{ij} \]

После вычитания минимальных элементов получаем полностью редуцированную матрицу, где величины \( d_i \) и \( d_j \) называются \textbf{константами приведения}.

\[
    \begin{array}{|c|c|c|c|c|c|c|c|c|c|c|}
        \hline
        \textbf{i j} & \textbf{1} & \textbf{2} & \textbf{3} & \textbf{4} & \textbf{5} & \textbf{6} & \textbf{7} & \textbf{8} & \textbf{9} & \textbf{10} \\
        \hline
        \textbf{1}   & M          & 22         & 56         & 6          & 51         & 53         & 0          & 32         & 46         & 72          \\
        \hline
        \textbf{2}   & 38         & M          & 38         & 76         & 44         & 32         & 0          & 20         & 49         & 1           \\
        \hline
        \textbf{3}   & 21         & 40         & M          & 78         & 0          & 45         & 57         & 38         & 0          & 21          \\
        \hline
        \textbf{4}   & 19         & 10         & 5          & M          & 74         & 65         & 16         & 0          & 80         & 10          \\
        \hline
        \textbf{5}   & 66         & 12         & 0          & 18         & M          & 35         & 6          & 12         & 30         & 0           \\
        \hline
        \textbf{6}   & 0          & 75         & 21         & 87         & 11         & M          & 74         & 3          & 32         & 63          \\
        \hline
        \textbf{7}   & 92         & 28         & 96         & 0          & 4          & 0          & M          & 22         & 33         & 29          \\
        \hline
        \textbf{8}   & 77         & 0          & 46         & 23         & 61         & 82         & 15         & M          & 42         & 40          \\
        \hline
        \textbf{9}   & 74         & 26         & 10         & 51         & 0          & 34         & 0          & 50         & M          & 61          \\
        \hline
        \textbf{10}  & 41         & 66         & 7          & 22         & 0          & 44         & 47         & 61         & 83         & M           \\
        \hline
    \end{array}
\]

Сумма констант приведения определяет нижнюю границу \( H \):
\[ H = \sum d_i + \sum d_j \]
\[ H = 12 + 18 + 13 + 9 + 1 + 3 + 4 + 2 + 23 + 17 + 0 + 0 + 0 + 0 + 0 + 15 + 0 + 0 + 0 + 3 = 120 \]

\textbf{Шаг №1.}
\textbf{Определяем ребро ветвления} и разобьем все множество маршрутов относительно этого ребра на два подмножества \((i,j)\) и \((i^*,j^*)\).
С этой целью для всех клеток матрицы с нулевыми элементами заменяем поочередно нули на \( M \) (бесконечность) и определяем для них сумму образовавшихся констант приведения, они приведены в скобках.

\[
    \begin{aligned}
        d(1,7)  & = 6 + 0 = 6;    \\
        d(2,7)  & = 1 + 0 = 1;    \\
        d(3,5)  & = 0 + 0 = 0;    \\
        d(3,9)  & = 0 + 30 = 30;  \\
        d(4,8)  & = 5 + 3 = 8;    \\
        d(5,3)  & = 0 + 5 = 5;    \\
        d(5,10) & = 0 + 1 = 1;    \\
        d(6,1)  & = 3 + 19 = 22;  \\
        d(7,4)  & = 0 + 6 = 6;    \\
        d(7,6)  & = 0 + 32 = 32;  \\
        d(8,2)  & = 15 + 10 = 25; \\
        d(9,5)  & = 0 + 0 = 0;    \\
        d(9,7)  & = 0 + 0 = 0;    \\
        d(10,5) & = 7 + 0 = 7;
    \end{aligned}
\]

Наибольшая сумма констант приведения равна \( (0 + 32) = 32 \) для ребра \((7,6)\), следовательно, множество разбивается на два подмножества \((7,6)\) и \((7^*,6^*)\).
\textbf{Исключение ребра} \((7,6)\) проводим путем замены элемента \( d_{76} = 0 \) на \( M \), после чего осуществляем очередное приведение матрицы расстояний для образовавшегося подмножества \((7^*,6^*)\), в результате получим редуцированную матрицу.

\[
    \begin{array}{|c|c|c|c|c|c|c|c|c|c|c|c|}
        \hline
        \textbf{i j} & \textbf{1} & \textbf{2} & \textbf{3} & \textbf{4} & \textbf{5} & \textbf{6} & \textbf{7} & \textbf{8} & \textbf{9} & \textbf{10} & \textbf{di} \\
        \hline
        \textbf{1}   & M          & 22         & 56         & 6          & 51         & 53         & 0          & 32         & 46         & 72          & 0           \\
        \hline
        \textbf{2}   & 38         & M          & 38         & 76         & 44         & 32         & 0          & 20         & 49         & 1           & 0           \\
        \hline
        \textbf{3}   & 21         & 40         & M          & 78         & 0          & 45         & 57         & 38         & 0          & 21          & 0           \\
        \hline
        \textbf{4}   & 19         & 10         & 5          & M          & 74         & 65         & 16         & 0          & 80         & 10          & 0           \\
        \hline
        \textbf{5}   & 66         & 12         & 0          & 18         & M          & 35         & 6          & 12         & 30         & 0           & 0           \\
        \hline
        \textbf{6}   & 0          & 75         & 21         & 87         & 11         & M          & 74         & 3          & 32         & 63          & 0           \\
        \hline
        \textbf{7}   & 92         & 28         & 96         & 0          & 4          & M          & M          & 22         & 33         & 29          & 0           \\
        \hline
        \textbf{8}   & 77         & 0          & 46         & 23         & 61         & 82         & 15         & M          & 42         & 40          & 0           \\
        \hline
        \textbf{9}   & 74         & 26         & 10         & 51         & 0          & 34         & 0          & 50         & M          & 61          & 0           \\
        \hline
        \textbf{10}  & 41         & 66         & 7          & 22         & 0          & 44         & 47         & 61         & 83         & M           & 0           \\
        \hline
        \textbf{dj}  & 0          & 0          & 0          & 0          & 0          & 32         & 0          & 0          & 0          & 0           & 32          \\
        \hline
    \end{array}
\]

Нижняя граница гамильтоновых циклов этого подмножества:
\[ H(7^*,6^*) = 120 + 32 = 152 \]
\textbf{Включение ребра} \((7,6)\) проводится путем исключения всех элементов 7-ой строки и 6-го столбца, в которой элемент \( d_{67} \) заменяем на \( M \), для исключения образования негамильтонова цикла.
В результате получим другую сокращенную матрицу (9 x 9), которая подлежит операции приведения.
После операции приведения сокращенная матрица будет иметь вид:

\[
    \begin{array}{|c|c|c|c|c|c|c|c|c|c|c|}
        \hline
        \textbf{i j} & \textbf{1} & \textbf{2} & \textbf{3} & \textbf{4} & \textbf{5} & \textbf{7} & \textbf{8} & \textbf{9} & \textbf{10} & \textbf{di} \\
        \hline
        \textbf{1}   & M          & 22         & 56         & 6          & 51         & 0          & 32         & 46         & 72          & 0           \\
        \hline
        \textbf{2}   & 38         & M          & 38         & 76         & 44         & 0          & 20         & 49         & 1           & 0           \\
        \hline
        \textbf{3}   & 21         & 40         & M          & 78         & 0          & 57         & 38         & 0          & 21          & 0           \\
        \hline
        \textbf{4}   & 19         & 10         & 5          & M          & 74         & 16         & 0          & 80         & 10          & 0           \\
        \hline
        \textbf{5}   & 66         & 12         & 0          & 18         & M          & 6          & 12         & 30         & 0           & 0           \\
        \hline
        \textbf{6}   & 0          & 75         & 21         & 87         & 11         & M          & 3          & 32         & 63          & 0           \\
        \hline
        \textbf{8}   & 77         & 0          & 46         & 23         & 61         & 15         & M          & 42         & 40          & 0           \\
        \hline
        \textbf{9}   & 74         & 26         & 10         & 51         & 0          & 0          & 50         & M          & 61          & 0           \\
        \hline
        \textbf{10}  & 41         & 66         & 7          & 22         & 0          & 47         & 61         & 83         & M           & 0           \\
        \hline
        \textbf{dj}  & 0          & 0          & 0          & 6          & 0          & 0          & 0          & 0          & 0           & 6           \\
        \hline
    \end{array}
\]

Сумма констант приведения сокращенной матрицы:
\[ \sum d_i + \sum d_j = 6 \]
Нижняя граница подмножества \((7,6)\) равна:
\[ H(7,6) = 120 + 6 = 126 \leq 152 \]
Поскольку нижняя граница этого подмножества \((7,6)\) меньше, чем подмножества \((7^*,6^*)\), то ребро \((7,6)\) включаем в маршрут с новой границей \( H = 126 \).

\textbf{Шаг №2.}
\textbf{Определяем ребро ветвления}.

\[
    \begin{aligned}
        d(1,4)  & = 0 + 12 = 12;  \\
        d(1,7)  & = 0 + 0 = 0;    \\
        d(2,7)  & = 1 + 0 = 1;    \\
        d(3,5)  & = 0 + 0 = 0;    \\
        d(3,9)  & = 0 + 30 = 30;  \\
        d(4,8)  & = 5 + 3 = 8;    \\
        d(5,3)  & = 0 + 5 = 5;    \\
        d(5,10) & = 0 + 1 = 1;    \\
        d(6,1)  & = 3 + 19 = 22;  \\
        d(8,2)  & = 15 + 10 = 25; \\
        d(9,5)  & = 0 + 0 = 0;    \\
        d(9,7)  & = 0 + 0 = 0;    \\
        d(10,5) & = 7 + 0 = 7;
    \end{aligned}
\]
\[ \text{max: } d(3,9) = 30. \]

\textbf{Исключение ребра} \((3,9)\): \( d_{39} = M \).

\[
    \begin{array}{|c|c|c|c|c|c|c|c|c|c|c|}
        \hline
        \textbf{i j} & \textbf{1} & \textbf{2} & \textbf{3} & \textbf{4} & \textbf{5} & \textbf{7} & \textbf{8} & \textbf{9} & \textbf{10} & \textbf{di} \\
        \hline
        \textbf{1}   & M          & 22         & 56         & 0          & 51         & 0          & 32         & 46         & 72          & 0           \\
        \hline
        \textbf{2}   & 38         & M          & 38         & 70         & 44         & 0          & 20         & 49         & 1           & 0           \\
        \hline
        \textbf{3}   & 21         & 40         & M          & 72         & 0          & 57         & 38         & M          & 21          & 0           \\
        \hline
        \textbf{4}   & 19         & 10         & 5          & M          & 74         & 16         & 0          & 80         & 10          & 0           \\
        \hline
        \textbf{5}   & 66         & 12         & 0          & 12         & M          & 6          & 12         & 30         & 0           & 0           \\
        \hline
        \textbf{6}   & 0          & 75         & 21         & 81         & 11         & M          & 3          & 32         & 63          & 0           \\
        \hline
        \textbf{8}   & 77         & 0          & 46         & 17         & 61         & 15         & M          & 42         & 40          & 0           \\
        \hline
        \textbf{9}   & 74         & 26         & 10         & 45         & 0          & 0          & 50         & M          & 61          & 0           \\
        \hline
        \textbf{10}  & 41         & 66         & 7          & 16         & 0          & 47         & 61         & 83         & M           & 0           \\
        \hline
        \textbf{dj}  & 0          & 0          & 0          & 0          & 0          & 0          & 0          & 30         & 0           & 30          \\
        \hline
    \end{array}
\]

\[ H(3^*,9^*) = 126 + 30 = 156 \]

\textbf{Включение ребра} \((3,9)\): \( d_{93} = M \).

\[
    \begin{array}{|c|c|c|c|c|c|c|c|c|c|}
        \hline
        \textbf{i j} & \textbf{1} & \textbf{2} & \textbf{3} & \textbf{4} & \textbf{5} & \textbf{7} & \textbf{8} & \textbf{10} & \textbf{di} \\
        \hline
        \textbf{1}   & M          & 22         & 56         & 0          & 51         & 0          & 32         & 72          & 0           \\
        \hline
        \textbf{2}   & 38         & M          & 38         & 70         & 44         & 0          & 20         & 1           & 0           \\
        \hline
        \textbf{4}   & 19         & 10         & 5          & M          & 74         & 16         & 0          & 10          & 0           \\
        \hline
        \textbf{5}   & 66         & 12         & 0          & 12         & M          & 6          & 12         & 0           & 0           \\
        \hline
        \textbf{6}   & 0          & 75         & 21         & 81         & 11         & M          & 3          & 63          & 0           \\
        \hline
        \textbf{8}   & 77         & 0          & 46         & 17         & 61         & 15         & M          & 40          & 0           \\
        \hline
        \textbf{9}   & 74         & 26         & M          & 45         & 0          & 0          & 50         & 61          & 0           \\
        \hline
        \textbf{10}  & 41         & 66         & 7          & 16         & 0          & 47         & 61         & M           & 0           \\
        \hline
        \textbf{dj}  & 0          & 0          & 0          & 0          & 0          & 0          & 0          & 0           & 0           \\
        \hline
    \end{array}
\]

\[ \sum d_i + \sum d_j = 0 \]
\[ H(3,9) = 126 + 0 = 126 \leq 156 \]
Ребро \((3,9)\) включаем в маршрут с новой границей \( H = 126 \).
\textbf{Шаг №3.}
\textbf{Определяем ребро ветвления}.

\[
    \begin{aligned}
        d(1,4)  & = 0 + 12 = 12;  \\
        d(1,7)  & = 0 + 0 = 0;    \\
        d(2,7)  & = 1 + 0 = 1;    \\
        d(4,8)  & = 5 + 3 = 8;    \\
        d(5,3)  & = 0 + 5 = 5;    \\
        d(5,10) & = 0 + 1 = 1;    \\
        d(6,1)  & = 3 + 19 = 22;  \\
        d(8,2)  & = 15 + 10 = 25; \\
        d(9,5)  & = 0 + 0 = 0;    \\
        d(9,7)  & = 0 + 0 = 0;    \\
        d(10,5) & = 7 + 0 = 7;
    \end{aligned}
\]
\[ \text{max: } d(8,2) = 25. \]

\textbf{Исключение ребра} \((8,2)\): \( d_{82} = M \).

\[
    \begin{array}{|c|c|c|c|c|c|c|c|c|c|}
        \hline
        \textbf{i j} & \textbf{1} & \textbf{2} & \textbf{3} & \textbf{4} & \textbf{5} & \textbf{7} & \textbf{8} & \textbf{10} & \textbf{di} \\
        \hline
        \textbf{1}   & M          & 22         & 56         & 0          & 51         & 0          & 32         & 72          & 0           \\
        \hline
        \textbf{2}   & 38         & M          & 38         & 70         & 44         & 0          & 20         & 1           & 0           \\
        \hline
        \textbf{4}   & 19         & 10         & 5          & M          & 74         & 16         & 0          & 10          & 0           \\
        \hline
        \textbf{5}   & 66         & 12         & 0          & 12         & M          & 6          & 12         & 0           & 0           \\
        \hline
        \textbf{6}   & 0          & 75         & 21         & 81         & 11         & M          & 3          & 63          & 0           \\
        \hline
        \textbf{8}   & 77         & M          & 46         & 17         & 61         & 15         & M          & 40          & 15          \\
        \hline
        \textbf{9}   & 74         & 26         & M          & 45         & 0          & 0          & 50         & 61          & 0           \\
        \hline
        \textbf{10}  & 41         & 66         & 7          & 16         & 0          & 47         & 61         & M           & 0           \\
        \hline
        \textbf{dj}  & 0          & 10         & 0          & 0          & 0          & 0          & 0          & 0           & 25          \\
        \hline
    \end{array}
\]

\[ H(8^*,2^*) = 126 + 25 = 151 \]

\textbf{Включение ребра} \((8,2)\): \( d_{28} = M \).

\[
    \begin{array}{|c|c|c|c|c|c|c|c|c|}
        \hline
        \textbf{i j} & \textbf{1} & \textbf{3} & \textbf{4} & \textbf{5} & \textbf{7} & \textbf{8} & \textbf{10} & \textbf{di} \\
        \hline
        \textbf{1}   & M          & 56         & 0          & 51         & 0          & 32         & 72          & 0           \\
        \hline
        \textbf{2}   & 38         & 38         & 70         & 44         & 0          & M          & 1           & 0           \\
        \hline
        \textbf{4}   & 19         & 5          & M          & 74         & 16         & 0          & 10          & 0           \\
        \hline
        \textbf{5}   & 66         & 0          & 12         & M          & 6          & 12         & 0           & 0           \\
        \hline
        \textbf{6}   & 0          & 21         & 81         & 11         & M          & 3          & 63          & 0           \\
        \hline
        \textbf{9}   & 74         & M          & 45         & 0          & 0          & 50         & 61          & 0           \\
        \hline
        \textbf{10}  & 41         & 7          & 16         & 0          & 47         & 61         & M           & 0           \\
        \hline
        \textbf{dj}  & 0          & 0          & 0          & 0          & 0          & 0          & 0           & 0           \\
        \hline
    \end{array}
\]

\[ \sum d_i + \sum d_j = 0 \]
\[ H(8,2) = 126 + 0 = 126 \leq 151 \]
Ребро \((8,2)\) включаем в маршрут с новой границей \( H = 126 \).

\textbf{Шаг №4.}
\textbf{Определяем ребро ветвления}.

\[
    \begin{aligned}
        d(1,4)  & = 0 + 12 = 12; \\
        d(1,7)  & = 0 + 0 = 0;   \\
        d(2,7)  & = 1 + 0 = 1;   \\
        d(4,8)  & = 5 + 3 = 8;   \\
        d(5,3)  & = 0 + 5 = 5;   \\
        d(5,10) & = 0 + 1 = 1;   \\
        d(6,1)  & = 3 + 19 = 22; \\
        d(9,5)  & = 0 + 0 = 0;   \\
        d(9,7)  & = 0 + 0 = 0;   \\
        d(10,5) & = 7 + 0 = 7;
    \end{aligned}
\]
\[ \text{max: } d(6,1) = 22. \]

\textbf{Исключение ребра} \((6,1)\): \( d_{61} = M \).

\[
    \begin{array}{|c|c|c|c|c|c|c|c|c|}
        \hline
        \textbf{i j} & \textbf{1} & \textbf{3} & \textbf{4} & \textbf{5} & \textbf{7} & \textbf{8} & \textbf{10} & \textbf{di} \\
        \hline
        \textbf{1}   & M          & 56         & 0          & 51         & 0          & 32         & 72          & 0           \\
        \hline
        \textbf{2}   & 38         & 38         & 70         & 44         & 0          & M          & 1           & 0           \\
        \hline
        \textbf{4}   & 19         & 5          & M          & 74         & 16         & 0          & 10          & 0           \\
        \hline
        \textbf{5}   & 66         & 0          & 12         & M          & 6          & 12         & 0           & 0           \\
        \hline
        \textbf{6}   & M          & 21         & 81         & 11         & M          & 3          & 63          & 3           \\
        \hline
        \textbf{9}   & 74         & M          & 45         & 0          & 0          & 50         & 61          & 0           \\
        \hline
        \textbf{10}  & 41         & 7          & 16         & 0          & 47         & 61         & M           & 0           \\
        \hline
        \textbf{dj}  & 19         & 0          & 0          & 0          & 0          & 0          & 0           & 22          \\
        \hline
    \end{array}
\]

\[ H(6^*,1^*) = 126 + 22 = 148 \]

\textbf{Включение ребра} \((6,1)\): \( d_{16} = M \).

\[
    \begin{array}{|c|c|c|c|c|c|c|c|}
        \hline
        \textbf{i j} & \textbf{3} & \textbf{4} & \textbf{5} & \textbf{7} & \textbf{8} & \textbf{10} & \textbf{di} \\
        \hline
        \textbf{1}   & 56         & 0          & 51         & 0          & 32         & 72          & 0           \\
        \hline
        \textbf{2}   & 38         & 70         & 44         & 0          & M          & 1           & 0           \\
        \hline
        \textbf{4}   & 5          & M          & 74         & 16         & 0          & 10          & 0           \\
        \hline
        \textbf{5}   & 0          & 12         & M          & 6          & 12         & 0           & 0           \\
        \hline
        \textbf{9}   & M          & 45         & 0          & 0          & 50         & 61          & 0           \\
        \hline
        \textbf{10}  & 7          & 16         & 0          & 47         & 61         & M           & 0           \\
        \hline
        \textbf{dj}  & 0          & 0          & 0          & 0          & 0          & 0           & 0           \\
        \hline
    \end{array}
\]

\[ \sum d_i + \sum d_j = 0 \]
\[ H(6,1) = 126 + 0 = 126 \leq 148 \]
Запрещаем переходы: \((1,7)\),
Ребро \((6,1)\) включаем в маршрут с новой границей \( H = 126 \).

\textbf{Шаг №5.}
\textbf{Определяем ребро ветвления}.

\[
    \begin{aligned}
        d(1,4)  & = 32 + 12 = 44; \\
        d(2,7)  & = 1 + 0 = 1;    \\
        d(4,8)  & = 5 + 12 = 17;  \\
        d(5,3)  & = 0 + 5 = 5;    \\
        d(5,10) & = 0 + 1 = 1;    \\
        d(9,5)  & = 0 + 0 = 0;    \\
        d(9,7)  & = 0 + 0 = 0;    \\
        d(10,5) & = 7 + 0 = 7;
    \end{aligned}
\]
\[ \text{max: } d(1,4) = 44. \]

\textbf{Исключение ребра} \((1,4)\): \( d_{14} = M \).

\[
    \begin{array}{|c|c|c|c|c|c|c|c|}
        \hline
        \textbf{i j} & \textbf{3} & \textbf{4} & \textbf{5} & \textbf{7} & \textbf{8} & \textbf{10} & \textbf{di} \\
        \hline
        \textbf{1}   & 56         & M          & 51         & M          & 32         & 72          & 32          \\
        \hline
        \textbf{2}   & 38         & 70         & 44         & 0          & M          & 1           & 0           \\
        \hline
        \textbf{4}   & 5          & M          & 74         & 16         & 0          & 10          & 0           \\
        \hline
        \textbf{5}   & 0          & 12         & M          & 6          & 12         & 0           & 0           \\
        \hline
        \textbf{9}   & M          & 45         & 0          & 0          & 50         & 61          & 0           \\
        \hline
        \textbf{10}  & 7          & 16         & 0          & 47         & 61         & M           & 0           \\
        \hline
        \textbf{dj}  & 0          & 12         & 0          & 0          & 0          & 0           & 44          \\
        \hline
    \end{array}
\]

\[ H(1^*,4^*) = 126 + 44 = 170 \]

\textbf{Включение ребра} \((1,4)\): \( d_{41} = M \).

\[
    \begin{array}{|c|c|c|c|c|c|c|}
        \hline
        \textbf{i j} & \textbf{3} & \textbf{5} & \textbf{7} & \textbf{8} & \textbf{10} & \textbf{di} \\
        \hline
        \textbf{2}   & 38         & 44         & 0          & M          & 1           & 0           \\
        \hline
        \textbf{4}   & 5          & 74         & 16         & 0          & 10          & 0           \\
        \hline
        \textbf{5}   & 0          & M          & 6          & 12         & 0           & 0           \\
        \hline
        \textbf{9}   & M          & 0          & 0          & 50         & 61          & 0           \\
        \hline
        \textbf{10}  & 7          & 0          & 47         & 61         & M           & 0           \\
        \hline
        \textbf{dj}  & 0          & 0          & 0          & 0          & 0           & 0           \\
        \hline
    \end{array}
\]

\[ \sum d_i + \sum d_j = 0 \]
\[ H(1,4) = 126 + 0 = 126 \leq 170 \]
Запрещаем переходы: \((4,7)\), \((4,6)\),
Ребро \((1,4)\) включаем в маршрут с новой границей \( H = 126 \).

\textbf{Шаг №6.}
\textbf{Определяем ребро ветвления}.

\[
    \begin{aligned}
        d(2,7)  & = 1 + 0 = 1;   \\
        d(4,8)  & = 5 + 12 = 17; \\
        d(5,3)  & = 0 + 5 = 5;   \\
        d(5,10) & = 0 + 1 = 1;   \\
        d(9,5)  & = 0 + 0 = 0;   \\
        d(9,7)  & = 0 + 0 = 0;   \\
        d(10,5) & = 7 + 0 = 7;
    \end{aligned}
\]
\[ \text{max: } d(4,8) = 17. \]

\textbf{Исключение ребра} \((4,8)\): \( d_{48} = M \).

\[
    \begin{array}{|c|c|c|c|c|c|c|}
        \hline
        \textbf{i j} & \textbf{3} & \textbf{5} & \textbf{7} & \textbf{8} & \textbf{10} & \textbf{di} \\
        \hline
        \textbf{2}   & 38         & 44         & 0          & M          & 1           & 0           \\
        \hline
        \textbf{4}   & 5          & 74         & M          & M          & 10          & 5           \\
        \hline
        \textbf{5}   & 0          & M          & 6          & 12         & 0           & 0           \\
        \hline
        \textbf{9}   & M          & 0          & 0          & 50         & 61          & 0           \\
        \hline
        \textbf{10}  & 7          & 0          & 47         & 61         & M           & 0           \\
        \hline
        \textbf{dj}  & 0          & 0          & 0          & 12         & 0           & 17          \\
        \hline
    \end{array}
\]

\[ H(4^*,8^*) = 126 + 17 = 143 \]

\textbf{Включение ребра} \((4,8)\): \( d_{84} = M \).

\[
    \begin{array}{|c|c|c|c|c|c|}
        \hline
        \textbf{i j} & \textbf{3} & \textbf{5} & \textbf{7} & \textbf{10} & \textbf{di} \\
        \hline
        \textbf{2}   & 38         & 44         & 0          & 1           & 0           \\
        \hline
        \textbf{5}   & 0          & M          & 6          & 0           & 0           \\
        \hline
        \textbf{9}   & M          & 0          & 0          & 61          & 0           \\
        \hline
        \textbf{10}  & 7          & 0          & 47         & M           & 0           \\
        \hline
        \textbf{dj}  & 0          & 0          & 0          & 0           & 0           \\
        \hline
    \end{array}
\]

\[ \sum d_i + \sum d_j = 0 \]
\[ H(4,8) = 126 + 0 = 126 \leq 143 \]
Запрещаем переходы: \((2,7)\), \((2,6)\), \((2,1)\), \((2,4)\),
Ребро \((4,8)\) включаем в маршрут с новой границей \( H = 126 \).

\textbf{Шаг №7.}
\textbf{Определяем ребро ветвления}.

\[
    \begin{array}{|c|c|c|c|c|c|}
        \hline
        \textbf{i j} & \textbf{3}    & \textbf{5} & \textbf{7} & \textbf{10} & \textbf{di} \\
        \hline
        \textbf{2}   & 38            & 44         & M          & 1           & 0           \\
        \hline
        \textbf{5}   & \textbf{0(7)} & M          & 6          & 0(1)        & 0           \\
        \hline
        \textbf{9}   & M             & 0(0)       & 0(6)       & 61          & 0           \\
        \hline
        \textbf{10}  & 7             & 0(7)       & 47         & M           & 7           \\
        \hline
        \textbf{dj}  & 7             & 0          & 6          & 1           & 0           \\
        \hline
    \end{array}
\]

\[
    \begin{aligned}
        d(5,3)  & = 0 + 7 = 7; \\
        d(5,10) & = 0 + 1 = 1; \\
        d(9,5)  & = 0 + 0 = 0; \\
        d(9,7)  & = 0 + 6 = 6; \\
        d(10,5) & = 7 + 0 = 7;
    \end{aligned}
\]
\[ \text{max: } d(5,3) = 7. \]

\textbf{Исключение ребра} \((5,3)\): \( d_{53} = M \).

\[
    \begin{array}{|c|c|c|c|c|c|}
        \hline
        \textbf{i j} & \textbf{3} & \textbf{5} & \textbf{7} & \textbf{10} & \textbf{di} \\
        \hline
        \textbf{2}   & 38         & 44         & M          & 1           & 1           \\
        \hline
        \textbf{5}   & M          & M          & 6          & 0           & 0           \\
        \hline
        \textbf{9}   & M          & 0          & 0          & 61          & 0           \\
        \hline
        \textbf{10}  & 7          & 0          & 47         & M           & 0           \\
        \hline
        \textbf{dj}  & 7          & 0          & 0          & 0           & 8           \\
        \hline
    \end{array}
\]

\[ H(5^*,3^*) = 126 + 8 = 134 \]

\textbf{Включение ребра} \((5,3)\): \( d_{35} = M \).

\[
    \begin{array}{|c|c|c|c|c|}
        \hline
        \textbf{i j} & \textbf{5} & \textbf{7} & \textbf{10} & \textbf{di} \\
        \hline
        \textbf{2}   & 44         & M          & 1           & 1           \\
        \hline
        \textbf{9}   & 0          & 0          & 61          & 0           \\
        \hline
        \textbf{10}  & 0          & 47         & M           & 0           \\
        \hline
        \textbf{dj}  & 0          & 0          & 1           & 2           \\
        \hline
    \end{array}
\]

\[ \sum d_i + \sum d_j = 2 \]
\[ H(5,3) = 126 + 2 = 128 \leq 134 \]
Запрещаем переходы: \((2,7)\), \((2,6)\), \((2,1)\), \((2,4)\), \((9,5)\),
Ребро \((5,3)\) включаем в маршрут с новой границей \( H = 128 \).

\textbf{Шаг №8.}
\textbf{Определяем ребро ветвления}.

\[
    \begin{array}{|c|c|c|c|c|}
        \hline
        \textbf{i j} & \textbf{5} & \textbf{7}      & \textbf{10} & \textbf{di} \\
        \hline
        \textbf{2}   & 43         & M               & 0(104)      & 43          \\
        \hline
        \textbf{9}   & M          & \textbf{0(108)} & 61          & 61          \\
        \hline
        \textbf{10}  & 0(90)      & 47              & M           & 47          \\
        \hline
        \textbf{dj}  & 43         & 47              & 61          & 0           \\
        \hline
    \end{array}
\]

\[
    \begin{aligned}
        d(2,10) & = 43 + 61 = 104; \\
        d(9,7)  & = 61 + 47 = 108; \\
        d(10,5) & = 47 + 43 = 90;
    \end{aligned}
\]
\[ \text{max: } d(9,7) = 108. \]

\textbf{Исключение ребра} \((9,7)\): \( d_{97} = M \).

\[
    \begin{array}{|c|c|c|c|c|}
        \hline
        \textbf{i j} & \textbf{5} & \textbf{7} & \textbf{10} & \textbf{di} \\
        \hline
        \textbf{2}   & 43         & M          & 0           & 0           \\
        \hline
        \textbf{9}   & M          & M          & 61          & 61          \\
        \hline
        \textbf{10}  & 0          & 47         & M           & 0           \\
        \hline
        \textbf{dj}  & 0          & 47         & 0           & 108         \\
        \hline
    \end{array}
\]

\[ H(9^*,7^*) = 128 + 108 = 236 \]

\textbf{Включение ребра} \((9,7)\): \( d_{79} = M \).

\[
    \begin{array}{|c|c|c|c|}
        \hline
        \textbf{i j} & \textbf{5} & \textbf{10} & \textbf{di} \\
        \hline
        \textbf{2}   & 43         & 0           & 0           \\
        \hline
        \textbf{10}  & 0          & M           & 0           \\
        \hline
        \textbf{dj}  & 0          & 0           & 0           \\
        \hline
    \end{array}
\]

\[ \sum d_i + \sum d_j = 0 \]
\[ H(9,7) = 128 + 0 = 128 \leq 236 \]
Ребро \((9,7)\) включаем в маршрут с новой границей \( H = 128 \).
В соответствии с этой матрицей включаем в гамильтонов маршрут ребра \((2,10)\) и \((10,5)\).
В результате по дереву ветвлений гамильтонов цикл образуют ребра:
\[ (7,6), (6,1), (1,4), (4,8), (8,2), (2,10), (10,5), (5,3), (3,9), (9,7), \]
Длина маршрута равна \( F(M_k) = 127 \).
\begin{figure}[H]
    \centering
    \includegraphics[width=0.7\textwidth]{8/Pasted image 20241219230336.png}
\end{figure}

\clearpage



\subsection*{Общая идея методов ветвей и границ}
Задача \( \mathop{f(x)}\limits_{x \in X} \rightarrow \min \)
\begin{enumerate}
    \item В зависимости от специфики задачи выбирается некоторый способ вычисления оценок снизу $d(X')$, функции $f(x)$ на множествах $X' \subset X$: (в частности, может быть $X' = X$)
          \[
              f(x) \geq d(X'), \quad x \in X'.
          \]
          Оценка снизу часто вычисляется путем релаксации, т.е. замены задачи минимизации $f(x)$ по множеству $X'$ задачей минимизации по некоторому более широкому множеству.
          (Например, релаксация целочисленной или частично целочисленной задачи может состоять в отбрасывании требования целочисленности.)

    \item Выбирается также правило ветвления,
          состоящее в выборе разветвляемого подмножества $X'$ из числа подмножеств,
          на которые к данному шагу разбито множество $X$,
          и выборе способа разбиения $X'$ на непересекающиеся подмножества.
          Обычно из числа кандидатов на ветвление выбирается множество $X'$ с наименьшей оценкой, поскольку именно в таком множестве естественно искать минимум в первую очередь.
          При этом рассматриваются только такие способы вычисления оценок снизу, в которых оценки для подмножеств, получившихся в результате разветвления $X'$, не меньше $d(X')$.
\end{enumerate}
\subsection*{Метод ветвей и границ решения задачи коммвояжера}
Дано n городов, $C = \left| c_{i,j} \right|, \quad i, j = \overline{1, n}$ - матрица стоимостей переездов из $i$-х городов в $j$-е.
Коммивояжер должен выехать из своего города, заехать в каждый город только один раз и вернуться в исходный город. Нужно найти замкнутый маршрут объезда всех городов минимальной стоимости.
\[
    \begin{aligned}
         & x_{i,j} = \begin{cases} 1, \text{если коммивояжер едет из i в j} \\ 0, \text{иначе} \\ \end{cases}
        \qquad F=\sum_{i=1}^n \sum_{j=1}^n c_{i.j} \cdot x_{i,j} \rightarrow min                              \\
         & \sum_{i=1}^n x_{i,j} = 1 \ \forall j=\overline{1,n}, \qquad
        \sum_{j=1}^n x_{i,j} = 1 \ \forall i=\overline{1,n}                                                   \\
    \end{aligned}
\]

Из каждого i-го города только один выезд, в каждый j-й город, только один въезд.
\[u_i \in \overline{1,n} \text{- каким по счету мы посетим город i.}\]
\[u_i-u_j + n \cdot x_{i,j} \leq n - 1 \text{ - обеспечивает замкнутость маршрута и отсутствие петель}\]

\textbf{Определение:} Циклом \( t \) назовем набор из \( n \) упорядоченных пар городов, образующих маршрут, проходящий через каждый город только один раз:
\[
    t = \left\{(i_1, i_2), (i_2, i_3), \dots, (i_{n-1}, i_n), (i_n, i_1)\right\}.
\]
Издержки цикла:
\[
    z(t) = \sum_{k=1}^{n-1} C_{i_k, i_{k+1}} + C_{i_n, i_1}.
\]
Для каждого допустимого маршрута каждая строка и каждый столбец содержат ровно по одному элементу, соответствующему этому маршруту.

\[
    C = \begin{pmatrix}
        \infty  & \dots  & c_{1,n} \\
        \vdots  & \ddots & \vdots  \\
        c_{n,1} & \dots  & \infty  \\
    \end{pmatrix}
\]

\textbf{Определение:} Матрица, получаемая из данной вычитанием из элементов каждой строки минимального элемента этой строки, а затем вычитанием из элементов каждого столбца минимального элемента этого столбца, называется приведенной матрицей:
\[
    \begin{cases}
        \left|c_{i,j}\right| = \left| c_{i,j} - \mathop{\min{c_{i,j}}} \limits_{i=\overline{1,n}} \right|     \\
        \left|c''_{i,j}\right| = \left| c'_{i,j} - \mathop{\min{c'_{i,j}}} \limits_{j=\overline{1,n}} \right| \\
    \end{cases}
\]

\textbf{Определение:} Приводящая константа:
\[
    h=\sum_{i=1}^{n} \mathop{\min{c_{i,j}}} \limits_{j=\overline{1,n}} + \sum_{j=1}^{n} \mathop{\min{c'_{i,j}}} \limits_{i=\overline{1,n}}
\]
Пусть \( t \) — некоторый маршрут.

\begin{itemize}
    \item \( z(t) \) — его издержки по исходной матрице.
    \item \( z'(t) \) — его издержки по приведенной матрице.
\end{itemize}

\[
    z(t) = z'(t) + h
\]

\( h \) является нижней границей издержек для всех циклов \( t \) исходной матрицы расстояний, поскольку \( h \) — сумма минимальных элементов строк и столбцов.


\begin{enumerate}
    \item[] \textbf{Алгоритм}
    \item \textbf{Приведение матрицы}
    \item \textbf{Ветвление}

          На каждом шаге алгоритма будет строиться одно звено
          оптимального маршрута. Для построения решения в начале
          целесообразно выбрать звено нулевой длины, а затем
          последовательно добавлять звенья нулевой или минимальной длины.

          Процесс ветвления можно представить в виде
          дерева, каждая вершина которого соответствует некоторому
          множеству маршрутов, являющегося подмножеством множества X.

          Пусть \( G_0 \) - множество всех маршрутов. Разобьем \( G_0 \) на два подмножества:

          \begin{itemize}
              \item Множество маршрутов, включающих переезд из \( i \) в \( j \). Обозначение: \(\{(i, j)\}\)
              \item Множество маршрутов, не включающих переезд из \( i \) в \( j \). Обозначение: \(\{\overline{(i, j)}\}\)
          \end{itemize}

          Пару городов \((i, j)\) для ветвления будем выбирать среди тех пар, которым в приведенной матрице соответствуют нулевые элементы, причем выбирается такая пара \((i, j)\),
          чтобы подмножество \(\{\overline{(i, j)}\}\) имело максимальную оценку.

          Разветвляя далее множество с меньшей оценкой, в конце концов будет получено подмножество, содержащее один маршрут.
          Двигаясь по дереву в обратном направлении, получим маршрут.

          Выбор пары городов \((i, j)\) для ветвления:
          \begin{itemize}
              \item \( c''_{i,j} = 0 \).

              \item Оценка множества \(\{\overline{(i, j)}\}\) должна быть максимальной.

                    Рассмотрим маршруты, которые будут включены в \(\{\overline{(i, j)}\}\). Поскольку город \(i\) должен быть связан с некоторым другим городом,
                    то каждый маршрут из \(\{\overline{(i, j)}\}\) должен содержать звено,
                    длина которого не меньше минимального элемента \(i\)-ой строки, не считая \( c''_{i,j} \).

                    Вычисляем сумму этих минимальных элементов для каждого \(c''_{ij} = 0\).
                    Назовем эту сумму оценкой пары \((i,j)\) или штрафом за использование звена \((i,j)\):
                    \[
                        \Theta_{i,j} = \mathop{\min{c''_{i,j'}}} \limits_{\forall j' \neq j} + \mathop{\min{c''_{i',j}}} \limits_{\forall i' \neq i}
                    \]

                    В качетсве пары городов для ветвления выбираем ту пару, для которой оценка будет максимальной.
          \end{itemize}

    \item \textbf{Вычисление нижней границы множества \(\{\overline{(i, j)}\}\)}

          Нижняя граница множества \(\{\overline{(i, j)}\}\) определяется как сумма оценки разветвляемого множества и максимального значения \(\Theta_{i,j}\).
    \item \textbf{Исключение строки и столбца.}

          Так как из каждого города можно выезжать только один раз и в каждый город можно въезжать только один раз, то строку \(i\) и столбец \(j\) из дальнейшего рассмотрения исключаем.
          Чтобы не получить замкнутых неполных циклов, нужно наложить необходимые запреты, в частности, на переезд из \(j\) в \(i\), т.е. положить \(C_{i,j} = \infty\).
    \item \textbf{Завершение}

          Если усеченная матрица ещё не имеет размерности \(2 \times 2\),
          то приводим полученную матрицу и находим оценку множества \(\{(i, j)\}\) как сумму оценки разветвляемого множества и полученной приводящей константы, переход к пункту 2.

          Иначе

          определяемые усеченной матрицы пары городов завершают маршрут.
          Приводя эту матрицу и добавляя приводящую константу к оценке последнего разветвляемого множества, получим оценку маршрута.

          \textbf{Критерий оптимальности:}

          Если оценка не больше оценок всех тупиковых ветвей, то маршрут, описанный деревом ветвей, является оптимальным.
          Иначе процесс ветвления должен быть продолжен, исходя из множества с меньшей оценкой.



\end{enumerate}
\end{document}