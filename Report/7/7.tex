\documentclass[17pt]{extarticle}
\usepackage{../mystyle}

\begin{document}
\section*{Работа №7}
\subsection*{Метод ветвей и границ решения задачи коммвояжера}

\subsubsection{Решение программой}

Работа программы:
\begin{verbatim}
Введите размер матрицы (n): 10
Выберите опцию:
1. Ввести свою матрицу
2. Сгенерировать случайную матрицу
Ваш выбор (1 или 2): 2
Сгенерированная матрица:
0 34 68 18 63 80 12 44 58 87
56 0 56 94 62 65 18 38 67 22
34 53 0 91 13 73 70 51 13 37
28 19 14 0 83 89 25 9 89 22
67 13 1 19 0 51 7 13 31 4
3 78 24 90 14 0 77 6 35 69
96 32 100 4 8 19 0 26 37 36
79 2 48 25 63 99 17 0 44 45
97 49 33 74 23 72 23 73 0 87
58 83 24 39 17 76 64 78 100 0

Минимальная стоимость: 127
Оптимальный путь: 0 -> 3 -> 7 -> 1 -> 9 -> 4 -> 2 -> 8 -> 6 -> 5 -> 0
\end{verbatim}

\subsection*{Решение руками}
\textbf{Задача коммивояжера.}
Возьмем в качестве произвольного маршрута:
\[ X_0 = (1,2);(2,3);(3,4);(4,5);(5,6);(6,7);(7,8);(8,9);(9,10);(10,1) \]
Тогда \( F(X_0) = 34 + 56 + 91 + 83 + 51 + 77 + 26 + 44 + 87 + 58 = 607 \)
Для определения нижней границы множества воспользуемся \textbf{операцией редукции} или приведения матрицы по строкам, для чего необходимо в каждой строке матрицы \( D \) найти минимальный элемент.
\[ d_i = \min_j d_{ij} \]
Затем вычитаем \( d_i \) из элементов рассматриваемой строки. В связи с этим во вновь полученной матрице в каждой строке будет как минимум один ноль.
Такую же операцию редукции проводим по столбцам, для чего в каждом столбце находим минимальный элемент:
\[ d_j = \min_i d_{ij} \]

После вычитания минимальных элементов получаем полностью редуцированную матрицу, где величины \( d_i \) и \( d_j \) называются \textbf{константами приведения}.

\[
    \begin{array}{|c|c|c|c|c|c|c|c|c|c|c|}
        \hline
        \textbf{i j} & \textbf{1} & \textbf{2} & \textbf{3} & \textbf{4} & \textbf{5} & \textbf{6} & \textbf{7} & \textbf{8} & \textbf{9} & \textbf{10} \\
        \hline
        \textbf{1}   & M          & 22         & 56         & 6          & 51         & 53         & 0          & 32         & 46         & 72          \\
        \hline
        \textbf{2}   & 38         & M          & 38         & 76         & 44         & 32         & 0          & 20         & 49         & 1           \\
        \hline
        \textbf{3}   & 21         & 40         & M          & 78         & 0          & 45         & 57         & 38         & 0          & 21          \\
        \hline
        \textbf{4}   & 19         & 10         & 5          & M          & 74         & 65         & 16         & 0          & 80         & 10          \\
        \hline
        \textbf{5}   & 66         & 12         & 0          & 18         & M          & 35         & 6          & 12         & 30         & 0           \\
        \hline
        \textbf{6}   & 0          & 75         & 21         & 87         & 11         & M          & 74         & 3          & 32         & 63          \\
        \hline
        \textbf{7}   & 92         & 28         & 96         & 0          & 4          & 0          & M          & 22         & 33         & 29          \\
        \hline
        \textbf{8}   & 77         & 0          & 46         & 23         & 61         & 82         & 15         & M          & 42         & 40          \\
        \hline
        \textbf{9}   & 74         & 26         & 10         & 51         & 0          & 34         & 0          & 50         & M          & 61          \\
        \hline
        \textbf{10}  & 41         & 66         & 7          & 22         & 0          & 44         & 47         & 61         & 83         & M           \\
        \hline
    \end{array}
\]

Сумма констант приведения определяет нижнюю границу \( H \):
\[ H = \sum d_i + \sum d_j \]
\[ H = 12 + 18 + 13 + 9 + 1 + 3 + 4 + 2 + 23 + 17 + 0 + 0 + 0 + 0 + 0 + 15 + 0 + 0 + 0 + 3 = 120 \]

\textbf{Шаг №1.}
\textbf{Определяем ребро ветвления} и разобьем все множество маршрутов относительно этого ребра на два подмножества \((i,j)\) и \((i^*,j^*)\).
С этой целью для всех клеток матрицы с нулевыми элементами заменяем поочередно нули на \( M \) (бесконечность) и определяем для них сумму образовавшихся констант приведения, они приведены в скобках.

\[
    \begin{aligned}
        d(1,7)  & = 6 + 0 = 6;    \\
        d(2,7)  & = 1 + 0 = 1;    \\
        d(3,5)  & = 0 + 0 = 0;    \\
        d(3,9)  & = 0 + 30 = 30;  \\
        d(4,8)  & = 5 + 3 = 8;    \\
        d(5,3)  & = 0 + 5 = 5;    \\
        d(5,10) & = 0 + 1 = 1;    \\
        d(6,1)  & = 3 + 19 = 22;  \\
        d(7,4)  & = 0 + 6 = 6;    \\
        d(7,6)  & = 0 + 32 = 32;  \\
        d(8,2)  & = 15 + 10 = 25; \\
        d(9,5)  & = 0 + 0 = 0;    \\
        d(9,7)  & = 0 + 0 = 0;    \\
        d(10,5) & = 7 + 0 = 7;
    \end{aligned}
\]

Наибольшая сумма констант приведения равна \( (0 + 32) = 32 \) для ребра \((7,6)\), следовательно, множество разбивается на два подмножества \((7,6)\) и \((7^*,6^*)\).
\textbf{Исключение ребра} \((7,6)\) проводим путем замены элемента \( d_{76} = 0 \) на \( M \), после чего осуществляем очередное приведение матрицы расстояний для образовавшегося подмножества \((7^*,6^*)\), в результате получим редуцированную матрицу.

\[
    \begin{array}{|c|c|c|c|c|c|c|c|c|c|c|c|}
        \hline
        \textbf{i j} & \textbf{1} & \textbf{2} & \textbf{3} & \textbf{4} & \textbf{5} & \textbf{6} & \textbf{7} & \textbf{8} & \textbf{9} & \textbf{10} & \textbf{di} \\
        \hline
        \textbf{1}   & M          & 22         & 56         & 6          & 51         & 53         & 0          & 32         & 46         & 72          & 0           \\
        \hline
        \textbf{2}   & 38         & M          & 38         & 76         & 44         & 32         & 0          & 20         & 49         & 1           & 0           \\
        \hline
        \textbf{3}   & 21         & 40         & M          & 78         & 0          & 45         & 57         & 38         & 0          & 21          & 0           \\
        \hline
        \textbf{4}   & 19         & 10         & 5          & M          & 74         & 65         & 16         & 0          & 80         & 10          & 0           \\
        \hline
        \textbf{5}   & 66         & 12         & 0          & 18         & M          & 35         & 6          & 12         & 30         & 0           & 0           \\
        \hline
        \textbf{6}   & 0          & 75         & 21         & 87         & 11         & M          & 74         & 3          & 32         & 63          & 0           \\
        \hline
        \textbf{7}   & 92         & 28         & 96         & 0          & 4          & M          & M          & 22         & 33         & 29          & 0           \\
        \hline
        \textbf{8}   & 77         & 0          & 46         & 23         & 61         & 82         & 15         & M          & 42         & 40          & 0           \\
        \hline
        \textbf{9}   & 74         & 26         & 10         & 51         & 0          & 34         & 0          & 50         & M          & 61          & 0           \\
        \hline
        \textbf{10}  & 41         & 66         & 7          & 22         & 0          & 44         & 47         & 61         & 83         & M           & 0           \\
        \hline
        \textbf{dj}  & 0          & 0          & 0          & 0          & 0          & 32         & 0          & 0          & 0          & 0           & 32          \\
        \hline
    \end{array}
\]

Нижняя граница гамильтоновых циклов этого подмножества:
\[ H(7^*,6^*) = 120 + 32 = 152 \]
\textbf{Включение ребра} \((7,6)\) проводится путем исключения всех элементов 7-ой строки и 6-го столбца, в которой элемент \( d_{67} \) заменяем на \( M \), для исключения образования негамильтонова цикла.
В результате получим другую сокращенную матрицу (9 x 9), которая подлежит операции приведения.
После операции приведения сокращенная матрица будет иметь вид:

\[
    \begin{array}{|c|c|c|c|c|c|c|c|c|c|c|}
        \hline
        \textbf{i j} & \textbf{1} & \textbf{2} & \textbf{3} & \textbf{4} & \textbf{5} & \textbf{7} & \textbf{8} & \textbf{9} & \textbf{10} & \textbf{di} \\
        \hline
        \textbf{1}   & M          & 22         & 56         & 6          & 51         & 0          & 32         & 46         & 72          & 0           \\
        \hline
        \textbf{2}   & 38         & M          & 38         & 76         & 44         & 0          & 20         & 49         & 1           & 0           \\
        \hline
        \textbf{3}   & 21         & 40         & M          & 78         & 0          & 57         & 38         & 0          & 21          & 0           \\
        \hline
        \textbf{4}   & 19         & 10         & 5          & M          & 74         & 16         & 0          & 80         & 10          & 0           \\
        \hline
        \textbf{5}   & 66         & 12         & 0          & 18         & M          & 6          & 12         & 30         & 0           & 0           \\
        \hline
        \textbf{6}   & 0          & 75         & 21         & 87         & 11         & M          & 3          & 32         & 63          & 0           \\
        \hline
        \textbf{8}   & 77         & 0          & 46         & 23         & 61         & 15         & M          & 42         & 40          & 0           \\
        \hline
        \textbf{9}   & 74         & 26         & 10         & 51         & 0          & 0          & 50         & M          & 61          & 0           \\
        \hline
        \textbf{10}  & 41         & 66         & 7          & 22         & 0          & 47         & 61         & 83         & M           & 0           \\
        \hline
        \textbf{dj}  & 0          & 0          & 0          & 6          & 0          & 0          & 0          & 0          & 0           & 6           \\
        \hline
    \end{array}
\]

Сумма констант приведения сокращенной матрицы:
\[ \sum d_i + \sum d_j = 6 \]
Нижняя граница подмножества \((7,6)\) равна:
\[ H(7,6) = 120 + 6 = 126 \leq 152 \]
Поскольку нижняя граница этого подмножества \((7,6)\) меньше, чем подмножества \((7^*,6^*)\), то ребро \((7,6)\) включаем в маршрут с новой границей \( H = 126 \).

\textbf{Шаг №2.}
\textbf{Определяем ребро ветвления}.

\[
    \begin{aligned}
        d(1,4)  & = 0 + 12 = 12;  \\
        d(1,7)  & = 0 + 0 = 0;    \\
        d(2,7)  & = 1 + 0 = 1;    \\
        d(3,5)  & = 0 + 0 = 0;    \\
        d(3,9)  & = 0 + 30 = 30;  \\
        d(4,8)  & = 5 + 3 = 8;    \\
        d(5,3)  & = 0 + 5 = 5;    \\
        d(5,10) & = 0 + 1 = 1;    \\
        d(6,1)  & = 3 + 19 = 22;  \\
        d(8,2)  & = 15 + 10 = 25; \\
        d(9,5)  & = 0 + 0 = 0;    \\
        d(9,7)  & = 0 + 0 = 0;    \\
        d(10,5) & = 7 + 0 = 7;
    \end{aligned}
\]
\[ \text{max: } d(3,9) = 30. \]

\textbf{Исключение ребра} \((3,9)\): \( d_{39} = M \).

\[
    \begin{array}{|c|c|c|c|c|c|c|c|c|c|c|}
        \hline
        \textbf{i j} & \textbf{1} & \textbf{2} & \textbf{3} & \textbf{4} & \textbf{5} & \textbf{7} & \textbf{8} & \textbf{9} & \textbf{10} & \textbf{di} \\
        \hline
        \textbf{1}   & M          & 22         & 56         & 0          & 51         & 0          & 32         & 46         & 72          & 0           \\
        \hline
        \textbf{2}   & 38         & M          & 38         & 70         & 44         & 0          & 20         & 49         & 1           & 0           \\
        \hline
        \textbf{3}   & 21         & 40         & M          & 72         & 0          & 57         & 38         & M          & 21          & 0           \\
        \hline
        \textbf{4}   & 19         & 10         & 5          & M          & 74         & 16         & 0          & 80         & 10          & 0           \\
        \hline
        \textbf{5}   & 66         & 12         & 0          & 12         & M          & 6          & 12         & 30         & 0           & 0           \\
        \hline
        \textbf{6}   & 0          & 75         & 21         & 81         & 11         & M          & 3          & 32         & 63          & 0           \\
        \hline
        \textbf{8}   & 77         & 0          & 46         & 17         & 61         & 15         & M          & 42         & 40          & 0           \\
        \hline
        \textbf{9}   & 74         & 26         & 10         & 45         & 0          & 0          & 50         & M          & 61          & 0           \\
        \hline
        \textbf{10}  & 41         & 66         & 7          & 16         & 0          & 47         & 61         & 83         & M           & 0           \\
        \hline
        \textbf{dj}  & 0          & 0          & 0          & 0          & 0          & 0          & 0          & 30         & 0           & 30          \\
        \hline
    \end{array}
\]

\[ H(3^*,9^*) = 126 + 30 = 156 \]

\textbf{Включение ребра} \((3,9)\): \( d_{93} = M \).

\[
    \begin{array}{|c|c|c|c|c|c|c|c|c|c|}
        \hline
        \textbf{i j} & \textbf{1} & \textbf{2} & \textbf{3} & \textbf{4} & \textbf{5} & \textbf{7} & \textbf{8} & \textbf{10} & \textbf{di} \\
        \hline
        \textbf{1}   & M          & 22         & 56         & 0          & 51         & 0          & 32         & 72          & 0           \\
        \hline
        \textbf{2}   & 38         & M          & 38         & 70         & 44         & 0          & 20         & 1           & 0           \\
        \hline
        \textbf{4}   & 19         & 10         & 5          & M          & 74         & 16         & 0          & 10          & 0           \\
        \hline
        \textbf{5}   & 66         & 12         & 0          & 12         & M          & 6          & 12         & 0           & 0           \\
        \hline
        \textbf{6}   & 0          & 75         & 21         & 81         & 11         & M          & 3          & 63          & 0           \\
        \hline
        \textbf{8}   & 77         & 0          & 46         & 17         & 61         & 15         & M          & 40          & 0           \\
        \hline
        \textbf{9}   & 74         & 26         & M          & 45         & 0          & 0          & 50         & 61          & 0           \\
        \hline
        \textbf{10}  & 41         & 66         & 7          & 16         & 0          & 47         & 61         & M           & 0           \\
        \hline
        \textbf{dj}  & 0          & 0          & 0          & 0          & 0          & 0          & 0          & 0           & 0           \\
        \hline
    \end{array}
\]

\[ \sum d_i + \sum d_j = 0 \]
\[ H(3,9) = 126 + 0 = 126 \leq 156 \]
Ребро \((3,9)\) включаем в маршрут с новой границей \( H = 126 \).
\textbf{Шаг №3.}
\textbf{Определяем ребро ветвления}.

\[
    \begin{aligned}
        d(1,4)  & = 0 + 12 = 12;  \\
        d(1,7)  & = 0 + 0 = 0;    \\
        d(2,7)  & = 1 + 0 = 1;    \\
        d(4,8)  & = 5 + 3 = 8;    \\
        d(5,3)  & = 0 + 5 = 5;    \\
        d(5,10) & = 0 + 1 = 1;    \\
        d(6,1)  & = 3 + 19 = 22;  \\
        d(8,2)  & = 15 + 10 = 25; \\
        d(9,5)  & = 0 + 0 = 0;    \\
        d(9,7)  & = 0 + 0 = 0;    \\
        d(10,5) & = 7 + 0 = 7;
    \end{aligned}
\]
\[ \text{max: } d(8,2) = 25. \]

\textbf{Исключение ребра} \((8,2)\): \( d_{82} = M \).

\[
    \begin{array}{|c|c|c|c|c|c|c|c|c|c|}
        \hline
        \textbf{i j} & \textbf{1} & \textbf{2} & \textbf{3} & \textbf{4} & \textbf{5} & \textbf{7} & \textbf{8} & \textbf{10} & \textbf{di} \\
        \hline
        \textbf{1}   & M          & 22         & 56         & 0          & 51         & 0          & 32         & 72          & 0           \\
        \hline
        \textbf{2}   & 38         & M          & 38         & 70         & 44         & 0          & 20         & 1           & 0           \\
        \hline
        \textbf{4}   & 19         & 10         & 5          & M          & 74         & 16         & 0          & 10          & 0           \\
        \hline
        \textbf{5}   & 66         & 12         & 0          & 12         & M          & 6          & 12         & 0           & 0           \\
        \hline
        \textbf{6}   & 0          & 75         & 21         & 81         & 11         & M          & 3          & 63          & 0           \\
        \hline
        \textbf{8}   & 77         & M          & 46         & 17         & 61         & 15         & M          & 40          & 15          \\
        \hline
        \textbf{9}   & 74         & 26         & M          & 45         & 0          & 0          & 50         & 61          & 0           \\
        \hline
        \textbf{10}  & 41         & 66         & 7          & 16         & 0          & 47         & 61         & M           & 0           \\
        \hline
        \textbf{dj}  & 0          & 10         & 0          & 0          & 0          & 0          & 0          & 0           & 25          \\
        \hline
    \end{array}
\]

\[ H(8^*,2^*) = 126 + 25 = 151 \]

\textbf{Включение ребра} \((8,2)\): \( d_{28} = M \).

\[
    \begin{array}{|c|c|c|c|c|c|c|c|c|}
        \hline
        \textbf{i j} & \textbf{1} & \textbf{3} & \textbf{4} & \textbf{5} & \textbf{7} & \textbf{8} & \textbf{10} & \textbf{di} \\
        \hline
        \textbf{1}   & M          & 56         & 0          & 51         & 0          & 32         & 72          & 0           \\
        \hline
        \textbf{2}   & 38         & 38         & 70         & 44         & 0          & M          & 1           & 0           \\
        \hline
        \textbf{4}   & 19         & 5          & M          & 74         & 16         & 0          & 10          & 0           \\
        \hline
        \textbf{5}   & 66         & 0          & 12         & M          & 6          & 12         & 0           & 0           \\
        \hline
        \textbf{6}   & 0          & 21         & 81         & 11         & M          & 3          & 63          & 0           \\
        \hline
        \textbf{9}   & 74         & M          & 45         & 0          & 0          & 50         & 61          & 0           \\
        \hline
        \textbf{10}  & 41         & 7          & 16         & 0          & 47         & 61         & M           & 0           \\
        \hline
        \textbf{dj}  & 0          & 0          & 0          & 0          & 0          & 0          & 0           & 0           \\
        \hline
    \end{array}
\]

\[ \sum d_i + \sum d_j = 0 \]
\[ H(8,2) = 126 + 0 = 126 \leq 151 \]
Ребро \((8,2)\) включаем в маршрут с новой границей \( H = 126 \).

\textbf{Шаг №4.}
\textbf{Определяем ребро ветвления}.

\[
    \begin{aligned}
        d(1,4)  & = 0 + 12 = 12; \\
        d(1,7)  & = 0 + 0 = 0;   \\
        d(2,7)  & = 1 + 0 = 1;   \\
        d(4,8)  & = 5 + 3 = 8;   \\
        d(5,3)  & = 0 + 5 = 5;   \\
        d(5,10) & = 0 + 1 = 1;   \\
        d(6,1)  & = 3 + 19 = 22; \\
        d(9,5)  & = 0 + 0 = 0;   \\
        d(9,7)  & = 0 + 0 = 0;   \\
        d(10,5) & = 7 + 0 = 7;
    \end{aligned}
\]
\[ \text{max: } d(6,1) = 22. \]

\textbf{Исключение ребра} \((6,1)\): \( d_{61} = M \).

\[
    \begin{array}{|c|c|c|c|c|c|c|c|c|}
        \hline
        \textbf{i j} & \textbf{1} & \textbf{3} & \textbf{4} & \textbf{5} & \textbf{7} & \textbf{8} & \textbf{10} & \textbf{di} \\
        \hline
        \textbf{1}   & M          & 56         & 0          & 51         & 0          & 32         & 72          & 0           \\
        \hline
        \textbf{2}   & 38         & 38         & 70         & 44         & 0          & M          & 1           & 0           \\
        \hline
        \textbf{4}   & 19         & 5          & M          & 74         & 16         & 0          & 10          & 0           \\
        \hline
        \textbf{5}   & 66         & 0          & 12         & M          & 6          & 12         & 0           & 0           \\
        \hline
        \textbf{6}   & M          & 21         & 81         & 11         & M          & 3          & 63          & 3           \\
        \hline
        \textbf{9}   & 74         & M          & 45         & 0          & 0          & 50         & 61          & 0           \\
        \hline
        \textbf{10}  & 41         & 7          & 16         & 0          & 47         & 61         & M           & 0           \\
        \hline
        \textbf{dj}  & 19         & 0          & 0          & 0          & 0          & 0          & 0           & 22          \\
        \hline
    \end{array}
\]

\[ H(6^*,1^*) = 126 + 22 = 148 \]

\textbf{Включение ребра} \((6,1)\): \( d_{16} = M \).

\[
    \begin{array}{|c|c|c|c|c|c|c|c|}
        \hline
        \textbf{i j} & \textbf{3} & \textbf{4} & \textbf{5} & \textbf{7} & \textbf{8} & \textbf{10} & \textbf{di} \\
        \hline
        \textbf{1}   & 56         & 0          & 51         & 0          & 32         & 72          & 0           \\
        \hline
        \textbf{2}   & 38         & 70         & 44         & 0          & M          & 1           & 0           \\
        \hline
        \textbf{4}   & 5          & M          & 74         & 16         & 0          & 10          & 0           \\
        \hline
        \textbf{5}   & 0          & 12         & M          & 6          & 12         & 0           & 0           \\
        \hline
        \textbf{9}   & M          & 45         & 0          & 0          & 50         & 61          & 0           \\
        \hline
        \textbf{10}  & 7          & 16         & 0          & 47         & 61         & M           & 0           \\
        \hline
        \textbf{dj}  & 0          & 0          & 0          & 0          & 0          & 0           & 0           \\
        \hline
    \end{array}
\]

\[ \sum d_i + \sum d_j = 0 \]
\[ H(6,1) = 126 + 0 = 126 \leq 148 \]
Запрещаем переходы: \((1,7)\),
Ребро \((6,1)\) включаем в маршрут с новой границей \( H = 126 \).

\textbf{Шаг №5.}
\textbf{Определяем ребро ветвления}.

\[
    \begin{aligned}
        d(1,4)  & = 32 + 12 = 44; \\
        d(2,7)  & = 1 + 0 = 1;    \\
        d(4,8)  & = 5 + 12 = 17;  \\
        d(5,3)  & = 0 + 5 = 5;    \\
        d(5,10) & = 0 + 1 = 1;    \\
        d(9,5)  & = 0 + 0 = 0;    \\
        d(9,7)  & = 0 + 0 = 0;    \\
        d(10,5) & = 7 + 0 = 7;
    \end{aligned}
\]
\[ \text{max: } d(1,4) = 44. \]

\textbf{Исключение ребра} \((1,4)\): \( d_{14} = M \).

\[
    \begin{array}{|c|c|c|c|c|c|c|c|}
        \hline
        \textbf{i j} & \textbf{3} & \textbf{4} & \textbf{5} & \textbf{7} & \textbf{8} & \textbf{10} & \textbf{di} \\
        \hline
        \textbf{1}   & 56         & M          & 51         & M          & 32         & 72          & 32          \\
        \hline
        \textbf{2}   & 38         & 70         & 44         & 0          & M          & 1           & 0           \\
        \hline
        \textbf{4}   & 5          & M          & 74         & 16         & 0          & 10          & 0           \\
        \hline
        \textbf{5}   & 0          & 12         & M          & 6          & 12         & 0           & 0           \\
        \hline
        \textbf{9}   & M          & 45         & 0          & 0          & 50         & 61          & 0           \\
        \hline
        \textbf{10}  & 7          & 16         & 0          & 47         & 61         & M           & 0           \\
        \hline
        \textbf{dj}  & 0          & 12         & 0          & 0          & 0          & 0           & 44          \\
        \hline
    \end{array}
\]

\[ H(1^*,4^*) = 126 + 44 = 170 \]

\textbf{Включение ребра} \((1,4)\): \( d_{41} = M \).

\[
    \begin{array}{|c|c|c|c|c|c|c|}
        \hline
        \textbf{i j} & \textbf{3} & \textbf{5} & \textbf{7} & \textbf{8} & \textbf{10} & \textbf{di} \\
        \hline
        \textbf{2}   & 38         & 44         & 0          & M          & 1           & 0           \\
        \hline
        \textbf{4}   & 5          & 74         & 16         & 0          & 10          & 0           \\
        \hline
        \textbf{5}   & 0          & M          & 6          & 12         & 0           & 0           \\
        \hline
        \textbf{9}   & M          & 0          & 0          & 50         & 61          & 0           \\
        \hline
        \textbf{10}  & 7          & 0          & 47         & 61         & M           & 0           \\
        \hline
        \textbf{dj}  & 0          & 0          & 0          & 0          & 0           & 0           \\
        \hline
    \end{array}
\]

\[ \sum d_i + \sum d_j = 0 \]
\[ H(1,4) = 126 + 0 = 126 \leq 170 \]
Запрещаем переходы: \((4,7)\), \((4,6)\),
Ребро \((1,4)\) включаем в маршрут с новой границей \( H = 126 \).

\textbf{Шаг №6.}
\textbf{Определяем ребро ветвления}.

\[
    \begin{aligned}
        d(2,7)  & = 1 + 0 = 1;   \\
        d(4,8)  & = 5 + 12 = 17; \\
        d(5,3)  & = 0 + 5 = 5;   \\
        d(5,10) & = 0 + 1 = 1;   \\
        d(9,5)  & = 0 + 0 = 0;   \\
        d(9,7)  & = 0 + 0 = 0;   \\
        d(10,5) & = 7 + 0 = 7;
    \end{aligned}
\]
\[ \text{max: } d(4,8) = 17. \]

\textbf{Исключение ребра} \((4,8)\): \( d_{48} = M \).

\[
    \begin{array}{|c|c|c|c|c|c|c|}
        \hline
        \textbf{i j} & \textbf{3} & \textbf{5} & \textbf{7} & \textbf{8} & \textbf{10} & \textbf{di} \\
        \hline
        \textbf{2}   & 38         & 44         & 0          & M          & 1           & 0           \\
        \hline
        \textbf{4}   & 5          & 74         & M          & M          & 10          & 5           \\
        \hline
        \textbf{5}   & 0          & M          & 6          & 12         & 0           & 0           \\
        \hline
        \textbf{9}   & M          & 0          & 0          & 50         & 61          & 0           \\
        \hline
        \textbf{10}  & 7          & 0          & 47         & 61         & M           & 0           \\
        \hline
        \textbf{dj}  & 0          & 0          & 0          & 12         & 0           & 17          \\
        \hline
    \end{array}
\]

\[ H(4^*,8^*) = 126 + 17 = 143 \]

\textbf{Включение ребра} \((4,8)\): \( d_{84} = M \).

\[
    \begin{array}{|c|c|c|c|c|c|}
        \hline
        \textbf{i j} & \textbf{3} & \textbf{5} & \textbf{7} & \textbf{10} & \textbf{di} \\
        \hline
        \textbf{2}   & 38         & 44         & 0          & 1           & 0           \\
        \hline
        \textbf{5}   & 0          & M          & 6          & 0           & 0           \\
        \hline
        \textbf{9}   & M          & 0          & 0          & 61          & 0           \\
        \hline
        \textbf{10}  & 7          & 0          & 47         & M           & 0           \\
        \hline
        \textbf{dj}  & 0          & 0          & 0          & 0           & 0           \\
        \hline
    \end{array}
\]

\[ \sum d_i + \sum d_j = 0 \]
\[ H(4,8) = 126 + 0 = 126 \leq 143 \]
Запрещаем переходы: \((2,7)\), \((2,6)\), \((2,1)\), \((2,4)\),
Ребро \((4,8)\) включаем в маршрут с новой границей \( H = 126 \).

\textbf{Шаг №7.}
\textbf{Определяем ребро ветвления}.

\[
    \begin{array}{|c|c|c|c|c|c|}
        \hline
        \textbf{i j} & \textbf{3}    & \textbf{5} & \textbf{7} & \textbf{10} & \textbf{di} \\
        \hline
        \textbf{2}   & 38            & 44         & M          & 1           & 0           \\
        \hline
        \textbf{5}   & \textbf{0(7)} & M          & 6          & 0(1)        & 0           \\
        \hline
        \textbf{9}   & M             & 0(0)       & 0(6)       & 61          & 0           \\
        \hline
        \textbf{10}  & 7             & 0(7)       & 47         & M           & 7           \\
        \hline
        \textbf{dj}  & 7             & 0          & 6          & 1           & 0           \\
        \hline
    \end{array}
\]

\[
    \begin{aligned}
        d(5,3)  & = 0 + 7 = 7; \\
        d(5,10) & = 0 + 1 = 1; \\
        d(9,5)  & = 0 + 0 = 0; \\
        d(9,7)  & = 0 + 6 = 6; \\
        d(10,5) & = 7 + 0 = 7;
    \end{aligned}
\]
\[ \text{max: } d(5,3) = 7. \]

\textbf{Исключение ребра} \((5,3)\): \( d_{53} = M \).

\[
    \begin{array}{|c|c|c|c|c|c|}
        \hline
        \textbf{i j} & \textbf{3} & \textbf{5} & \textbf{7} & \textbf{10} & \textbf{di} \\
        \hline
        \textbf{2}   & 38         & 44         & M          & 1           & 1           \\
        \hline
        \textbf{5}   & M          & M          & 6          & 0           & 0           \\
        \hline
        \textbf{9}   & M          & 0          & 0          & 61          & 0           \\
        \hline
        \textbf{10}  & 7          & 0          & 47         & M           & 0           \\
        \hline
        \textbf{dj}  & 7          & 0          & 0          & 0           & 8           \\
        \hline
    \end{array}
\]

\[ H(5^*,3^*) = 126 + 8 = 134 \]

\textbf{Включение ребра} \((5,3)\): \( d_{35} = M \).

\[
    \begin{array}{|c|c|c|c|c|}
        \hline
        \textbf{i j} & \textbf{5} & \textbf{7} & \textbf{10} & \textbf{di} \\
        \hline
        \textbf{2}   & 44         & M          & 1           & 1           \\
        \hline
        \textbf{9}   & 0          & 0          & 61          & 0           \\
        \hline
        \textbf{10}  & 0          & 47         & M           & 0           \\
        \hline
        \textbf{dj}  & 0          & 0          & 1           & 2           \\
        \hline
    \end{array}
\]

\[ \sum d_i + \sum d_j = 2 \]
\[ H(5,3) = 126 + 2 = 128 \leq 134 \]
Запрещаем переходы: \((2,7)\), \((2,6)\), \((2,1)\), \((2,4)\), \((9,5)\),
Ребро \((5,3)\) включаем в маршрут с новой границей \( H = 128 \).

\textbf{Шаг №8.}
\textbf{Определяем ребро ветвления}.

\[
    \begin{array}{|c|c|c|c|c|}
        \hline
        \textbf{i j} & \textbf{5} & \textbf{7}      & \textbf{10} & \textbf{di} \\
        \hline
        \textbf{2}   & 43         & M               & 0(104)      & 43          \\
        \hline
        \textbf{9}   & M          & \textbf{0(108)} & 61          & 61          \\
        \hline
        \textbf{10}  & 0(90)      & 47              & M           & 47          \\
        \hline
        \textbf{dj}  & 43         & 47              & 61          & 0           \\
        \hline
    \end{array}
\]

\[
    \begin{aligned}
        d(2,10) & = 43 + 61 = 104; \\
        d(9,7)  & = 61 + 47 = 108; \\
        d(10,5) & = 47 + 43 = 90;
    \end{aligned}
\]
\[ \text{max: } d(9,7) = 108. \]

\textbf{Исключение ребра} \((9,7)\): \( d_{97} = M \).

\[
    \begin{array}{|c|c|c|c|c|}
        \hline
        \textbf{i j} & \textbf{5} & \textbf{7} & \textbf{10} & \textbf{di} \\
        \hline
        \textbf{2}   & 43         & M          & 0           & 0           \\
        \hline
        \textbf{9}   & M          & M          & 61          & 61          \\
        \hline
        \textbf{10}  & 0          & 47         & M           & 0           \\
        \hline
        \textbf{dj}  & 0          & 47         & 0           & 108         \\
        \hline
    \end{array}
\]

\[ H(9^*,7^*) = 128 + 108 = 236 \]

\textbf{Включение ребра} \((9,7)\): \( d_{79} = M \).

\[
    \begin{array}{|c|c|c|c|}
        \hline
        \textbf{i j} & \textbf{5} & \textbf{10} & \textbf{di} \\
        \hline
        \textbf{2}   & 43         & 0           & 0           \\
        \hline
        \textbf{10}  & 0          & M           & 0           \\
        \hline
        \textbf{dj}  & 0          & 0           & 0           \\
        \hline
    \end{array}
\]

\[ \sum d_i + \sum d_j = 0 \]
\[ H(9,7) = 128 + 0 = 128 \leq 236 \]
Ребро \((9,7)\) включаем в маршрут с новой границей \( H = 128 \).
В соответствии с этой матрицей включаем в гамильтонов маршрут ребра \((2,10)\) и \((10,5)\).
В результате по дереву ветвлений гамильтонов цикл образуют ребра:
\[ (7,6), (6,1), (1,4), (4,8), (8,2), (2,10), (10,5), (5,3), (3,9), (9,7), \]
Длина маршрута равна \( F(M_k) = 127 \).
\begin{figure}[H]
    \centering
    \includegraphics[width=0.7\textwidth]{8/Pasted image 20241219230336.png}
\end{figure}




\end{document}