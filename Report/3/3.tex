\documentclass[17pt]{extarticle}
\usepackage{../mystyle}


\begin{document}
\section*{Работа №3}
\subsection*{a) \( F = -8x_1 + 3x_2 \to \min (\max) \)}

\[
    \begin{cases}
        x_1 + 2x_2 \leq 14   \\
        -4x_1 + 3x_2 \leq 12 \\
        x_1 \leq 6           \\
        x_1 \geq 0, \quad x_2 \geq 0
    \end{cases}
\]

\subsubsection*{Решим сначала на максимум:}

\textbf{Решение исходной задачи:}
\[
    x_{\text{max}} = (0, 4)
\]

\subsubsection*{Формулировка двойственной задачи:}

1. \textbf{Целевая функция:}
\[
    F^* = 14y_1 + 12y_2 + 6y_3 \to \min
\]

2. \textbf{Ограничения:}
\[
    \begin{cases}
        y_1 - 4y_2 + y_3 \geq -8 \\
        2y_1 + 3y_2 \geq 3       \\
        y_1 \geq 0, \quad y_2 \geq 0, \quad y_3 \geq 0
    \end{cases}
\]
\subsubsection*{Решение:}
\[
    \begin{cases}
        (x_1 + 2x_2 - 14) y_1^* = 0   \\
        (-4x_1 + 3x_2 - 12) y_2^* = 0 \\
        (x_1 - 6) y_3^* = 0           \\
    \end{cases}
\]
\[
    \begin{cases}
        (0 + 2 \cdot 4 - 14) y_1^* = 0 \Rightarrow y_1^* = 0          \\
        (-4 \cdot 0 + 3 \cdot 4 - 12) y_2^* = 0 \Rightarrow y_2^* > 0 \\
        (0 - 6) y_3^* = 0 \Rightarrow y_3^* = 0                       \\
    \end{cases}
\]

\subsubsection*{Теорема о дополняющей нежесткости:}
Если \( x_1 \) и \( x_2 \) — оптимальное решение прямой задачи, а \( y_1 \), \( y_2 \), \( y_3 \) — оптимальное решение двойственной задачи, то:

1. \( x_1(y_1 - 4y_2 + y_3 + 8) = 0 \)
2. \( x_2(2y_1 + 3y_2 - 3) = 0 \)

Из решения прямой задачи:
\[
    \begin{cases}
        0 \cdot (y_1 - 4y_2 + y_3 + 8) = 0 \\
        4 \cdot (2y_1 + 3y_2 - 3) = 0      \\
    \end{cases}
    \Rightarrow y^* = (0, 1, 0) \text{ — Оптимальное решение двойственной задачи}
\]
\[
    F^* = 14 \cdot 0 + 12 \cdot 1 + 6 \cdot 0 = 12
\]

\subsubsection*{Решим на максимум через теорему 3:}
Берем данные из последней таблицы решения этого номера в работе 2:
\[
    y^* =
    \begin{pmatrix}
        0 & 3 & 0
    \end{pmatrix}
    \cdot
    \begin{pmatrix}
        1 & -0.67 & 0 \\
        0 & 0.33  & 0 \\
        0 & 0     & 1
    \end{pmatrix}
    = (0, 1, 0)
\]
Получили то же самое.

\subsubsection*{Решим теперь на минимум. Решим так же, находя максимум функции \( G = -F \):}
\textbf{Решение исходной задачи:}
\[
    x_{\text{max}} = (6, 0)
\]

\subsubsection*{Прямая задача:}
\[
    G = 8x_1 - 3x_2 \to \max
\]
\[
    \begin{cases}
        x_1 + 2x_2 \leq 14   \\
        -4x_1 + 3x_2 \leq 12 \\
        x_1 \leq 6           \\
        x_1 \geq 0, \quad x_2 \geq 0
    \end{cases}
\]

\subsubsection*{Двойственная задача:}
1. \textbf{Целевая функция:}
\[
    G^* = 14y_1 + 12y_2 + 6y_3 \to \max
\]
2. \textbf{Ограничения:}
\[
    \begin{cases}
        y_1 - 4y_2 + y_3 \geq 8 \\
        2y_1 + 3y_2 \geq -3     \\
        y_1 \geq 0, \quad y_2 \geq 0, \quad y_3 \geq 0
    \end{cases}
\]

\subsubsection*{Решение:}
\[
    \begin{cases}
        (x_1 + 2x_2 - 14) y_1^* = 0   \\
        (-4x_1 + 3x_2 - 12) y_2^* = 0 \\
        (x_1 - 6) y_3^* = 0           \\
    \end{cases}
\]
\[
    \begin{cases}
        (6 + 2 \cdot 0 - 14) y_1^* = 0 \Rightarrow y_1^* = 0          \\
        (-4 \cdot 6 + 3 \cdot 0 - 12) y_2^* = 0 \Rightarrow y_2^* = 0 \\
        (6 - 6) y_3^* = 0 \Rightarrow y_3^* > 0                       \\
    \end{cases}
\]

\subsubsection*{Теорема о дополняющей нежесткости:}
Если \( x_1 \) и \( x_2 \) — оптимальное решение прямой задачи, а \( y_1 \), \( y_2 \), \( y_3 \) — оптимальное решение двойственной задачи, то:

1. \( x_1(y_1^* - 4y_2^* + y_3^* - 8) = 0 \)
2. \( x_2(2y_1^* + 3y_2^* + 3) = 0 \)

Из решения прямой задачи \( x_1 = 6 \) и \( x_2 = 0 \):
\[
    \begin{cases}
        6 \cdot (y_1^* - 4y_2^* + y_3^* - 8) = 0 \\
        0 \cdot (2y_1^* + 3y_2^* + 3) = 0        \\
    \end{cases}
    \Rightarrow y^* = (0, 0, 8) \text{ — Оптимальное решение двойственной задачи}
\]
\[
    G^* = 14 \cdot 0 + 12 \cdot 0 + 6 \cdot 8 = 48 \Rightarrow F^* = -G^* = -48
\]

\subsection*{b) \( F = 2x_1 + x_2 \to \min (\max) \)}

\[
    \begin{cases}
        2x_1 + x_2 \geq 10 \\
        -4x_1 + x_2 \leq 8 \\
        x_1 \geq 0, \quad x_2 \geq 0
    \end{cases}
\]

\(\nexists F_{\text{max}}\)

\subsubsection*{Решим теперь на минимум. Решим так же, находя максимум функции \( G = -F \):}

\textbf{Решение исходной задачи:}
\[
    x_{\text{max}} = (5, 0)
\]

\subsubsection*{Прямая задача:}
\[
    G = -2x_1 - x_2 \to \max
\]
\[
    \begin{cases}
        -2x_1 - x_2 \leq -10 \\
        -4x_1 + x_2 \leq 8   \\
        x_1 \geq 0, \quad x_2 \geq 0
    \end{cases}
\]

\subsubsection*{Двойственная задача:}
1. \textbf{Целевая функция:}
\[
    G^* = -10y_1 + 8y_2 \to \max
\]
2. \textbf{Ограничения:}
\[
    \begin{cases}
        -2y_1 - 4y_2 \geq -2 \\
        -y_1 + y_2 \geq -1   \\
        y_1 \geq 0, \quad y_2 \geq 0
    \end{cases}
\]

\subsubsection*{Решение:}
\[
    \begin{cases}
        (-2x_1 - x_2 + 10)y_1^* = 0 \\
        (-4x_1 + x_2 - 8)y_2^* = 0  \\
    \end{cases}
\]
\[
    \begin{cases}
        (-2 \cdot 5 - 0 + 10)y_1^* = 0 \Rightarrow y_1^* > 0 \\
        (-4 \cdot 5 + 0 - 8)y_2^* = 0 \Rightarrow y_2^* = 0  \\
    \end{cases}
\]

\subsubsection*{Теорема о дополняющей нежесткости:}
1. \( x_1(-2y_1 - 4y_2 + 2) = 0 \)
2. \( x_2(-y_1 + y_2 + 1) = 0 \)

Из решения прямой задачи:
\[
    \begin{cases}
        5 \cdot (-2y_1 - 4y_2 + 2) = 0 \\
        0 \cdot (-y_1 + y_2 + 1) = 0   \\
    \end{cases}
    \Rightarrow y^* = (1, 0) \text{ — Оптимальное решение двойственной задачи}
\]
\[
    G^* = -10 \cdot 1 + 8 \cdot 0 = -10 \Rightarrow F^* = -G^* = 10
\]

\subsection*{c) \( F = 2x_1 - x_2 \to \min (\max) \)}

\[
    \begin{cases}
        3x_1 - x_2 \geq 21 \\
        x_1 \leq 2         \\
        x_1 \geq 0, \quad x_2 \geq 0
    \end{cases}
\]

\(\nexists F_{\text{min}}\), \(\nexists F_{\text{max}}\)

\end{document}