\documentclass[17pt]{extarticle}
\usepackage{../mystyle}

\begin{document}
\section{Геометрическая интерпретация двумерной задачи линейного программирования}

Рассмотрим двумерную задачу: \(F = c_1 x_1 + c_2 x_2 \rightarrow \max\)

\[
    \begin{cases}
        a_{11}x_1 + a_{12}x_2 \leq b_1 \\
        a_{21}x_1 + a_{22}x_2 \leq b_2 \\
        \vdots                         \\
        a_{m1}x_1 + a_{m2}x_2 \leq b_m
    \end{cases}
\]

Каждое из ограничений \( a_{i1} x_1 + a_{i2} x_2 \leq b_i \) определяет в плоскости с системой координат \( x_1, 0, x_2 \) множество точек,
лежащих по одну сторону от прямой \( a_{i1} x_1 + a_{i2} x_2 = b_i \) (т.е. полуплоскость).

Множество всех точек плоскости, координаты которых удовлетворяют \\ всем ограничениям, т.е. принадлежат сразу всем полуплоскостям,
определяемым отдельными ограничениями, будет представлять собой допустимое множество.

Пусть допустимая область задачи оказалась непустой. \\
Мы хотим найти те точки допустимой области, координаты которых дают
целевой функции наибольшее значение.
Построим линию уровня целевой функции $F = c_1 x_1 + c_2 x_2 = \alpha$.
Перемещая линию уровня в направлении вектора $grad(G)=(c_1, c_2)=\overline{n}$,
нормального к линии уровня, будем получать в пересечении этой линии
с допустимой областью точки, в которых целевая функция принимает
новое значение, большее, чем на предшествующих линиях уровня.
Пересечение допустимой области с линией уровня в том ее положении,
когда дальнейшее перемещение дает пустое множество, и будет
множеством оптимальных точек задачи линейного программирования.

Случаи при решении ЗЛП:
\begin{itemize}
    \item Решение достигается в угловой точке
    \item Целевая функция не ограничена
    \item Бесконечное множество решений
\end{itemize}
\end{document}