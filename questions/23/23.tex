\documentclass[17pt]{extarticle}
\usepackage{../mystyle}

\begin{document}
\section{Метод динамического программирования \\ решения задачи распределения ресурсов. \\ Постановка задачи. \\ Рекуррентные соотношения Беллмана.}
Имеется однородный ресурс в количестве \( S \) единиц, который должен быть распределен между \( N \) предприятиями.
Использование \( i \)-ым предприятием \( x_i \) единиц ресурса дает доход, определяемый значением нелинейной функции \( f_i(x_i) \).
Требуется найти распределение ресурсов между предприятиями, обеспечивающее максимальный доход.
\[
    F = \sum_{i=1}^n f_i(x_i) \rightarrow \max
\]
\[
    \sum_{i=1}^n x_i = S, \quad x_i \geq 0, \quad i = \overline{1, n}
\]
В данной задаче принятие решений является однократным, а многошаговость вводится формально.
Вместо того, чтобы рассматривать допустимые варианты распределения ресурсов между \( n \) предприятиями и оценивать их эффективность,
будем рассматривать следующий многошаговый процесс:

\begin{enumerate}
    \item 1 шаг состоит в оценке эффективности выделения ресурса на 1-ое предприятие (первое направление);
    \item 2 шаг: выделение ресурса на первые два предприятия;
    \item \ldots
    \item \( n \)-ый шаг: оценка эффективности распределения на \( n \) предприятий.
\end{enumerate}

Следовательно, получаем \( n \) этапов, на каждом из которых состояние системы описывается объемом ресурса,
подлежащим распределению между \( k \) предприятиями.
Управлениями будут являться решения об объеме ресурса, выделенного \( k \)-му предприятию.
Задача состоит в выборе таких управлений, при которых целевая функция принимает максимальное значение.

Для применения схемы ДП погружают данную задачу в семейство
задач с любым числом шагов $k \leq n$ и любым запасом ресурса $X_C \leq S$.

Пусть \( W_i(C) \) — максимальный доход при распределении объема \( C \) ресурса между \( i \) предприятиями, \( i = 1, n-1 \).
\[
    W_i(C) = \max \sum_{i=1}^k f_i(x_i), \quad k = 1, n,
\]
где максимум берется по всем неотрицательным \( x_i \), таким что \( x_1 + \ldots + x_k = C \).
Следовательно, применение принципа оптимальности приводит к рекуррентным соотношениям:
\[
    W_i(C) = \max_{0 \leq x_i \leq C} \{ f_i(x_i) + W_{i-1}(C - x_i) \}, \quad i = 2, n-1, \quad \text{при } \forall \, \text{допустимых } C
\]
\[
    W_1(C) = \max_{0 \leq x_1 \leq C} \{ f_1(x_1) \}, \quad \text{при } \forall \, \text{допустимых } C \, (0 \leq C \leq S)
\]
Значение функции \( W_n(C) \) вычисляется лишь для данного значения \( C = S \):
\[
    W_n(S) = \max_{0 \leq x_n \leq S} \{ f_n(x_n) + W_{n-1}(S - x_n) \}
\]

Рекуррентные соотношения позволяют вычислить значения \\ \( W_1(C) \), \( W_2(C) \), ..., \( W_n(C) \)
при всех допустимых \( C \) и найти оптимальные политики. Оптимальный доход для исходной задачи определяется значением \( W_n(S) \).

$\Rightarrow$, зная $W_n(S)$, можно определить $x_n^0$, соответствующее оптимальному решению: \\
\( x_{n-1}^0 \) определяется из \( W_{n-1}(S - x_n^0) \) \\
\( x_{n-2}^0 \) определяется из \( W_{n-2}(S - x_n^0 - x_{n-1}^0) \) \\
\vdots \\
\( x_1^0 \) определяется из \( W_1(S - x_n^0 - x_{n-1}^0 - \ldots - x_2^0) \) \\




\end{document}