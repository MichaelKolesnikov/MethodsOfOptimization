\documentclass[17pt]{extarticle}
\usepackage{../mystyle}

\begin{document}
\section{Приведение задачи линейного программирования к каноническому виду. Формы записи задач ЛП.}
\begin{definition}
    Задачу ЗЛП, представленную в виде
    \(
    f = C_1 X_1 + C_2 X_2 + \cdots + C_n X_n \xrightarrow{?} \max(\min), x = (x_1, \ldots, x_n) \colon x_i \geq 0, i=\overline{1,n} \)
    \[
        \begin{cases}
            a_{11}x_1 + a_{12}x_2 + \cdots + a_{1n}x_n = b_1 \\
            a_{21}x_1 + a_{22}x_2 + \cdots + a_{2n}x_n = b_2 \\
            \vdots                                           \\
            a_{m1}x_1 + a_{m2}x_2 + \cdots + a_{mn}x_n = b_m
        \end{cases}
    \]
    называется \textbf{канонической ЗЛП}.
\end{definition}

\begin{algorithm}
    \caption{Приведение ЗЛП к каноническому виду}
    \begin{algorithmic}[1]
        \If{\( f \to \max \)}
        \State Заменить \( f \) на \( -f \) и минимизировать: \( f \to \min \).
        \EndIf

        \For{каждого ограничения \( i = 1, 2, \dots, m \)}
        \If{\( R_i = \leq \)}
        \State Добавить неотрицательную дополнительную переменную \( s_i \geq 0 \):
        \[
            a_{i1}x_1 + a_{i2}x_2 + \cdots + a_{in}x_n + s_i = b_i.
        \]
        \ElsIf{\( R_i = \geq \)}
        \State Вычесть неотрицательную дополнительную переменную \( s_i \geq 0 \):
        \[
            a_{i1}x_1 + a_{i2}x_2 + \cdots + a_{in}x_n - s_i = b_i.
        \]
        \EndIf
        \EndFor

        \For{каждой переменной \( x_j, j = 1, 2, \dots, n \)}
        \If{\( x_j \) не ограничена по знаку}
        \State Заменить \( x_j \) на \( x_j^+ - x_j^- \), где \( x_j^+ \geq 0 \), \( x_j^- \geq 0 \).
        \EndIf
        \EndFor
    \end{algorithmic}
\end{algorithm}

Виды задач линейного программирования (ЛП)
\begin{enumerate}
    \item Общего вида
          \[
              \begin{cases}
                  \sum\limits_{j=1}^{n} c_j x_j \to \max                                 \\
                  \sum\limits_{j=1}^{n} a_{ij} x_j \le b_i, \quad i = \overline{1,m_1}   \\
                  \sum\limits_{j=1}^{n} a_{ij} x_j = b_i, \quad i = \overline{m_1+1,m_2} \\
                  \sum\limits_{j=1}^{n} a_{ij} x_j \ge b_i, \quad i = \overline{m_2+1,m}
              \end{cases}
          \]
    \item Неотрицательных переменных
          \[
              \begin{cases}
                  \sum\limits_{j=1}^{n} c_j x_j \to \max                               \\
                  \sum\limits_{j=1}^{n} a_{ij} x_j \le b_i, \quad i = \overline{1,m_1} \\
                  \sum\limits_{j=1}^{n} a_{ij} x_j = b_i, \quad i = \overline{m_1+1,m} \\
                  x_j \ge 0, \quad j = \overline{1,n}
              \end{cases}
          \]
    \item Стандартная форма
          \[
              \begin{cases}
                  \sum\limits_{j=1}^{n} c_j x_j \to \max                             \\
                  \sum\limits_{j=1}^{n} a_{ij} x_j \le b_i, \quad i = \overline{1,m} \\
                  x_j \ge 0, \quad j = \overline{1,n}
              \end{cases}
          \]
    \item Каноническая форма
          \[
              \begin{cases}
                  \sum\limits_{j=1}^{n} c_j x_j \to \max                           \\
                  \sum\limits_{j=1}^{n} a_{ij} x_j = b_i, \quad i = \overline{1,m} \\
                  x_j \ge 0, \quad j = \overline{1,n}
              \end{cases}
          \]
    \item Матричная стандартная форма
          \[
              \begin{cases}
                  f=(c, x) \to \max \\
                  A x \le b         \\
                  x \ge 0
              \end{cases}
          \]
\end{enumerate}
\end{document}