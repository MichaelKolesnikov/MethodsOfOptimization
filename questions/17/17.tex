\documentclass[17pt]{extarticle}
\usepackage{../mystyle}

\begin{document}
\section{Задачи целочисленного линейного программирования. Общие подходы к решению.}
Общая постановка задачи целочисленного
программирования отличается от общей
постановки задачи ЛП лишь наличием
дополнительного ограничения: требования
целочисленности, в соответствии с которым
значения всех или части переменных
являются целыми числами.
\[
    f = C_1 X_1 + C_2 X_2 + \cdots + C_n X_n \rightarrow \max
\]
\[
    \begin{cases}
        a_{11}x_1 + a_{12}x_2 + \cdots + a_{1n}x_n = b_1 \\
        a_{21}x_1 + a_{22}x_2 + \cdots + a_{2n}x_n = b_2 \\
        \vdots                                           \\
        a_{m1}x_1 + a_{m2}x_2 + \cdots + a_{mn}x_n = b_m \\
    \end{cases} x_j \in \mathbb{N}_0, j=\overline{1,k}, k \leq n
\]
\begin{itemize}
    \item $k<n$ -- задача частично-целочисленная
    \item $k=n$ -- задача полностью целочисленная
\end{itemize}

\subsection{Методы отсечений}
\begin{enumerate}
    \item Решается задача ЛП, получающаяся из исходной отбрасыванием требования целочисленности \( x = (x_1, \ldots, x_n) \).

          Если найденное решение \( x^1 \) является целочисленным, то оно является решением ЗЦЛП.

          Если найденное решение \( x^1 \) не является целочисленным, то к ограничениям задачи, решаемой на первом этапе, добавляется ограничение вида:
          \(
          \sum_{j=1}^n a_{m+1,j} \cdot x_j \geq b_{m+1}
          \), которое:

          \begin{enumerate}
              \item Отсекает точку \( x^1 \), т.е. \(\sum_{j=1}^n a_{m+1,j} \cdot x_j^1 < b_{m+1}\)
              \item Сохраняет в допустимом множестве все целочисленные точки допустимого множества исходной задачи.
                    Такое ограничение называется правильным отсечением.
          \end{enumerate}

    \item На втором этапе находится решение \( x^2 \) задачи ЛП с дополнительным ограничением.
          Если \( x^2 \) не является целочисленным,
          тогда вводится новое правильное отсечение вида \(\sum_{j=1}^n a_{m+2,j} \cdot x_j \geq b_{m+2}\)
          и т.д., до тех пор, пока решение очередной задачи ЛП не окажется целочисленным.
\end{enumerate}

\subsection{Комбинаторные методы}
В основе комбинаторных методов лежит идея
перебора всех элементов множества допустимых
решений, удовлетворяющих требованию
целочисленности, с целью нахождения
оптимального решения.

Такими методами являются методы ветвей и границ.
Различные методы типа ветвей и границ существенно
используют специфику конкретной задачи и заметно
отличаются друг от друга.

Все они основаны на последовательном разбиении
допустимого множества на подмножества (ветвление) и
вычислении оценок (границ), позволяющих отбрасывать
подмножества, заведомо не содержащие решений задачи

\subsection{Общая идея методов ветвей и границ}
\textbf{Задача:}
\[
    f(x)_{x \in X} \rightarrow \min
\]
\begin{enumerate}
    \item В зависимости от специфики задачи выбирается некоторый способ вычисления оценок
          снизу \( d(X') \) функции \( f(x) \) на множествах \( X' \subset X \):
          (в частности, может быть \( X' = X \))
          \[
              f(x) \geq d(X'), \, x \in X'.
          \]
          Оценка снизу часто вычисляется путем релаксации,
          т.е. замены задачи минимизации \( f(x) \) по множеству \( X' \) задачей минимизации по некоторому более широкому множеству.
          (Например, релаксация целочисленной или частично целочисленной задачи может состоять в отбрасывании требования целочисленности.)

    \item Выбирается также правило ветвления, состоящее в выборе разветвляемого подмножества \( X' \) из числа подмножеств,
          на которые к данному шагу разбито множество \( X \), и выборе способа разбиения \( X' \) на непересекающиеся подмножества.

          Обычно из числа кандидатов на ветвление выбирается множество \( X' \) с наименьшей оценкой,
          поскольку именно в таком множестве естественно искать минимум в первую очередь.

          При этом рассматриваются только такие способы вычисления оценок снизу, в которых оценки для подмножеств,
          получившихся в результате разветвления \( X' \), не меньше \( d(X') \).
\end{enumerate}

\subsection{Метод отсечений Гомори}
1. Полностью целочисленная задача:
Рассмотрим полностью целочисленную задачу, представленную в канонической форме:
\[
    F = \sum_{j=1}^{n} c_j x_j \to \max
\]
\[
    \sum_{j=1}^{n} a_{ij} x_j = b_j, \, i = 1, m
\]
\[
    x_j \geq 0, \, j = 1, n, \, x_j \in \mathbb{N}_0, \, j = 1, n
\]
Будем считать, что \( c_j, \, a_{ij}, \, b_j \in \mathbb{Z} \).
Иначе строим правильное отсечение
Для этого выбираем любое базисное \( x_i^* \), которому соответствует нецелое значение, и выписываем \( i \)-ую строку оптимальной симплекс-таблицы:
\[
    x_i + \sum_{j \in S} a_{ij}^* x_j = b_i^* \quad (1)
\]
где \( S \) – множество индексов свободных переменных.
Полагая, что в (1) все переменные целочисленные, получаем:
\[
    \sum_{j \in S} a_{ij}^* x_j - b_i^* = a \in \mathbb{Z} \quad \text{для любых } d_j \in \mathbb{Z} \text{ существует } d \in \mathbb{Z}:
\]
\[
    \sum_{j \in S} (a_{ij}^* - d_j)x_j - (b_i^* - d_i) = d
\]
При этом, если считать \( d_j = \lfloor a_{ij}^* \rfloor, j \in S, d_i = \lfloor b_i^* \rfloor \), получим:
\[
    \sum_{j \in S} y_j x_j = d + y_i, d \in \mathbb{Z} \quad (2)
\]
где \( y_j = \{a_{ij}^*\}, y_i = \{b_i^*\} \).
\[
    \sum_{j \in S} \{a_{ij}^*\} x_j = d + \{b_i^*\}, \, d \in \mathbb{Z}
\]
Т.к. левая часть равенства (2) является неотрицательной
\((Y_{ij} \geq 0, \, x_j \geq 0)\), то \(d \in \mathbb{N}_0\) и
отсечение Гомори:
\[
    \sum_{j \in S} \{a_{ij}^*\} x_j \geq \{b_i^*\}
\]
2. Частично целочисленная задача
Из (1) следует, что
\[
    x_i + \sum_{j \in S} a_{ij} x_j = \lfloor b_i^* \rfloor + \{b_i^*\} \quad (3)
\]
\[
    x_i - \lfloor b_i^* \rfloor = \{b_i^*\} - \sum_{j \in S} a_{ij}^* x_j
\]
При этом для переменной \(x_i\), удовлетворяющей требованию \(x_i \in \mathbb{N}_0\), выполняется одно из условий:
a) \(x_i \leq \lfloor b_i^* \rfloor\)
б) \(x_i \geq \lfloor b_i^* \rfloor + 1\)
Согласно (3) эти условия могут быть представлены в следующем виде:
\[
    \sum_{j \in S} a_{ij}^* x_j \geq \{b_i^*\} \quad (4)
\]
\[
    \sum_{j \in S} a_{ij}^* x_j \leq \{b_i^*\} - 1 \quad (5)
\]

Пусть \( S^+ \) — множество значений \( j \): \( a_{ij}^* \geq 0 \)

\( S^- \) — множество значений \( j \): \( a_{ij}^* < 0 \quad S = S^+ \cup S^- \)

Из (4) следует, что
\[
    \sum_{j \in S^+} a_{ij}^* x_j \geq \{b_i^*\} \quad (6)
\]

а (5) может быть преобразовано к следующему неравенству:
\[
    \sum_{j \in S^-} a_{ij}^* x_j \leq \{b_i^*\} - 1 \quad \text{или} \quad \frac{\{b_i^*\}}{\{b_i^*\} - 1} \sum_{j \in S^-} a_{ij}^* x_j \geq \{b_i^*\} \quad (7)
\]

Неравенства (4), (5) и следствия из них (6), (7) не могут выполняться одновременно. Но независимо от того, какой случай имеет место, для каждого допустимого решения будет выполняться ограничение:

\[
    \sum_{j \in S^+} a_{ij}^* x_j + \frac{\{b_i^*\}}{\{b_i^*\} - 1} \sum_{j \in S^-} a_{ij}^* x_j \geq \{b_i^*\} \quad (8)
\]

Неравенство (8) определяет новое дополнительное ограничение.
Это ограничение получено без учета требования целочисленности для некоторых переменных модели.

\[
    \sum_{j \in S^+} (a_{ij}^* - \lfloor a_{ij}^* \rfloor) x_j + \frac{\{b_i^*\}}{1 - \{b_i^*\}} \sum_{j \in S^-} (1 - \{a_{ij}^*\}) x_j
    \geq
    \{b_i^*\} - \lfloor a_{ij}^* \rfloor
\]

\[
    \sum_{j \in S^+} a_{ij}x_j + \frac{\{b_i^*\}}{\{b_i^*\} - 1} \sum_{j \in S^-} a_{ij}x_j \geq \{b_i^*\} \quad (8)
\]

(8) можно записать в следующем виде:

\[
    \sum_{j \in S} \gamma_{ij}x_j \geq \{b_i^*\}
\]

для переменных, которые могут быть нецелыми:

\[
    \gamma_{ij} =
    \begin{cases}
        \frac{\{b_i^*\}}{\{b_i^*\} - 1} a_{ij}, & \text{если } a_{ij}^* < 0    \\
        a_{ij},                                 & \text{если } a_{ij}^* \geq 0
    \end{cases}
\]

для переменных, которые могут быть только целыми:

\[
    \gamma_{ij} =
    \begin{cases}
        \{a_{ij}^*\},                                     & \text{если } \{a_{ij}^*\} \leq \{b_i^*\} \\
        \frac{1 - \{a_{ij}^*\}}{1 - \{b_i^*\}} \{b_i^*\}, & \text{если } \{a_{ij}^*\} > \{b_i^*\}
    \end{cases}
\]


\end{document}