\documentclass[17pt]{extarticle}
\usepackage{../mystyle}

\begin{document}
\section{Общая задача нелинейного \\ программирования. Особенности решения.}
\subsection{Общая задача нелинейного программирования (НЛП)}
\[
    f(x) \rightarrow \min \quad (1)
\]
\[
    g_i(x) \leq 0, \quad i = \overline{1, m} \quad (2)
\]
\[
    x \in X \subset \mathbb{R}^n \quad (3)
\]
где \( f(x) \), \( g_i(x) \), \( i = \overline{1, m} \) -- нелинейные функции. \\
Решение \(x^* = \arg \min_{x \in D} f(x)\)

Наиболее существенное отличие задачи
нелинейного программирования от линейных задач
заключается в том, что оптимальное решение может
находиться как на границе допустимого множества,
так и являться его внутренней точкой.

\end{document}