\documentclass[17pt]{extarticle}
\usepackage{../mystyle}

\begin{document}
\section{Анализ устойчивости двойственных оценок}

\subsection{Максимальные значения целевой функции}
Будем рассматривать максимальные значения целевой функции прямой задачи как функцию свободных членов системы ограничений:
\[
    F_{\max} \left( b_1, \ldots, b_m \right).
\]
\begin{proposition}
    В оптимальном плане двойственной задачи значение переменной \( y_i^* \) численно равно частной производной функции
    \[
        F_{\max} \left( b_1, \ldots, b_m \right)
    \]
    по соответствующему аргументу:
    \[
        \frac{\partial F_{\max}}{\partial b_i} = y_i^*, \quad i = \overline{1, m}.
    \]

    Двойственные оценки ресурсов показывают,
    на сколько единиц изменяется доход (\( F \)) от реализации продукции при изменении запаса соответствующего ресурса на одну единицу.

    Большей оценке соответствует наиболее дефицитный ресурс. Для недефицитного ресурса \( y_i^* = 0 \).
\end{proposition}

\subsection{Интервалы устойчивости}
Представляет интерес определить такие интервалы изменения \( b_i \), в которых оптимальный план двойственной задачи не меняется.
\[
    x_b^* = A_B^{-1}(b + \Delta b) = A_B^{-1}b + A_B^{-1}\Delta b = x_b + A_B^{-1}\Delta b
\]
Это имеет место для всех тех значений \( b_i + \Delta b_i \), при которых \( x_b^* \) не содержит отрицательных (т.е. являются допустимым решением).
\[
    A_B^{-1}
    \begin{pmatrix}
        b_1 + \Delta b_1 \\
        \vdots           \\
        b_m + \Delta b_m
    \end{pmatrix}
    \geq 0
\]
Определим интервалы устойчивости для нашей задачи:
\[
    A_B^{-1} =
    \begin{pmatrix}
        0 & -\frac{1}{2} & \frac{1}{4}  \\
        1 & \frac{1}{2}  & -\frac{1}{4} \\
        0 & 1            & 0
    \end{pmatrix}, \quad b =
    \begin{pmatrix}
        180 \\
        210 \\
        800
    \end{pmatrix}
\]

\[
    A_B^{-1}(b + \Delta b) =
    \begin{pmatrix}
        0 & -\frac{1}{2} & \frac{1}{4}  \\
        1 & \frac{1}{2}  & -\frac{1}{4} \\
        0 & 1            & 0
    \end{pmatrix}
    \begin{pmatrix}
        180 + \Delta b_1 \\
        210 + \Delta b_2 \\
        800 + \Delta b_3
    \end{pmatrix}
    =
    \begin{pmatrix}
        95 - \frac{1}{2} \cdot \Delta b_2 + \frac{1}{4} \cdot \Delta b_3              \\
        85 + \Delta b_1 + \frac{1}{2} \cdot \Delta b_2 - \frac{1}{4} \cdot \Delta b_3 \\
        210 + \Delta b_2
    \end{pmatrix}
    \geq 0
\]

Частные случаи \\
1) Если \( \Delta b_2 = 0 \), \( \Delta b_3 = 0 \):
\[
    85 + \Delta b_1 \geq 0 \Rightarrow \Delta b_1 \geq -85
\]

2) Если \( \Delta b_1 = 0 \), \( \Delta b_3 = 0 \):
\[
    \begin{cases}
        95 - \frac{1}{2} \cdot \Delta b_2 \geq 0 \\
        210 + \Delta b_2 \geq 0                  \\
        85 + \frac{1}{2} \cdot \Delta b_2 \geq 0
    \end{cases}
    \quad \begin{cases}
        \Delta b_2 \leq 190  \\
        \Delta b_2 \geq -210 \\
        \Delta b_2 \geq -170
    \end{cases}
    \Rightarrow -170 \leq \Delta b_2 \leq 190
\]

3) Если \( \Delta b_1 = 0 \), \( \Delta b_2 = 0 \):
\[
    \begin{cases}
        95 + \frac{1}{4} \cdot \Delta b_3 \geq 0 \\
        85 - \frac{1}{4} \cdot \Delta b_3 \geq 0
    \end{cases}
    \quad \begin{cases}
        \Delta b_3 \geq -380 \\
        \Delta b_3 \leq 340
    \end{cases}
    \Rightarrow -380 \leq \Delta b_3 \leq 340
\]
$\Rightarrow$ Интервалы изменения ресурсов:
\[
    210 - 170 \leq b_2 \leq 210 + 190 \quad 800 - 380 \leq b_3 \leq 800 + 380
\]
\[
    40 \leq b_2 \leq 400 \quad 420 \leq b_3 \leq 1140
\]
Первый вид ресурса в оптимальном плане недоиспользован, является недефицитным.
Увеличение данного ресурса приведёт лишь к росту его остатка. При этом изменений в оптимальном плане не будет, т.к. \( y_1^* = 0 \).

Предельные значения изменения всякого из ресурсов, для которого двойственные оценки остаются неизменными, определяются следующим образом:
\[
    \Delta b_i^- = \max_{x_{ji} > 0}
    \begin{cases}
        -x_j^* \\
        x_{ji}
    \end{cases}
    \le \Delta b_i \le \min_{x_{ji} < 0}
    \begin{cases}
        -95 \\
        -\frac{1}{2}
    \end{cases}
\]
\[
    -85 \le \Delta b_1
\]
\[
    -170 \le \Delta b_2 \le 190
\]
\[
    \max_{x_{ji} > 0}
    \begin{cases}
        -95 \\
        \frac{1}{4}
    \end{cases}
    \le \Delta b_3 \le
    \begin{cases}
        -85 \\
        -\frac{1}{4}
    \end{cases}
    \quad -380 \le \Delta b_3 \le 340
\]

\subsection{Интервалы устойчивости с точки зрения дохода изделия}
Аналогично можно определить интервалы устойчивости с точки зрения дохода изделия.

\[
    \max_{x_{jk} < 0} \left\{
    \begin{array}{c}
        x_j^* \\
        x_{jk}
    \end{array}
    \right\} \leq x_k \leq \min_{x_{jk} > 0} \left\{
    \begin{array}{c}
        x_j^* \\
        x_{jk}
    \end{array}
    \right\}
\]

Пример для \( x_3 \) и \( x_4 \):

\[
    \max \left\{
    \begin{array}{c}
        \frac{95}{-\frac{3}{2}}
    \end{array}
    \right\} \leq x_3 \leq \min \left\{
    \begin{array}{c}
        \frac{85}{\frac{7}{2}}, \frac{210}{3}
    \end{array}
    \right\} \quad -60 \leq x_3 \leq \left[ \frac{170}{7} \right] = 24
\]

\[
    \max \{Q\} \leq x_4 \leq \min \left\{
    \begin{array}{c}
        \frac{85}{1}, \frac{210}{2}
    \end{array}
    \right\} = 85 \quad x_4 \leq 85
\]

\(\Rightarrow\) В производство можно вводить изделие \( C \) до 24 единиц или изделие \( D \) до 85 единиц.
\end{document}