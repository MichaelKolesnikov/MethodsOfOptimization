\documentclass[17pt]{extarticle}
\usepackage{../mystyle}

\begin{document}
\section{Геометрический способ решения двумерной задачи нелинейного программирования}
\[
    f(x_1, x_2) \to \min
\]
\[
    g_i(x_1, x_2) \leq 0, \quad i = 1, m
\]
Функция \( f(x) \) называется целевой функцией, а неравенства \( g_i(x) \leq 0 \), \( j = 1, \ldots, m \) называются ограничениями задачи.
Множество точек, удовлетворяющих ограничениям задачи, называется допустимым множеством задачи.

Решить задачу нелинейного программирования графически — значит найти такую точку из допустимого множества,
через которую проходит линия уровня \( f(x_1, x_2) = C \), имеющая максимальное значение величины \( C \) из всех линий уровня,
проходящих через допустимые точки задачи.

\begin{enumerate}
    \item Строится допустимое множество задачи. \\ На плоскости наносятся геометрические места точек,
          соответствующих каждому уравнению из ограничений задачи \( g_i(x_1, x_2) \leq 0 \), \( j = 1, \ldots, m \).
          Если допустимое множество задачи пусто, то задача не имеет решения.

    \item Строятся линии уровня целевой функции \( f(x_1, x_2) = C \) при различных значениях параметра \( C \).

    \item Определяется направление возрастания (для задачи максимизации) или убывания (для задачи минимизации) линий уровня целевой функции.

    \item Определяется точка допустимого множества, через которую проходит линия уровня с максимальным (для задачи максимизации)
          или минимальным (для задачи минимизации) значением параметра \( C \). Если целевая функция не ограничена сверху (для задачи максимизации)
          или не ограничена снизу (для задачи минимизации) на допустимом множестве, то задача не имеет решения.

    \item Для найденной точки определяют ее координаты \( x = (x_1, x_2) \) и значение целевой функции в данной точке.
\end{enumerate}

Наиболее существенное отличие задачи
нелинейного программирования от линейных задач
заключается в том, что оптимальное решение может
находиться как на границе допустимого множества,
так и являться его внутренней точкой.


\end{document}