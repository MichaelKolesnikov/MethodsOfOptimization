\documentclass[17pt]{extarticle}
\usepackage{../mystyle}

\begin{document}
\section{Транспортная задача. Постановка, \\ математическая модель. Свойства \\ классической ТЗ}
\subsection{Постановка задачи}
\begin{itemize}
    \item \( m \) — поставщиков однородной продукции (источников);
    \item \( n \) — потребителей однородной продукции (стоков);
    \item \( a_i \) — запасы \( i \)-го поставщика;
    \item \( b_j \) — потребности (спрос) \( j \)-го потребителя;
    \item \( c_{ij} \) — стоимость перевозки из пункта \( i \) в пункт \( j \);
    \item \( x_{ij} \) — количество груза, перевезённого из пункта \( i \) в пункт \( j \).
\end{itemize}

\subsection{Математическая модель}
\[
    \sum_{i=1}^{m} \sum_{j=1}^{n} c_{ij}x_{ij} \rightarrow \min
\]

\[
    \sum_{j=1}^{n} x_{ij} \leq a_i, \quad i = \overline{1, m}
\]

\[
    \sum_{i=1}^{m} x_{ij} \geq b_j, \quad j = \overline{1, n}
\]

\[
    x_{ij} \geq 0, \quad i = \overline{1, m}, \quad j = \overline{1, n}
\]

\subsection{Определение закрытой и открытой задач}
\begin{definition}
    Транспортная задача, в которой сумма запасов равна сумме потребностей, называется \textbf{закрытой}. В противном случае задача называется \textbf{открытой}.
\end{definition}

В случае, если транспортная задача является открытой, невозможно удовлетворить всех потребителей (если сумма потребностей больше суммы запасов) или вывезти все грузы от поставщиков (если сумма запасов больше, чем сумма потребностей).

\subsection{Классическая транспортная задача}
\begin{itemize}
    \item \( m \) — поставщиков однородной продукции (источников);
    \item \( n \) — потребителей однородной продукции (стоков);
    \item \( a_i \) — мощность \( i \)-го источника;
    \item \( b_j \) — мощность \( j \)-го стока;
    \item \( c_{ij} \) — стоимость перевозки из пункта \( i \) в пункт \( j \).
\end{itemize}

\[
    \sum_{i=1}^m \sum_{j=1}^n c_{ij} x_{ij} \rightarrow \min
\]

\[
    \sum_{j=1}^n x_{ij} = a_i, \quad i = \overline{1, m}
\]

\[
    \sum_{i=1}^m x_{ij} = b_j, \quad j = \overline{1, n}
\]

\[
    x_{ij} \geq 0, \quad i = \overline{1, m}, \quad j = \overline{1, n}
\]

\subsection{Приведение открытой ТЗ к закрытой}
1) Если сумма запасов больше суммы потребностей \( \left( \sum_{i=1}^m a_i > \sum_{j=1}^n b_j \right) \), то введём в таблицу ещё одного потребителя, потребность которого определим как
\[
    \sum_{i=1}^m a_i - \sum_{j=1}^n b_j.
\]
Так как грузы к новому потребителю (фиктивному) отправляться не будут, то и стоимость этих перевозок равна нулю, т.е. цены (тарифы) в новой строке будут равны 0.

2) Если сумма запасов меньше суммы потребностей \( \left( \sum_{i=1}^m a_i < \sum_{j=1}^n b_j \right) \), то вводим в таблицу ещё одного поставщика, запас груза у которого определим как
\[
    \sum_{j=1}^n b_j - \sum_{i=1}^m a_i.
\]
Цены в новом столбце проставим равными нулю из тех же соображений, что и в первом случае.

\subsection{Решение транспортной задачи}
\begin{itemize}
    \item Любая транспортная задача, как задача ЛП, может быть решена симплекс-методом. Однако специфика задач рассмотренного класса (каждая неизвестная входит лишь в два уравнения-ограничения, и коэффициенты при неизвестных в ограничениях равны единице) позволила выработать более эффективные вычислительные методы.
    \item Транспортную задачу можно представить с помощью сети, что позволяет использовать для их решения эффективные алгоритмы.
\end{itemize}

\begin{theorem}
    Необходимым и достаточным условием разрешимости транспортной задачи является равенство суммы запасов сумме потребностей.
\end{theorem}

Так как транспортная задача является задачей линейного программирования, то и методика нахождения оптимального решения остаётся той же:
\begin{itemize}
    \item находится первоначальный опорный план,
    \item проверяется на оптимальность, и если план не оптимален, то
    \item переход к другому опорному плану, улучшающему целевую функцию в смысле оптимума (а именно уменьшающую значение целевой функции).
\end{itemize}

Критерий отсутствия решения не требуется, так как решению подлежат лишь закрытые ТЗ.
\end{document}