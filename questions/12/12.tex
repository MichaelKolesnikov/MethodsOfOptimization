\documentclass[17pt]{extarticle}
\usepackage{../mystyle}

\begin{document}
\section{Экономическая интерпретация двойственных \\ задач}
\subsection{Задача об оптимальном плане производства продукции}
\begin{itemize}
    \item \( n \) — видов продукции, \( j = \overline{1, n} \);
    \item \( m \) — видов ресурсов (сырья), \( i = \overline{1, m} \);
    \item \( a_{ij} \) — количество ресурса \( i \)-го вида, требующегося для производства единицы продукции \( j \)-го вида;
    \item \( b_i \) — запасы ресурса \( i \)-го вида;
    \item \( c_j \) — доход (прибыль) от реализации единицы продукции \( j \)-го вида.
    \item Необходимо найти такой план производства продукции, при котором достигается максимальная прибыль,
          для реализации которого достаточно имеющихся ресурсов.
    \item Оценить каждый из видов сырья, используемых для производства продукции.
          Оценки, приписываемые каждому из видов сырья должны быть такими, чтобы оценка всего используемого сырья была минимальна,
          а суммарная оценка сырья, используемого на производство единицы продукции любого вида, - не меньше цены единицы продукции данного вида.
    \item Найти интервалы устойчивости двойственных оценок по отношению к изменениям ресурсов каждого типа
\end{itemize}

\begin{tabular}{|c|c|c|c|c|c|}
    \hline
                              & A & B & C & D & Запасы \\
    \hline
    $C_1$                     & 1 & 0 & 2 & 1 & 180    \\
    \hline
    $C_2$                     & 0 & 1 & 3 & 2 & 210    \\
    \hline
    $C_3$                     & 4 & 2 & 0 & 4 & 800    \\
    \hline
    Цена за единицу продукции & 9 & 6 & 4 & 7 &        \\
    \hline
\end{tabular} \\
Построим модели
\[
    F = 9x_1 + 6x_2 + 4x_3 + 7x_4 \rightarrow \max
\]
\[
    \begin{cases}
        x_1 + 2x_3 + x_4 \leq 180 \\
        x_2 + 2x_3 + x_4 \leq 210 \\
        4x_1 + 2x_2 + 4x_4 \leq 800
    \end{cases}, x_j \geq 0, \quad j = \overline{1,4}
\]
\[
    F^* = 180y_1 + 210y_2 + 800y_3 \rightarrow \min
\]
\[
    \begin{cases}
        y_1 + 4y_3 \geq 9  \\
        y_2 + 2y_3 \geq 6  \\
        2y_1 + 3y_2 \geq 4 \\
        y_1 + 2y_2 + 4y_3 \geq 7
    \end{cases}, y_i \geq 0, \quad i = \overline{1,3}
\]
Приведем к канонической форме
\[
    F = 9x_1 + 6x_2 + 4x_3 + 7x_4 \rightarrow \max
\]
\[
    \begin{cases}
        x_1 + 2x_3 + x_4 + x_5 = 180  \\
        x_2 + 3x_3 + 2x_4 + x_6 = 210 \\
        4x_1 + 2x_2 + 4x_4 + x_7 = 800
    \end{cases}, x_j \geq 0, \quad j = \overline{1,7}
\]
\begin{tabular}{|c|c|c|c|c|c|c|c|c|c|}
    \hline
          &     &       & 9   & 6  & 4  & 7  & 0  & 0  & 0  \\
    \hline
    базис & Сб. & В     & А1  & А2 & А3 & А4 & А5 & А6 & А7 \\
    \hline
    А5    & 0   & 180   & [1] & 0  & 2  & 1  & 1  & 0  & 0  \\
    \hline
    А6    & 0   & 210   & 0   & 1  & 3  & 2  & 0  & 1  & 0  \\
    \hline
    А7    & 0   & 800   & 4   & 2  & 0  & 4  & 0  & 0  & 1  \\
    \hline
          &     & F = 0 & -9  & -6 & -4 & -7 & 0  & 0  & 0  \\
    \hline
\end{tabular}
\begin{itemize}
    \item При данном плане ничего не производится, сырье не используется, F = 0.
    \item $\Delta_j$ показывают на сколько увеличится F (цена за произведенную продукцию) при введении в план единицы j-го вида продукции.
    \item Отсюда следует, что целесообразно включить в план изделие А в объеме $\min\{180/1, 800/4\}= 180$.
    \item Тогда сможем изготовить 180 единиц изделия А. На это потребуется 180 единиц C1 и 180 ∙ 4 C3.
    \item Т.е. максимум количества изделия А ограничивается запасами сырья C1. При этом все сырье C1 израсходуется.
\end{itemize}
Оптимальная симплекс-таблица
\begin{center}
    \begin{tabular}{|c|c|c|c|c|c|c|c|c|c|}
        \hline
        \textbf{базис} & \textbf{Сб.} & \textbf{В} & \textbf{9} & \textbf{6} & \textbf{4}       & \textbf{7} & \textbf{0} & \textbf{0}       & \textbf{0}       \\
        \hline
                       &              &            & \( A_1 \)  & \( A_2 \)  & \( A_3 \)        & \( A_4 \)  & \( A_5 \)  & \( A_6 \)        & \( A_7 \)        \\
        \hline
        \( A_1 \)      & 9            & 95         & 1          & 0          & \(-\frac{3}{2}\) & 0          & 0          & \(-\frac{1}{2}\) & \(\frac{1}{4}\)  \\
        \hline
        \( A_5 \)      & 0            & 85         & 0          & 0          & \(\frac{7}{2}\)  & 1          & 1          & \(\frac{1}{2}\)  & \(-\frac{1}{4}\) \\
        \hline
        \( A_2 \)      & 6            & 210        & 0          & 1          & 3                & 2          & 0          & 1                & 0                \\
        \hline
                       &              & 2115       & 0          & 0          & \(\frac{1}{2}\)  & 5          & 0          & \(\frac{3}{2}\)  & \(\frac{9}{4}\)  \\
        \hline
    \end{tabular}
\end{center}
\[
    x^* = (95, 210, 0, 0) \quad y^* = \left(0, \frac{3}{2}, \frac{9}{4}\right)
\]
При оптимальном плане производится 95 изделий \( A \), 210 изделий \( B \), при этом остаётся неиспользованными 85 единиц \( C_1 \).

1. Подставим \( x^* \) в ограничения прямой задачи:
\[
    \begin{cases}
        95 + 2 \cdot 0 + 0 < 180          \\
        210 + 3 \cdot 0 + 2 \cdot 0 = 210 \\
        4 \cdot 95 + 2 \cdot 210 + 4 \cdot 0 = 800
    \end{cases}
\]

Второе и третье ограничения выполняются как «\( = \)» \(\Rightarrow\) ресурсы 2-го и 3-го видов полностью используются в оптимальном плане,
являются дефицитными (\( y_2^* = \frac{3}{2} > 0, y_3^* > 0 \)).

Первое ограничение выполняется как строгое «\( < \)» \(\Rightarrow\) ресурс первого вида не является дефицитным (\( y_1^* = 0 \)).
Его остатки \( x_5^* = 85 \) \(\Rightarrow\) положительную двойственную оценку имеют лишь те виды ресурсов,
которые полностью используются в оптимальном плане.

2. Подставим $y^*$ в ограничение двойственной задачи
\[
    \begin{cases}
        0 + 4 \cdot \frac{9}{4} = 9                       \\
        \frac{3}{2} + 2 \cdot \frac{9}{4} = 6             \\
        2 \cdot 0 + 3 \cdot \frac{3}{2} > 4               \\
        0 + 2 \cdot \frac{3}{2} + 4 \cdot \frac{9}{4} > 7 \\
    \end{cases}
\]

Первое и второе ограничения выполняются как «\( = \)» \(\Rightarrow\) двойственные оценки ресурсов,
используемых для производства единицы продукции \( A \) и \( B \),
равны в точности доходам \(\Rightarrow\) производить эти изделия целесообразно \(\Rightarrow\) \( x_1^* = 95 > 0 \), \( x_2^* = 210 > 0 \).

Третье и четвёртое ограничения выполняются как «\( > \)» \(\Rightarrow\) производить изделия \( C \) и \( D \)
экономически невыгодно \(\Rightarrow\) \( x_3^* = 0 \), \( x_4^* = 0 \).

3. Величина двойственной оценки показывает, насколько возрастает значение целевой функции при увеличении дефицитного ресурса на одну единицу.

Увеличение ресурса \( C_2 \) на одну единицу приведёт к получению нового оптимального плана, в котором прибыль возрастает на \( \frac{3}{2} \):
\[
    2115 + \frac{3}{2}.
\]
При этом коэффициенты матрицы \( A^{-1}_B \) (столбца \( A_6 \)) оптимальной симплекс-таблицы показывают, что указанное увеличение прибыли достигается за счёт:
\begin{itemize}
    \item уменьшения выпуска изделий \( A \) на \( \frac{1}{2} \) единицы,
    \item увеличения выпуска изделия \( B \) на 1 единицу,
    \item увеличения остатка ресурса \( C_1 \) на \( \frac{1}{2} \) единицы (использование ресурса \( C_1 \) сократится на \( \frac{1}{2} \) единицы).
\end{itemize}

Увеличение ресурса \( C_3 \) на 1 единицу приведёт к получению нового оптимального плана, в котором прибыль возрастает на \( \frac{9}{4} \):
\[
    2115 + \frac{9}{4}.
\]
Это произойдёт за счёт:
\begin{itemize}
    \item увеличения выпуска изделия \( A \) на \( \frac{1}{4} \) единицы,
    \item при этом расход сырья \( C_1 \) возрастает (остаток уменьшится) на \( \frac{1}{4} \) единицы.
\end{itemize}

Двойственные оценки связаны с оптимальным
планом прямой задачи. \\ Всякое изменение исходных
данных прямой задачи оказывает влияние на ее
оптимальный план и на систему двойственных
оценок.
В свою очередь двойственные оценки служат
инструментом анализа и принятия правильного
решения в условиях меняющихся коммерческих
ситуаций.
\end{document}