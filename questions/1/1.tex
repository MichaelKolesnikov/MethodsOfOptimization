\documentclass[17pt]{extarticle}
\usepackage{../mystyle}

\begin{document}
\section{Задача принятия решений. Типы задач принятия решений}

Есть некая среда. В этой среде есть управляющая и управляемая подсистемы.
Управляющая подсистема посылает управляющие воздействия(УВ) \\ управляемой подсистеме.

\begin{enumerate}
      \item Управляющая подсистема может воздействовать на объект управления с помощью альтернативных управляющих воздействий: УВ1, УВ2, УВn
      \item Состояние Объекта управления определяется двумя факторами: \\ выбранным УВ; состоянием среды
      \item Принципиальным является следующее обстоятельство:
            Управляющая подсистема не может воздействовать на среду, более того, она, как правило, не имеет полной информации о состоянии среды
      \item Цель Управляющей подсистемы: перевести объект управления в наиболее предпочтительное для себя состояние,
            используя для этого любое УВ, имеющееся в ее распоряжении
      \item Выбор Управляющей подсистемой конкретного УВ (допустимой альтернативы, допустимого решения)
            называется \textbf{принятием решения}.
\end{enumerate}

При принятии решения основной задачей является нахождение оптимального решения.
На содержательном уровне оптимальное решение можно определить как наилучшее в следующем смысле:
оптимальное решение в наибольшей степени соответствует цели
Управляющей подсистемы в рамках имеющейся у нее информации о состоянии среды.

В зависимости от информации,
которую имеет при принятии решения \\ управляющая подсистема относительно состояния среды,
различают несколько основных типов задач принятия решения.

\begin{enumerate}
      \item \textit{Принятие решения в условиях определенности} характеризуется тем, \\
            что состояние среды является фиксированным (неизменным), причем управляющая система «знает», в каком состоянии находится среда.

      \item \textit{Принятие решения в условиях риска} означает,
            что управляющая подсистема имеет информацию стохастического характера о поведении среды
            (например, ей известно распределение вероятностей на множестве состояний среды).

      \item \textit{Принятие решения происходит в условиях неопределенности},
            если никакой дополнительной информации (кроме знания самого множества возможных состояний среды) управляющая подсистема не имеет.

      \item \textit{Принятие решения в теоретико-игровых условиях} имеет место тогда,
            когда среду можно трактовать как одну или несколько целенаправленных управляющих подсистем.
            В этом случае математическая модель принятия решения называется \textbf{теоретико-игровой моделью} (игрой).
\end{enumerate}

\end{document}