\documentclass[17pt]{extarticle}
\usepackage{../mystyle}

\begin{document}
\section{Решение транспортной задачи методом потенциалов}

\subsection{Двойственная задача}

Рассмотрим двойственную задачу для транспортной задачи:

\[
    G = (u_1, \ldots, u_m, v_1, \ldots, v_n) = \sum_{i=1}^m a_i u_i + \sum_{j=1}^n b_j v_j \rightarrow \max
\]
с ограничениями:$u_i + v_j = c_{ij}$, где переменные \( u_i, v_j \) не ограничены в знаке.

Из второй теоремы двойственности следует:
\[
    (u_i^* + v_j^* - c_{ij}) \cdot x_{ij}^* = 0
\]
что эквивалентно:
\[
    u_i^* + v_j^* = c_{ij} \quad \forall x_{ij}^* \neq 0 \tag{1}
\]

\[
    u_i^* + v_j^* \leq c_{ij} \quad \forall x_{ij}^* = 0 \tag{2}
\]

\subsection{Идея решения транспортной задачи}
\begin{itemize}
    \item На каждой итерации решения транспортной задачи для текущего опорного решения исходной задачи получают одно из соответствующих решений двойственной задачи, используя соотношения (1).
    \item Далее, для него осуществляют проверку условий (2).
    \item Если условия выполнены, то текущее опорное решение транспортной задачи является оптимальным.
    \item Иначе осуществляется переход к новому (лучшему) опорному решению, в котором значение целевой функции будет лучше (меньше), чем в предыдущем.
\end{itemize}

Для решения транспортной задачи необходимо:
\begin{itemize}
    \item Находить опорное решение транспортной задачи.
    \item Иметь правило перехода к новому опорному решению.
    \item Критерий отсутствия решения не требуется.
\end{itemize}

\subsection{Пример решения}
\[
    \begin{cases}
        u_1 + v_1 = 2 \\
        u_1 + v_2 = 3 \\
        u_1 + v_3 = 4 \\
        u_2 + v_3 = 1 \\
        u_2 + v_4 = 4 \\
        u_3 + v_4 = 7 \\
        u_3 + v_5 = 2
    \end{cases}
    \begin{cases}
        u_1 = 0  \\
        v_1 = 2  \\
        v_2 = 3  \\
        v_3 = 4  \\
        u_2 = -3 \\
        v_4 = 7  \\
        u_3 = 0, v_5 = 2
    \end{cases}
    \begin{cases}
        u_i = c_{ij} - v_j \\
        v_j = c_{ij} - u_i \\
    \end{cases}
\]
\subsection{Проверка на оптимальность}
Проверяем выполнение условий оптимальности:
\[
    u_i + v_j \leq c_{ij} \Rightarrow d_{ij} = c_{ij} - u_i - v_j \geq 0
\]
\[
    d_{12} = 8 - (-3) - 2 = 9 \geq 0
\]
\[
    d_{13} = 9 - 0 - 2 = 7 \geq 0
\]
\[
    d_{22} = 5 - (-3) - 3 = 5 \geq 0
\]
\[
    d_{23} = 8 - 0 - 3 = 5 \geq 0
\]
\[
    d_{33} = 4 - 0 - 4 = 0 \geq 0
\]
\[
    d_{41} = 2 - 0 - 7 = -5 < 0
\]
\[
    d_{51} = 4 - 0 - 2 = 2 \geq 0
\]
\[
    d_{52} = 1 - (-3) - 2 = 2 \geq 0
\]

Так как \( d_{41} = -5 < 0 \), клетка (4,1) является клеткой пересчета.

\subsection{Переход к новому опорному решению}

Для улучшения плана строим цикл пересчета. Циклом пересчета в таблице транспортной задачи называется ломаная линия, вершины которой находятся в заполненных клетках, в клетке пересчета она имеет начало и конец, а звенья располагаются вдоль строк и столбцов таблицы.

Новый план получаем следующим образом: в клетку пересчета записывается наименьшая из величин поставок, стоящих в минусовых клетках. Одновременно это число вычитается из величин поставок «-» клеток и прибавляется к величинам поставок «+» клеток.

\subsection{Пример нового плана}

\begin{longtable}{|c|c|c|c|c|}
    \hline
          & 1              & 2               & 3             & Запасы        \\
    \hline
    1     & $2^{60}$       & 8               & 9             & \( V_1 = 2 \) \\
    \hline
    2     & $3^{70}$       & 5               & 8             & \( V_2 = 3 \) \\
    \hline
    3     & \( 4^{10-W} \) & \( 1^{110+W} \) & 4             & \( V_3 = 4 \) \\
    \hline
    4     & \( 2^W \)      & \( 4^{70-W} \)  & $7^{60}$      & \( V_4 = 7 \) \\
    \hline
    5     & 4              & 1               & $2^{100}$     & \( V_5 = 2 \) \\
    \hline
    Спрос & \( u_1 = 0 \)  & \( u_2 = -3 \)  & \( u_3 = 0 \) & 480           \\
    \hline
\end{longtable}

\subsection{Проверка нового плана на оптимальность}
Проверяем выполнение условий (2) для незаполненных клеток:
\[
    d_{12} = 8 - 2 - 2 = 4 \geq 0
\]
\[
    d_{13} = 9 - 5 - 2 = 2 \geq 0
\]
\[
    d_{22} = 5 - 2 - 3 = 0 \geq 0
\]
\[
    d_{23} = 8 - 5 - 3 = 0 \geq 0
\]
\[
    d_{31} = 4 - 0 + 1 = 5 \geq 0
\]
\[
    d_{33} = 4 - 5 - (-1) = 0 \geq 0
\]
\[
    d_{51} = 4 - 0 - (-3) = 7 \geq 0
\]
\[
    d_{52} = 1 - 2 - (-3) = 2 \geq 0
\]

Условия (2) выполнены, следовательно, текущее опорное решение является оптимальным.
\[
    F^* = 1330
\]
\[
    G^* = 0 \cdot 140 + 2 \cdot 180 + 5 \cdot 160 + 60 \cdot 2 + 70 \cdot 3 - 120 + 2 \cdot 130 - 3 \cdot 100 = 1330
\]
\end{document}