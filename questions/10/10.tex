\documentclass[17pt]{extarticle}
\usepackage{../mystyle}

\begin{document}
\section{Метод искусственного базиса поиска \\ начальной опорной точки.}
Будем считать, что для задачи
\(
\langle c, x \rangle \rightarrow \max, D: Ax = b, \, x \geq 0
\)
выполнено условие \( b \geq 0 \).

Рассмотрим вспомогательную задачу:
\[
    \sum_{i=1}^m y_i \rightarrow \max \quad (2)
\]
\[
    \tilde{D}: Ax + y = b, \, x \geq 0, \, y \geq 0
\]
\[
    \tilde{D} = \biggl\{(x, y) \in \mathbb{R}^{n+m} : \sum_{j=1}^n A_j x_j + \sum_{i=1}^m e_i y_i = b, \, x \geq 0, \, y \geq 0\biggr\}
\]
\[
    \Rightarrow
    \begin{cases}
        (\vec{0}, b) - \text{опорная точка в } D.
    \end{cases}
\]

Целевая функция вспомогательной задачи ограничена сверху нулем (0). Следовательно, эта задача имеет оптимальное решение.
\[
    G = -y_1 - y_2 - \cdots - y_m \rightarrow \max
\]
\[
    \begin{cases}
        a_{11}x_1 + \cdots + a_{1n}x_n + y_1 = b_1        \\
        a_{21}x_1 + \cdots + a_{2n}x_n + 0y_1 + y_2 = b_2 \\
        \vdots                                            \\
        a_{m1}x_1 + \cdots + a_{mn}x_n + 0y_1 + \cdots + y_m = b_m
    \end{cases} \quad x_j \geq 0, \quad j = 1, \ldots, n
\]

Переменные $y_i \geq 0, \quad i = 1, \ldots, m$ называют искусственными переменными.

Очевидно, что векторы
\[
    A_{n+1} = \begin{pmatrix} 1 \\ 0 \\ 0 \\ \vdots \\ 0 \end{pmatrix}, \quad
    A_{n+2} = \begin{pmatrix} 0 \\ 1 \\ 0 \\ \vdots \\ 0 \end{pmatrix}, \quad
    \cdots, \quad
    A_{n+m} = \begin{pmatrix} 0 \\ 0 \\ \vdots \\ 1 \end{pmatrix}
\]
образуют базис для опорного решения \( x = (0, 0, \ldots, 0, b_1, b_2, \ldots, b_m) \), который называют искусственным базисом.

\begin{proposition}
    Пусть \( (x^*, y^*) \) — решение задачи (2) и
    \[
        f^* = -\sum_{i=1}^m y_i^*
    \]

    Если \( f^* = 0 \), то \( x^* \) — опорная точка множества \( D \).
    Если \( f^* < 0 \), то задача (1) не имеет допустимых точек: множество \( D \) — пусто.
\end{proposition}
\begin{proof}
    1) Если \( f^* = 0 \), \( y^* = 0 \), то \( (x, 0) \) — опорная точка множества \( \tilde{D} \) и \( D \), следовательно,
    оптимальный базис вспомогательной задачи можно взять в качестве начального для задачи (1).

    2) Если \( f^* < 0 \), то, если \( \exists x \in D \Rightarrow \exists(x, 0) \in \tilde{D} \),
    что несовместимо с условием \( f^* < 0 \Rightarrow \) задача (1) не имеет область допустимых решений.

    Решая вспомогательную задачу симплекс-методом, мы найдем оптимальное решение
    \[
        x^{(0)} = (x_1^{(0)}, \cdots, x_n^{(0)}, y_1^{(0)}, \cdots, y_m^{(0)}).
    \]

    Если в этом решении среди искусственных переменных есть положительные,
    то исходная задача линейного программирования неразрешима, так как её ОДР (область допустимых решений) пуста.

    Если же \( y_i^{(0)} = 0, \, i = 1, \ldots, m \), то базис,
    соответствующий оптимальному решению вспомогательной задачи, можно взять в качестве исходного базиса основной задачи.
\end{proof}

\begin{remark}
    Проблема зацикливания
    \begin{itemize}
        \item Вырожденные планы могут привести к зацикливанию, \\
              т.е. к многократному повторению процесса вычислений, не позволяющему получить оптимальный план.
        \item Можно использовать метод Крено: Элементы строк, имеющие одинаковые наименьшие симплексные отклонения, делятся на предполагаемые разрешающие элементы. За ведущую выбирается та сторона, в которой раньше встретится наименьшее частное при просмотре слева направо по столбцам.
    \end{itemize}

    Бесчисленное множество решений
    \begin{itemize}
        \item Если в строке \(\Delta\) оптимального плана находится нуль, принадлежащий свободной переменной, вектор которой не входит в базис, а в столбце этого вектора имеется хотя бы один положительный элемент, то задача имеет бесчисленное множество решений.
        \item Свободные переменные можно ввести в базис, в результате будет получен новый оптимальный план с другим набором базисных переменных.
    \end{itemize}
\end{remark}

\end{document}