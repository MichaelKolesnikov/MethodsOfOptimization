\documentclass[17pt]{extarticle}
\usepackage{../mystyle}

\begin{document}
\section{Свойства задачи линейного \\ программирования}

\begin{definition}
    Множество \( D \) — точек \( n \)-мерного евклидова пространства будем называть выпуклым, если
    \[
        \forall x^{(1)} = (x_1^{(1)}, \ldots, x_n^{(1)}), x^{(2)} = (x_1^{(2)}, \ldots, x_n^{(2)})
        \quad \forall \alpha \geq 0, \beta \geq 0 \colon \alpha + \beta = 1
    \]
    точка \(x = \alpha x^{(1)} + \beta x^{(2)}\) также принадлежит \( D \).
\end{definition}

\begin{definition}
    Вершиной выпуклого множества в \( \mathbb{R}^n \) назовём такую точку, которую нельзя представить в виде
    \(
    x = \alpha x^{(1)} + \beta x^{(2)}, \quad \alpha > 0, \beta > 0 \colon \alpha + \beta = 1
    \)
    ни при каких \(x^{(1)}, x^{(2)}\).
\end{definition}

\subsection{Свойства ЗЛП}
\begin{enumerate}
    \item Допустимая область ЗЛП выпукла, если она не пуста.
    \item Если допустимая область имеет вершины и ЗЛП имеет решение, то оно достигается по крайней мере в одной из вершин.
    \item Множество решений ЗЛП выпукло.
    \item Если допустимая область ограничена, то ЗЛП имеет оптимальное решение.
    \item Необходимым и достаточным условием существования решения ЗЛП на максимум(минимум) является
          ограниченность целевой функции сверху(соответственно снизу) в допустимой области.
\end{enumerate}

\end{document}