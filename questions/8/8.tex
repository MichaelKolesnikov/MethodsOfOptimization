\documentclass[17pt]{extarticle}
\usepackage{../mystyle}

\begin{document}
\section{Основная идея симплексного метода решения задачи ЛП при известном допустимом базисном решении.}
Свойства задачи линейного программирования наталкивают на следующую схему решения задачи линейного программирования,
известную, \\
как симплекс-метод.
Пусть рассматриваемая задача имеет непустое допустимое множество с вершинами.
Тогда тем или иным способом находим какую-нибудь вершину допустимого
множества и по определенным критериям определяем, не является ли она
оптимальной.
Если она оптимальна, то задача решена. Если нет, то, используя определенные правила, проверяем,
нельзя ли утверждать, что задача не имеет оптимального решения (целевая функция не ограничена
сверху или, соответственно, снизу на допустимом множестве).
Если утверждать это можно, то задача неразрешима. Если нельзя, то по определенному правилу ищем новую,
лучшую (в смысле значения целевой функции) вершину и переходим к пункту 1.
Для реализации предложенной схемы необходимо:
\begin{itemize}
    \item указать способ нахождения вершины допустимого множества,
    \item критерии оптимальности, неразрешимости,
    \item способ перехода от одной вершины к другой, лучшей в смысле значения целевой функции.
\end{itemize}

Идея симплекс-метода состоит в том, чтобы исходя из начального опорного решения найти новое опорное
решение, исключая для этого некоторый вектор $A_s, s \in \overline{1,m}$ из начального базиса и заменяя его
одним из небазисных векторов $A_r, r \in \overline{m + 1, n}$
таким образом, чтобы новое опорное решение не ухудшало значения целевой функции.
\end{document}