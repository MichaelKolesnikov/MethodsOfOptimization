\documentclass[17pt]{extarticle}
\usepackage{../mystyle}

\begin{document}
\section{Опорные точки допустимого множества. \\ Вырожденные и невырожденные опорные точки.
  Базис невырожденной опорной точки. \\ Теорема о связи опорной точки и вершины ОДР}

\[f = C_1 X_1 + C_2 X_2 + \cdots + C_n X_n \rightarrow \max, x = (x_1, \ldots, x_n) \colon x_i \geq 0, i=\overline{1,n} \]
\[
    \begin{cases}
        a_{11}x_1 + a_{12}x_2 + \cdots + a_{1n}x_n = b_1 \\
        a_{21}x_1 + a_{22}x_2 + \cdots + a_{2n}x_n = b_2 \\
        \vdots                                           \\
        a_{m1}x_1 + a_{m2}x_2 + \cdots + a_{mn}x_n = b_m
    \end{cases}
\]

Введем в рассмотрение векторы
\[
    A_j = \begin{pmatrix}
        a_{1j} \\
        a_{2j} \\
        \vdots \\
        a_{mj}
    \end{pmatrix}, j=\overline{1,n}
\]
\[
    f = (c, x) \rightarrow \max
\]
\[
    A_1 x_1 + A_2 x_2 + \dots + A_n x_n = b, x \geq 0
\]
Задачу ЛП можно трактовать следующим образом: из всех разложений вектора b по
векторам $A_1 \dots A_n$ с неотрицательными коэффициентами требуется
выбрать хотя бы одно такое, коэффициенты $x_j, j=\overline{1,n}$,
которого доставляют целевой функции f оптимальное значение.
Не ограничивая общности, считаем ранг матрицы $A$ равным $m$ и $n>m$ (случай $n=m$ тривиален).

\begin{definition}
    Ненулевое допустимое решение \( x = (x_1, \ldots, x_n) \) называется \textbf{опорным},
    если векторы \( A_j \), соответствующие отличным от нуля координатам вектора \( x \), линейно независимы.
\end{definition}

\begin{definition}
    Ненулевое опорное решение назовем \textbf{невырожденным}, если оно имеет точно \( m \) положительных координат.
\end{definition}

\begin{definition}
    Если число положительных координат опорного решения меньше \( m \), то оно называется \textbf{вырожденным}.
\end{definition}

\begin{definition}
    Упорядоченный набор из \( m \) линейно независимых векторов \( A_j \), соответствующих положительным координатам опорного решения,
    назовем \textbf{базисом}.
\end{definition}

\begin{theorem}[Теорема о связи опорного решения и вершины допустимого множества]
    Вектор \( x = (x_1, \ldots, x_n) \) тогда и только тогда является опорным решением задачи, когда точка \( x = (x_1, \ldots, x_n) \) является вершиной допустимого множества.
\end{theorem}

Таким образом, задача нахождения вершины допустимого множества свелась к задаче нахождения опорного решения, а, следовательно, к нахождению базиса.

Будем считать, что исходный базис \( A_1, A_2, \ldots, A_m \) дан.
Отправляясь от него, покажем, как найти опорное решение.
Сформулируем условие оптимальности решения, условие отсутствия решения.
Покажем, как перейти к базису, дающему лучшее решение.

\end{document}