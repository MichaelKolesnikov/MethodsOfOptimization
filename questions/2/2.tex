\documentclass[17pt]{extarticle}
\usepackage{../mystyle}

\begin{document}
\section{Линейное программирование. Постановка \\ задачи планирования производства, составление смеси (о диете).}
\subsection{Линейное программирование}

\begin{definition}
    Задача, состоящая в нахождении наибольшего (наименьшего) значения целевой функции
    \(
    f = C_1 X_1 + C_2 X_2 + \cdots + C_n X_n \xrightarrow{?} \max(\min)
    \)
    на множестве точек \( x = (x_1, \ldots, x_n) \), удовлетворяющих системе ограничений вида
    \[
        \begin{cases}
            a_{11}x_1 + a_{12}x_2 + \cdots + a_{1n}x_n \, R_1 \, b_1 \\
            a_{21}x_1 + a_{22}x_2 + \cdots + a_{2n}x_n \, R_2 \, b_2 \\
            \vdots                                                   \\
            a_{m1}x_1 + a_{m2}x_2 + \cdots + a_{mn}x_n \, R_n \, b_m
        \end{cases}
    \]
    называется \textbf{задачей линейного программирования общего вида}.
    Здесь:
    \begin{itemize}
        \item \( R_i, i = \overline{1, m} \) — один из знаков \( =, \geq, \leq \);
        \item \( C_j, j = \overline{1, n} \) и \( a_{ij}, i = \overline{1, m}, j = \overline{1, n} \) — заданные константы.
    \end{itemize}
\end{definition}

\begin{definition}
    Всякую точку \( X = (x_1, \ldots, x_n) \), компоненты которой удовлетворяют системе ограничений,
    будем называть допустимой точкой или допустимым решением задачи, или допустимым планом задачи.
\end{definition}

Задача линейного программирования состоит в нахождении такой допустимой точки \( x \) (такого допустимого плана)
среди множества допустимых точек,
при которой целевая функция принимает \textbf{max(min)} значение.

\begin{definition}
    Допустимое решение \( x^{(0)} = (x_1^{(0)}, \ldots, x_n^{(0)}) \),
    доставляющее целевой функции оптимальное значение (оптимум),
    будем называть оптимальным решением или оптимальным планом задачи.
\end{definition}

\begin{definition}
    Задача об оптимальном плане производства продукции
    \begin{itemize}
        \item $n$ видов продукции, $j = \overline{1, n}$
        \item $m$ видов ресурсов (сырья), $i = 1, m;$
        \item $a_{ij}$ - количество ресурса вида $i$, требующегося для производства единицы продукции вида $j$
        \item $b_i$ - запасы ресурса вида $i$
        \item $c_j$ - доход (прибыль) от реализации единицы продукции вида $j$
    \end{itemize}
    Необходимо найти такой план производства продукции, при котором \\ достигается максимальная прибыль,
    для реализации которого достаточно имеющихся ресурсов. Задача $f \rightarrow \max, \quad R_i \leq, \quad x_j \geq 0$
\end{definition}


\begin{definition}[Задача о диете (исторически одна из самых первых)]
    Условия:
    \begin{itemize}
        \item \( n \) — видов кормов, \( j =\overline{1, n} \);
        \item \( m \) — видов питательных веществ, \( i =\overline{1, m} \);
        \item \( a_{ij} \) — содержание \( i \)-го вида питательного вещества в единице \( j \)-го вида корма;
        \item \( b_i \) — необходимый минимум \( i \)-го питательного вещества в день, \( i=\overline{1,m} \);
        \item \( c_j \) — стоимость единицы \( j \)-го вида корма.
    \end{itemize}

    Необходимо составить рацион, имеющий минимальную стоимость,
    в котором содержание каждого вида питательных веществ было бы не менее установленного предела.
    $f \rightarrow \min, \quad x_j \geq 0, \quad R_i \geq$
\end{definition}


\end{document}