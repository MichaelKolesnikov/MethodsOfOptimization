\documentclass[17pt]{extarticle}
\usepackage{../mystyle}

\begin{document}
\section{Классическая задача нелинейного программирования. Метод множителей Лагранжа.
  Необходимые условия экстремума. \\ Необходимые условия экстремума второго порядка.
  Достаточные условия экстремума.}

\subsection*{Классическая задача на условный экстремум}

\subsubsection*{Постановка задачи}

\[
    f(x) \rightarrow \min \quad (4)
\]

\[
    g_i(x) = 0, \quad i = 1, m \quad (5)
\]

Эта задача — частный случай задачи (1-3), т.к. каждое \( g_i(x) = 0 \) можно заменить двумя неравенствами:
\[
    g_i(x) \leq 0, \quad g_i(x) \geq 0, \quad i = 1, m.
\]

В свою очередь ограничения (2) можно представить:
\[
    g_i(x) + Z_i^2 = 0, \quad i = 1, m.
\]

\subsection*{Метод множителей Лагранжа}

Задача условной оптимизации сводится к задаче безусловной оптимизации функции Лагранжа, составленной для этой задачи.

Обобщенная функция Лагранжа для задачи (4-5):

\[
    L(x, y_0, y) = y_0 f(x) + \sum_{i=1}^m y_i g_i(x), \quad y_i \in \mathbb{R}, \quad i = 1, m + 1
\]

\( y_i \) — множители Лагранжа.

Классическая функция Лагранжа для задачи (4-5):

\[
    L(x, y) = f(x) + \sum_{i=1}^m y_i g_i(x), \quad y_i \in \mathbb{R}, \quad i = 1, m
\]

\begin{theorem}[Теорема 1 (Необходимое условие оптимальности)]
    Пусть функции \( f, g_1, \ldots, g_m \) — непрерывно дифференцируемые в некоторой окрестности точки \( x^* \in \mathbb{R}^n \).
    Если \( x^* \) — точка локального экстремума задачи (4-5), то \(\exists\) числа \( y_0^*, y_1^*, \ldots, y_m^* \),
    не равные нулю одновременно и такие, что выполнены следующие условия:
    \begin{enumerate}
        \item Условие стационарности обобщенной функции Лагранжа по \( x \):
              \[
                  \frac{\partial L}{\partial x_j} \left( x^*, y_0^*, y^* \right) = 0, \quad j = 1, n
              \]

        \item Условие допустимости решения: \(g_i \left( x^* \right) = 0, \quad i = 1, m\)
    \end{enumerate}

    Если при этом градиенты \( g_i' \left( x^* \right) \),
    \( i = 1, m \) в точке \( x^* \) линейно независимы (выполняется условие регулярности), то \( y_0^* \neq 0 \).
\end{theorem}

\begin{theorem}[Необходимое условие экстремума 2-го порядка]
    Пусть функции \( f(x), g_1(x), \ldots, g_m(x) \) — дважды непрерывно дифференцируемы \\
    в точке \( x^* \in \mathbb{R}^n \) и непрерывно дифференцируемы в ее некоторой окрестности, \\
    причем градиенты \(g'_1(x^*), \ldots, g'_m(x^*)\) - линейно независимы. \\
    Если \( x^* \) — локальное решение задачи (4-5) (регулярная точка экстремума), то
    \[
        L''_{xx}(x^*, y^*) h, \quad h > 0 \quad \forall \quad y^*, \quad \text{удовлетворяющих}
    \]
    \[
        L'_x(x^*, y^*) = 0, \quad g'_i(x^*) = 0, \quad i = 1, m \quad \forall \quad h \in \mathbb{R}^n:
    \]
    \[
        \langle g'_i(x^*), h \rangle = 0, \quad i = 1, m
    \]
\end{theorem}

\begin{theorem}[Теорема 3 (Достаточные условия экстремума)]
    Пусть функции \( f, g_1, \ldots, g_m \) дважды дифференцируемы в точке \(x^* \in \mathbb{R}^n\),
    удовлетворяющей условиям допустимости $g_i(x^*) = 0, i = \overline{1, m} \ \& \ \exists y^*$:
    \begin{enumerate}
        \item $L'_x(x^*, y^*) = 0$
        \item $\langle L''_{xx}(x^*, y^*) h, h \rangle > 0 \quad \forall h \neq 0 \in \mathbb{R}^n$,
              для которых $\langle g_i'(x^*), h \rangle = 0, i = 1, m$
    \end{enumerate}
    то \( x^* \) — строгое локальное решение задачи (4-5).
\end{theorem}


\end{document}