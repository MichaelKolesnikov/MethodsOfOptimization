\documentclass[17pt]{extarticle}
\usepackage{../mystyle}

\begin{document}
\section{Методы северо-западного угла и минимальных элементов поиска начального опорного решения ТЗ}

\subsection{Метод северо-западного угла}
1) В верхнюю левую клетку (северо-западный угол) таблицы записываем наименьшее из чисел \( b_1 \) и \( a_1 \),
пересчитываем запасы и потребности,
и столбец с исчерпанным запасом или строку с удовлетворённой потребностью исключаем из дальнейшего расчёта.

В оставшейся части таблицы снова находим северо-западный угол, \\ заполняем эту клетку,
вычёркиваем строку или столбец и опять обращаемся к северо-западному углу и т.д.

Важнейшим условием построения опорного плана является назначение в выбранной клетке наибольшей возможной перевозки.

\subsection{Пример метода северо-западного угла}
\begin{center}
    \begin{tabular}{|c|c|c|c|c|}
        \hline
              & 1   & 2   & 3   & Запасы \\
        \hline
        1     & 2   & 8   & 9   & 60     \\
        \hline
        2     & 3   & 5   & 8   & 70     \\
        \hline
        3     & 4   & 1   & 4   & 120    \\
        \hline
        4     & 2   & 4   & 7   & 130    \\
        \hline
        5     & 4   & 1   & 2   & 100    \\
        \hline
        Спрос & 140 & 180 & 160 & 480    \\
        \hline
    \end{tabular}
\end{center}

Проверим, является ли задача закрытой.

\subsection{Пример заполнения таблицы методом северо-западного угла}
\begin{center}
    \begin{tabular}{|c|c|c|c|c|}
        \hline
              & 1            & 2            & 3             & Запасы \\
        \hline
        1     & \( 2^{60} \) & 8            & 9             & 60     \\
        \hline
        2     & \( 3^{70} \) & 5            & 8             & 70     \\
        \hline
        3     & \( 4^{10} \) & \( 1^{10} \) & 4             & 120    \\
        \hline
        4     & 2            & \( 4^{70} \) & \( 7^{60} \)  & 130    \\
        \hline
        5     & 4            & 1            & \( 2^{100} \) & 100    \\
        \hline
        Спрос & 140          & 180          & 160           & 480    \\
        \hline
    \end{tabular} $F=1380$
\end{center}

\subsection{Метод минимальных элементов}
2) Клетки ТЗ заполняются по такому же принципу, как в методе \\ северо-западного угла,
но в первую очередь заполняются клетки с минимальной стоимостью поставки.

\subsection{Теоремы о транспортной задаче}
\begin{theorem}
    Число положительных компонентов в опорном плане (число заполненных клеток в таблице) меньше или равно \( m + n - 1 \).
\end{theorem}

\begin{proof}
    В процессе построения опорного плана на каждом шаге заполняли одну клетку таблицы. При этом либо потребности, либо запасы в соответствующей строке или столбце становятся равными нулю (либо оба вместе). При заполнении последней клетки одновременно удовлетворялись спрос потребителя и исчерпывались запасы поставщика \(\Rightarrow\) число заполненных клеток максимум \( m + n - 1 \).

    Если в процессе построения плана встретится клетка (кроме последней), после заполнения которой запасы и потребности столбца и строки становятся равными нулю, то число неизвестных будет меньше \( m + n - 1 \).
\end{proof}

\begin{theorem}
    Если для транспортной задачи выполнены условия \( a_i \in \mathbb{N}_0 \), \( b_j \in \mathbb{N}_0 \), \( \mathbb{N}_0 = \{0, 1, \ldots\} \), то в любом её допустимом базисном решении базисные переменные принимают значения из \( \mathbb{N}_0 \).
\end{theorem}
\end{document}